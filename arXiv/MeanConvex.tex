% !TeX program = xelatex
\documentclass[12pt]{article}
\usepackage[unicode]{phyxmeow}
\geometry{a4paper,margin=1in}
\allowdisplaybreaks{}

\title{Mean Convex Region}
\author{Haotian Xue}

\begin{document}
\maketitle

\section{Introduction}
\begin{definition}[Mean convex]
  Let \(M\) be a Riemannian manifold, a smooth region \(\Omega\subset M\) is called \emph{mean convex}
  if the mean curvature vector of boundary \(H_{\partial\Omega}\) is zero or pointing inwards everywhere.
\end{definition}

\begin{theorem}
  Let \(\Omega\) be a smooth mean convex region in \(\mathbb{R}^{n+1}\) whose boundary \(\partial\Omega\) has at
  least two components. Then on each boundary component, \(|H|\) does not have a positive
  lower bound.
\end{theorem}

\section{Some facts about conformal change of metric}
In this section, suppose \((M,g)\) is a \(n+1\) dimensional Riemannian manifold, \(u\) is a smooth positive 
function on \(M\). Let \(f=-\log u\) and \(\tilde{g}=e^{2f}g=u^{-2}g\).

It's well-known that the Levi-Civita connection of \(\tilde{g}\) is given by 
\begin{equation}
  \widetilde{\nabla}_X Y=\nabla_X Y+(\nabla_X f)Y+(\nabla_Y f)X-g(X,Y)\nabla^{\sharp} f
,\end{equation}
and the Ricci curvature of \(\tilde{g}\) is given by
\begin{equation}
  \widetilde{\Ric}(\tilde{e}_j,\tilde{e}_j)=u^2 \Ric(e_j,e_j)+u\lap u+(n-1)uu_{jj}-n|\nabla u|^2
,\end{equation}
where all the differentiation on the right hand side respects to \(g\),
\(\nabla^{\sharp}f=g^{ij}(\nabla_j f) \partial_j\) is the gradient of \(f\).

Let \(\Sigma\subset M\) be a smooth hypersurface with induced metric, \(\{e_0,e_1,\ldots,e_{n}\}\) be an
orthonormal basis of \(g\) at some point on \(\Sigma\) where \(e_0\) is the normal direction. Let
\(\tilde{e}_i=u e_i\) be orthonormal under \(\tilde{g}\).

The mean curvature vector of \(\Sigma\) under \(\tilde{g}|_{\Sigma}\) is given by
\begin{align}
  \tilde{H}&=\sum_{i=1}^n (\tilde{\nabla}_{\tilde{e}_i}\tilde{e}_i)^\perp 
  =u^2 \sum_i (\tilde{\nabla}_{e_i}e_i)^\perp \\
  &=u^2\sum_{i}(\nabla_i e_i+2\nabla_i f e_i-\nabla^\sharp f)^\perp \\
  &=u^2\cdot (H-n \nabla^\sharp f)^\perp
,\end{align}
and the scalar mean curvature is given by 
\begin{equation}
  |\tilde{H}|_{\tilde{g}}=|\tilde{e}_0\cdot_{\tilde{g}}\tilde{H}|=u|(H-n\nabla^\sharp f)^\perp|
.\end{equation}

Let \(\gamma\) be a geodesic with endpoints \(p,q\) in \(M\) under metric \(\tilde{g}\). Let \(T\) be the unit
tangent of \(\gamma\) under \(g\) and \(\tilde{T}=uT\) be the unit normal under \(\tilde{g}\). The geodesic
equation of \(\gamma\) is then 
\begin{align}
  0&=\tilde{\nabla}_{\tilde{T}}\tilde{T}=\nabla_{uT}(uT)+2(\nabla_{uT}f)uT-|uT|^2\nabla^{\sharp} f \\
  &=u^2\nabla_T T-u(\nabla_T u)T-u\nabla^{\sharp} u
.\end{align}
Define the \(g\)-curvature and unit normal of \(\gamma\) by \(\nabla_T T:=\kappa N\). If \(\kappa=0\),
\(N\) could be chosen as any unit vector perpendicular to \(T\). Then
\begin{equation}\label{}
  0=(\kappa u-\nabla_N u)uN-(\nabla^{\sharp}u-(\nabla_T u)T-(\nabla_N u)N)u
.\end{equation}
For simplicity, write \(u_T=\nabla_T u\) and \(u_N=\nabla_N u\), we conclude that
\begin{equation}\label{}
  \kappa=u^{-1}u_N
,\end{equation}
and the gradient of \(u\), \(\nabla^{\sharp}u\) lies in the two dimensional subspace spanned by \(\{T,N\}\).
\begin{remark}
  The claim is still true when \(\kappa=0\), where \(\nabla^\sharp u\) has align with \(T\). In this case 
  \(u_N=0\) for whatever \(N\) chosen to be. 
\end{remark}

Finally, let \(X\) be a variation field along \(\gamma\) the second variation of length of \(\gamma\) is given
by 
\begin{equation}\label{}
  I(X,X)=\tilde{T}(q)\cdot_{\tilde{g}}\tilde{\nabla}_X X-\tilde{T}(p)\cdot_{\tilde{g}}\tilde{\nabla}_X X
  +\int_{\gamma}|(\tilde{\nabla}_{\tilde{T}}X)^\perp|^2-\tilde{R}(X,\tilde{T},\tilde{T},X)\dd{\tilde{s}}
,\end{equation}
where \(\dd{\tilde{s}}\) is the arclength parameter under \(\tilde{g}\).

\section{Proof of the theorem}
Suppose \(\Omega\subset M^{n+1}\) has two boundary components \(\Sigma_1\) and \(\Sigma_2\). Suppose \(u\) is 
compactly supported. Then there exists a shortest geodesic \(\gamma\) between \(\Sigma_1\) and \(\Sigma_2\)
inside \(\op{supp}u\) since the metric goes to infinity on the boundary of \(\op{supp}u\). Write the
endpoints of \(\gamma\) by \(p\in \Sigma_1\) and \(q\in \Sigma_2\). Note that
\begin{enumerate}[(1)]
  \item {\color{red}\(\gamma\) lies fully inside \(\Omega\) if \(M\) is simply connected.}
  \item \(\gamma\) is perpendicular to \(\Sigma_i\)'s at the endpoints. 
\end{enumerate}
Take \(X\) to be a variation field that is \(\tilde{g}\)-parallel along \(\gamma\),
then take trace of \(I(X,X)\) for \(X(p)\) running over an orthonormal basis of \(T_p \Sigma_1\). We have 
\begin{align}
  \op{tr} I&=-\tilde{T}(p)\cdot_{\tilde{g}}\tilde{H}_1+\tilde{T}(q)\cdot_{\tilde{g}}\tilde{H}_2
  -\int_{\gamma}\op{tr}_X \tilde{R}(X,\tilde{T},\tilde{T},X)\dd{\tilde{s}} \\
  &=-uT(p)\cdot (H_1-n\nabla^\sharp f)+uT(q)\cdot (H^2-n\nabla^\sharp f)
  -\int_{\gamma}\widetilde{\Ric}(\tilde{T},\tilde{T})\dd{\tilde{s}} \\
  &=-u(p)|H_1(p)|-u(q)|H_2(q)|+n(\nabla_T u(q)-\nabla_T u(p)) \\
  &\phantom{=}+\int_{\gamma}-u^2\Ric(T,T)+n|\nabla u|^2-u\lap u-(n-1)uu_{TT}\dd{\tilde{s}}
.\end{align}
Denote by \(\hat{\nabla}\) be the pull-back connection of \(g\) on \(\gamma\),
\begin{align}
  n(\nabla_T u(q)-\nabla_T u(p))&=n\int_{\gamma}\hat{\nabla}_T \nabla_T u \dd{s}
  =n\int_{\gamma}\hat{\nabla}_T (\nabla u\cdot T)\dd{s} \\
  &=n\int_{\gamma}\nabla^2_{T,T}u+\nabla u\cdot(\kappa N)\dd{s} \\
  &=\int_{\gamma}nuu_{TT}-n|u_N|^2\dd{\tilde{s}}
.\end{align}
So we have 
\begin{equation}\label{traced-variation}
  \op{tr}I=-u(p)|H_1(p)|-u(q)|H_2(q)|+\int_{\gamma}-u^2\Ric(T,T)+n|u_T|^2-(\lap u-u_{TT})u\dd{\tilde{s}}
.\end{equation}

Now suppose \(u=u(r)\) for some function \(r\) on \(M\) (think \(r\) to be the distance function).
Note that \(\nabla u\) will have the same direction as \(\nabla r\), the integral in \cref{traced-variation}
will become 
\begin{align}\label{substitute-r}
  \int_{\gamma}-u \Ric(T,T)+n\frac{(u')^2}{u}|r_T|^2-u'\lap r-u''(r)|r_N|^2+u'(r)r_{TT}\dd{s}
.\end{align}  

\textbf{Case 0:} \(M=\mathbb{R}^n\), \(g\) is Euclidean metric.

Take \(r\) be the Euclidean distance, we have 
\begin{equation}\label{}
\lap r=\frac{n}{r}, \quad |\nabla r|=1, \quad \text{and } r_{TT}=\frac{1-|r_{T}|^2}{r}=\frac{|r_N|^2}{r}
.\end{equation}
\Cref{substitute-r} becomes 
\begin{equation}\label{}
  \int_{\gamma}n\left(\frac{(u')^2}{u}|r_T|^2-\frac{u'}{r}\right)-\left(\frac{u'}{r}\right)'r|r_N|^2\dd{s}
.\end{equation}
Let \(u(r)=(1-\frac{r^2}{R^2})^2\), easy to verify \((u'/r)'\ge 0\), and using \(|r_T|\le 1\),
\begin{equation}\label{}
  \frac{(u')^2}{u}|r_T|^2-\frac{u'}{r}\le \frac{(u')^2}{u}-\frac{u'}{r}\le 16R^{-2}
.\end{equation}
Suppose \(|H_1|\ge c>0\), we have estimate on \cref{traced-variation} 
\begin{equation}\label{Rn-simplified}
  0\le \op{tr}I\le -u(p)c+16nR^{-2}\int_{\gamma}u\dd{\tilde{s}}
.\end{equation}
Consider letting \(R\to\infty\), choose a fixed curve \(\gamma_0\) in \(\Omega\) connecting \(\Sigma_1\) and 
\(\Sigma_2\). By the minimality of \(\gamma\) and \(u\le 1\) we get
\begin{equation}\label{}
  \int_{\gamma}u\dd{\tilde{s}}\le\int_{\gamma}1\dd{\tilde{s}}=\tilde{L}(\gamma)\le \tilde{L}(\gamma_0)
  =\int_{\gamma_0}\frac{\dd{s}}{u}=O(1)
.\end{equation}
For the last equality we used the fact that when \(R\to \infty\), \(u\) is bounded below on any compact set.
Hence \(u(p)=O(R^{-2})\). Let \(a=R-|p|\), then 
\begin{equation}\label{}
  a=\frac{R^2-|p|^2}{R+|p|}=\frac{R^2\sqrt{u(p)}}{R+|p|}=O(1)
.\end{equation}
For points \(O(aR^{-1})\)-close to \(p\), we have 
\begin{equation}\label{}
  u=R^{-4}(R^2-|x|^2)^2\le 4R^{-2}(R-|x|)^2=4R^{-2}(a+O(aR^{-1}))^2=O(a^2R^{-2})
.\end{equation}
Note that
\begin{equation}\label{}
  |\gamma(t)-p|\le \int_0^t 1\dd{s}=\int_0^t u\dd{\tilde{s}}
.\end{equation}
Suppose \(\gamma\) reaches out of ball \(B(p,aR^{-1})\), let \(t_1\) be the smallest \(t\) it gets out. Then 
\begin{equation}\label{}
  aR^{-1}=|\gamma(0)-\gamma(t_1)|\le \int_{0}^{t_1}1\dd{s}=\int_{0}^{t_1}u\dd{\tilde{s}}
  \le O(a^2R^{-2})\tilde{L}(\gamma)=O(a^2R^{-2})
.\end{equation}
This is a contradiction, so \(\gamma\) stays within \(B(p,aR^{-1})\). Then \(u=O(a^2R^{-2})\) uniformly
on \(\gamma\) and hence \(|p-\gamma(t)|=O(a^2R^{-2})\) for all \(t\).

Finally, we use \cref{Rn-simplified} again to get 
\begin{align}
  0&\le -u(p)c+16nR^{-2}\int_{\gamma}u\dd{\tilde{s}} \\
  &=-ca^2R^{-4}(2R-a)^2+R^{-2}\int_{\gamma}O(a^2R^{-2})\dd{\tilde{s}} \\
  &\le-ca^2R^{-2}+O(a^2R^{-4}) \\
  &=(-c+O(R^{-2}))a^2R^{-2}
.\end{align}
For \(R\) sufficiently large, this gives a contradiction. Hence \(c\) cannot be positive and the theorem 
is proved.

\end{document}

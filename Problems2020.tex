% !TeX program = xelatex
\documentclass[12pt]{article}
\usepackage[cambria,cjk]{phyxmeow}
\geometry{a4paper,margin=1in}
\addbibresource{../MyRefer.bib}
\allowdisplaybreaks{}

\title{Problems with Solution}
\author{ \(\Delta\) }

\begin{document}
\maketitle

\section{}
\begin{prop}
  Calculate \[
    \int_{\mathbb{R}}\frac{e^{-x^2}}{(x^2+\frac{1}{2})^2}\,dx
  .\]
\end{prop}
\begin{lemma}
  \[
    \int_{\mathbb{R}}f\hat{g}=\int_{\mathbb{R}}\hat{f}g
  .\]
  Where \( \hat{f}(\xi)=\int_{\mathbb{R}}f(z)e^{-2\pi iz\xi}\,dz\) is the Fourier transform.
\end{lemma}
\noindent For detailed discussion abort Fourier transform, see\cite{stein_fourier_2003}, Chapter 5.

More Generally, we will calculate \[ \int_{\mathbb{R}}\frac{e^{-x^2}}{(x^2+a^2)^2}\,dx.\]
Which, by lemma, equals
\[
  \int_{\mathbb{R}}e^{-x^2}\hat{f}(x)\,dx,\quad\text{where }f(x)=\frac{1}{(x^2+a^2)^2}
.\]
Thus we should calculate the Fourier integral
\[
  \int_{\mathbb{R}}\frac{e^{-2\pi i z\xi}}{(z^2+a^2)^2}\,dz
.\]
% 
Suppose \(\xi\le 0\), take \(\gamma\) as the contour consists of \([-R,R]\) and a semi-circle with radius \(R\) in upper-half plane, \(R\) sufficiently large.
According to the residue formula, integration along \(\gamma\) equals \(2\pi i\) times the residue at \(z=ai\).
Let \(R\to\infty\), it's easy to see that on the semi-circle, the numerator remains bounded and the denominator is \(O(R^2)\). So the integral along the semi-circle tends to \(0\).\\
For information abort residue formula, see~\cite{ahlfors_complex_1979}, Chapter 4.
% 
Then we calculate the residue, the integrand has an order \(2\) pole at \(ai\), so the residue equals
\[
  \left.\frac{\partial}{\partial z}\frac{e^{-2\pi iz\xi}}{(z+ai)^2}\right|_{z=ai}=\frac{e^{2\pi a\xi}(1-2\pi a\xi)}{4a^2i}
.\]
For \(\xi\ge 0\), same method applies to the lower-semi-circle. Finally we have
\[
  \int_{\mathbb{R}}\frac{e^{-2\pi i z\xi}}{(z^2+a^2)^2}\,dz=\frac{\pi e^{-2\pi ia|\xi|}(1+2\pi a|\xi|)}{2a^2}
.\]
% 
Next, we must calculate
\[
  \int_{\mathbb{R}}e^{-x^2}\frac{\pi e^{-2\pi a|x|}(1+2\pi a|x|)}{2a^2}\,dx
.\]
Which is nothing more than
\[
  \int_0^\infty e^{-x^2-Cx}\,dx\quad \text{and} \quad \int_0^\infty xe^{-x^2-Cx}\,dx
.\]
Use the fact that
\[
  \int_{\mathbb{R}}e^{-x^2}=\sqrt{\pi}
.\]
We have
\[
  \int_{\mathbb{R}}\frac{e^{-x^2}}{(x^2+a^2)^2}=\frac{\sqrt{\pi}}{a^2}+\frac{\sqrt{\pi} (1-2a^2)e^{a^2}}{a^3}\int_a^\infty e^{-x^2}\,dx
.\]
Let \(a^2=\frac{1}{2}\), the original integral equals \(2\sqrt{\pi} \).

\section{}
\begin{prop}
  Suppose
  \begin{gather*}
    \gamma_t(s):[0,1]\times[0,1]\to\mathbb{R}^3
    \qquad v_t(s)=\frac{\partial\gamma_t}{\partial t}\\
    A:\mathbb{R}^3\to\mathbb{R}^3 \qquad B=\nabla\times A\\
    \Phi_t=\int_{\gamma_t}A\cdot dx
    =\int_0^1A\cdot\frac{\partial\gamma_t}{\partial s}\,ds\\
  \end{gather*}
  Prove
  \[
    \frac{\partial \Phi_t}{\partial t}
    = \int_{\gamma_t}(B\times v_t)\cdot dx
    = \int_0^1 (B\times v_t)\cdot\frac{\partial\gamma_t}{\partial s}\,ds
  .\]
\end{prop}
\begin{proof}
  \begin{equation}\label{2:LHS}
    \begin{aligned}
      \text{LHS} =& \frac{\partial}{\partial t}\int_0^1 A\cdot \frac{\partial\gamma_t}{\partial s}\,ds \\
      =& \int_0^1 \frac{\partial A}{\partial t}\cdot\frac{\partial \gamma}{\partial s}+A\cdot \frac{\partial^2 \gamma}{\partial t\partial s}\,ds \\
      =& \int_0^1 \frac{\partial A}{\partial t}\cdot\frac{\partial \gamma}{\partial s}-\frac{\partial A}{\partial s}\cdot\frac{\partial \gamma}{\partial t}\,ds \\
    \end{aligned}
  \end{equation}
  Where
  \begin{equation*}
    \begin{aligned}
      \frac{\partial A}{\partial t}=& A'(x)x'(t)=A'\frac{\partial \gamma}{\partial t} \\
      =& \begin{bmatrix}\displaystyle \frac{\partial A_i}{\partial x_j} \end{bmatrix}_{3\times3}\begin{bmatrix}\displaystyle \frac{\partial \gamma_i}{\partial t} \end{bmatrix}_{3\times1} \\
      =& \begin{bmatrix}\displaystyle \sum_j \frac{\partial A_i}{\partial x_j} \frac{\partial \gamma_j}{\partial t} \end{bmatrix}_{3\times1}
    \end{aligned}
  \end{equation*}
  So
  \[
    \frac{\partial A}{\partial t} \cdot \frac{\partial \gamma}{\partial s}
    =\sum_i\sum_j \frac{\partial A_i}{\partial x_j} \frac{\partial \gamma_j}{\partial t} \frac{\partial \gamma_i}{\partial s}
  .\]
  And the integrand in \cref{2:LHS} equals
  \begin{equation}
    \sum_i\sum_j(\frac{\partial A_i}{\partial x_j} \frac{\partial \gamma_j}{\partial t} \frac{\partial \gamma_i}{\partial s}-\frac{\partial A_i}{\partial x_j} \frac{\partial \gamma_j}{\partial s} \frac{\partial \gamma_i}{\partial t})\label{intd}
  \end{equation}
  On the other hand, by formally using the triple-product formula
  \[
    a\times (b\times c)=(a\cdot c)b-(a\cdot b)c
  .\]
  We have
  \begin{equation}
    \begin{aligned}
      B\times v_t=& -v_t\times (\nabla \times A) \\
      =& (v_t\cdot \nabla)A-(v_t\cdot A)\nabla\quad\text{(formally)} \\
      =& \sum_j \frac{\partial \gamma_j}{\partial t} \frac{\partial}{\partial x_j}\begin{bmatrix} \displaystyle A_i \end{bmatrix}_{3\times 1}-\sum_j \frac{\partial \gamma_j}{\partial t}A_j \begin{bmatrix}\displaystyle \frac{\partial}{\partial x_i} \end{bmatrix}_{3\times 1} \\
      =& \begin{bmatrix} \displaystyle \sum_j ( \frac{\partial \gamma_j}{\partial t} \frac{\partial A_i}{\partial x_j} - \frac{\partial \gamma_j}{\partial t} \frac{\partial A_j}{\partial x_i} ) \end{bmatrix}_{3\times 1} \\
    \end{aligned}
  \end{equation}
  So the integrand in RHS equals
  \[
    \sum_i\sum_j (\frac{\partial \gamma_j}{\partial t} \frac{\partial A_i}{\partial x_j} - \frac{\partial \gamma_j}{\partial t} \frac{\partial A_j}{\partial x_i} )\frac{\partial \gamma_i}{\partial s}
  .\]
  Obviously this equals to \cref{intd}, and the proof is complete.
\end{proof}

\section{}
\[
  \sum_{n=0}^{\infty} (-1)^n\frac{\sin(2n+1)\theta}{(2n+1)^2}=\frac{\pi\theta}{4}
.\]
\[
  \sum_{n=0}^{\infty} (-1)^n\frac{z^{2n+1}}{(2n+1)^2}
.\]

\section{}
\[
  \int_0^\pi \frac{\sqrt{a+b\cos t}}{a^2-b^2} \,dt=\int_0^\pi \frac{1}{\sqrt{(a+b\cos t)^3} }\,dt
.\]
\[
  \zeta(s)\Gamma(s)=\int_0^\infty \frac{x^{s-1}}{e^x-1}\,dx
.\]
\begin{gather*}
  a_0=0,a_n=\frac{1}{n}-(-1)^n \frac{\log 2}{2^n}\\
  b_n=(-1)^n
\end{gather*}

\[
  |\sin(n)| \le \pi\Vert \frac{n}{\pi} \Vert
.\]
\[
  F:\mathbb{R}^4\to \mathbb{R}^2 \quad M=F^{-1}(0)
.\]
\[
  \frac{\partial x_i}{\partial x_j}=-\frac{\frac{\partial F_1}{\partial x_{i'}} \frac{\partial F_2}{\partial x_j} -\frac{\partial F_1}{\partial x_j} \frac{\partial F_2}{\partial x_{i'}} }{\frac{\partial F_1}{\partial x_{i'}} \frac{\partial F_2}{\partial x_i} -\frac{\partial F_1}{\partial x_i} \frac{\partial F_2}{\partial x_{i'}} }
.\]
Where \(x_i\) is considered as a function of \((x_j,x_{j'})\).

\[
  (\int_X p^2\,d\mu)(\int_X x^2 p^2\,d\mu)\ge \lambda
.\]
\[
  \int_0^{2\pi} |D_N|\,dx=\frac{1}{2N+1}+\frac{2}{\pi}\sum_{k=1}^N \frac{1}{k}\tan \frac{k\pi}{2N+1}=\frac{4}{\pi^2}\log N+O(1)
.\]
\[
  \frac{1}{\frac{\pi}{2}-x}-\frac{2}{\pi}\le \tan x\le \frac{1}{\frac{\pi}{2}-x}-\frac{2}{\pi}+x
.\]
\[
  (1-s^2)P_{m+2}^\ell-2(m+1)sP_{m+1}^\ell+(\ell-m)(\ell+m+1)P_m^\ell=0
.\]
\[
  (1-s^2) L_n^{(m+2)}-2(m+1)sL_n^{(m+1)}+(n-m)(n+m+1)L_n^{(m)}=0
.\]
\[
  \int_0^{2\pi}\log(1-2r\cos\theta+r^2)\cos n\theta\dd{\theta}
.\]
\[
  \int_{\mathbb{R}}\frac{1}{\cosh\pi t}\frac{y}{y^2+(x-t)^2}\,dt
.\]
\[
  f(t)=t^6 \int_0^\infty \frac{x^{n-1}}{1+x^n}e^{-itx}\,dx
.\]
\[
  \sum_{n\ge\geqslant 1} \frac{e^{inx}}{n}
.\]
\[
  \int_{0}^{\infty}\frac{\cos x-\cos x^2}{x}\dd{x}
  \quad\text{Answer:} -\frac{\gamma}{2}
.\]
\[
  \frac{\sum_i a_i}{n} \le \sqrt{\frac{\sum_i a_i^2}{n+1}}
.\]
\[
  \frac{\partial u}{\partial x}= \frac{y^2+2xy-x^2}{r^4}
.\]
\[
  \int_{-\infty}^\infty \frac{e^{ix}}{x^2+a^2}\,dx \quad \int_{-\infty}^\infty \frac{xe^{ix}}{x^2+a^2}\,dx
.\]
\[
  x^2- \frac{\pi^2}{3}=2\sum_{n\neq 0} \frac{1}{n^2}e^{inx+in\pi}
.\]
\[
  \int_{-\infty}^\infty\int_{-\infty}^\infty
  \frac{2(1+a^2)x}{(1+x^2+y^2)^3}\log((a-x)^2+y^2)
  \,\mathrm{d}x\mathrm{d}y
.\]
\[
  \int_{-\infty}^{\infty}e^{-ix\xi}\dd\xi
.\]

\[
  N_p(\vphi)=\sup_{|\alpha|,|\beta|\le p}|x^\alpha\partial^\beta\vphi|_{L^\infty}
.\]
\[
  \frac{n!}{(n_1)^{k_1}\cdots (n_r)^{k_r}}\cdot \frac{1}{k_1!\cdots k_r!}
.\]
\begin{align*}
  G(x) & =\frac{1}{1-x}\cdot \frac{1}{1-x^2}\cdot \frac{1}{1-x^3}\cdots        \\
       & = (1+x+x^2+x^3+\cdots )(1+x^2+(x^2)^2+\cdots )(1+x^3+(x^3)^2+\cdots )
.\end{align*}
\[
  \frac{1}{1-x^m}=\sum_{k\ge 0}(x^m)^k \to \sum_{k\ge 0}\frac{1}{m^k\cdot k!}(x^m)^k
  =e^{\frac{x^m}{k}}
.\]
\[
  \frac{1}{1-x}=e^{\frac{x}{1}}\cdot e^{\frac{x^2}{2}}\cdot e^{\frac{x^3}{3}}\cdots
  =1+x+x^2+x^3+\cdots
.\]
\[
  \sum_{k\ge 0}\frac{k^2}{1^k\cdot k!}(x^1)^k=x(x(e^x)')'=(x^2+x)e^x
.\]
\[
  a_n=\left.(x^2+x)e^x\cdot e^{\frac{x^2}{2}}\cdot e^{\frac{x^3}{3}}\cdots\right|_{x^n}
  =\left.\frac{x^2+x}{1-x}\right|_{x^n}=2\text{ when }n\ge 2
.\]
\[
  \nabla_i X^j=\pdv{X^j}{x^i}+\Gamma_{ip}^j X^p
.\] 

\printbibliography{}
\end{document}

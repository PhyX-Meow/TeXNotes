% !TEX program = xelatex
\documentclass[UTF8,12pt]{article}
\usepackage[unicode]{phyxmeow}

\geometry{a4paper,scale=0.8}
\allowdisplaybreaks

\begin{document}
	Let $A_n = \sqrt{1+2\sqrt{1+3\sqrt{\cdots\sqrt{1+n}}}}$.\\
	We will prove that $A_n$ has a limit and the limit is $3$.
	First, we have estimation
	\begin{equation}
		\begin{aligned}
			A_n&<\sqrt{3\sqrt{1+3\sqrt{\cdots\sqrt{1+n}}}}\\
			&<\sqrt{3\sqrt{4\sqrt{\cdots\sqrt{1+n}}}}\\
			&=(\prod_{k=3}^{n+1}k^{2^{-k}})^4.
		\end{aligned}
	\end{equation}
	Therefore, \[
		\log A_n < 4 \sum_{k=3}^{n+1}\frac{\log k}{2^k}
	.\]
	It's easy to show that when $n \to \infty$, the upper bound converges, so as $A_n$.\\
	Let $A$ be the limit.\\
	Define $\{a_n\}$ as \[
		a_1=A,\quad a_{n}=\frac{a_{n-1}^2-1}{n}
	.\]
	Suppose $A > 3$, say $A > 3+\epsilon$, where $\epsilon$ is a small positive number.\\
	If $a_{n-1} > n+1+\epsilon_{n-1}$, by the recursion formula, \[
		a_n=\frac{a_{n-1}^2-1}{n}>\frac{(n+1+\epsilon_{n-1})^2-1}{n}>n+2+2\epsilon_{n-1}
	.\]
	We can estimate that $a_n > n+2+2^{n-1}\epsilon$.\\
	However, exist $N$ s.t. when $n>N$, $A_n>A-\frac{\epsilon}{2}$ by the definition of limit.\\
	So if we use $A_n,n>N$ as $a_1$, then $1 = a_n > n+2+2^{n-2}\epsilon$, which is a contradiction.\\
	Suppose $A < 3$, we can similarly get $a_n < n+2-2^{n-1}\epsilon$.\\
	Since $2^{n-1}\epsilon$ grows much faster than $n+2$, a sufficiently large $n$ will make $a_n$ negative, 
	which, is again a contradiction.
\end{document}

% !TeX program = xelatex
\DocumentMetadata{pdfversion=1.7}
\documentclass[UTF8,12pt]{article}
\usepackage[concrete]{phyxmeow-common}
\usepackage{comment}
\geometry{a4paper,margin=1in}
\linespread{1.2}
\allowdisplaybreaks{}

\theoremstyle{plain}\newtheorem{theorem}{Theorem}
\theoremstyle{definition}\newtheorem{definition}[theorem]{Definition}
\theoremstyle{definition}\newtheorem{example}[theorem]{Example}
\theoremstyle{plain}\newtheorem{axiom}[theorem]{Axiom}
\theoremstyle{plain}\newtheorem{assertion}[theorem]{Assertion}
\theoremstyle{plain}\newtheorem{corollary}[theorem]{Corollary}
\theoremstyle{plain}\newtheorem{lemma}[theorem]{Lemma}
\theoremstyle{plain}\newtheorem{proposition}[theorem]{Proposition}
\theoremstyle{plain}\newtheorem{prop}[theorem]{Proposition}
\theoremstyle{plain}\newtheorem{conjecture}[theorem]{Conjecture}
\theoremstyle{plain}\newtheorem{conj}[theorem]{Conjecture}
\theoremstyle{plain}\newtheorem{problem}[theorem]{Problem}
\theoremstyle{remark}\newtheorem{notation}[theorem]{Notation}
\theoremstyle{definition}\newtheorem*{question}{Question}
\theoremstyle{definition}\newtheorem*{answer}{Answer}
\theoremstyle{definition}\newtheorem*{goal}{Goal}
\theoremstyle{plain}\newtheorem*{application}{Application}
\theoremstyle{plain}\newtheorem*{exercise}{Exercise}
\theoremstyle{remark}\newtheorem*{remark}{Remark}
\theoremstyle{remark}\newtheorem*{note}{\small{Note}}
\numberwithin{equation}{section}
\numberwithin{theorem}{section}
\numberwithin{figure}{section}

\DeclareMathOperator{\Exp}{Exp}
\DeclareMathOperator{\Hess}{Hess}
\newcommand{\Eps}{\mathcal{E}}

\title{Sacks \& Uhlenbeck's Minimal Sphere Immersion}
\author{Xue Haotian}
\date{June 2023}

\begin{document}
\maketitle

\begin{comment}
\section{Preliminaries}
A priori \(L^p\) estimate and Schauder estimate for elliptic equation:
\begin{theorem}[\(L^p\) estimate]
    Let \(\Omega\) be open set in \(\mathbb{R}^n\) and \(u\in W_{\text{loc}}^{2,p}
    (\Omega)\cap L^p(\Omega)\), \(1<p<\infty\). Assume \(u\) is a strong solution
    of equation \[
        Lu=a^{ij}u_{ij}+b^i u_i+cu=f
    .\] Suppose the coefficients satisfy
    \begin{itemize}
    \item \(a^{ij}\in C^0(\Omega)\), \(b^i,c\in L^\infty(\Omega)\), \(f\in
        L^p(\Omega)\);
    \item \(a^{ij}\xi_i\xi_j\ge \lambda|\xi|^2\), for some \(\lambda>0\),
        \(\forall\,x\in \Omega\xi\in \mathbb{R}^n\);
    \item \(|a^{ij}|,|b^i|,|c|\le M\) for some \(M>0\).
    \end{itemize}
    Then for any \(\Omega'\subset \subset \Omega\), \[
        \|u\|_{W^{2,p}(\Omega')}\le C(\|u\|_{L^p(\Omega)}+\|f\|_{L^p(\Omega)})
    .\] 
\end{theorem}
\begin{theorem}[Schauder estimate]
    Let \(\Omega\) be an open set in \(\mathbb{R}^n\), \(u\in C^{2,\alpha}(\Omega)\)
    be a bounded solution of equation \[
        Lu=a^{ij}u_{ij}+b^i u_i+cu=f
    .\] Suppose the coefficients satisfy
    \begin{itemize}
    \item \(a^{ij}\xi_i\xi_j\ge \lambda|\xi|^2\), for some \(\lambda>0\),
        \(\forall\,x\in \Omega,\xi\in \mathbb{R}^n\);
    \item \(\|a^{ij}\|_{C^{0,\alpha}(\Omega)}^{(0)},\|b^i\|_{C^{0,\alpha}(\Omega)}
        ^{(1)},\|c\|_{C^{0,\alpha}(\Omega)}^{(2)}\le M\) for some \(M>0\).
    \end{itemize}
    Then \[
        \|u\|_{C^{2,\alpha}(\Omega)}^*\le C(\|u\|_{C^0(\Omega)}
        +\|f\|_{C^{0,\alpha}(\Omega)}^{(2)})
    .\] 
\end{theorem}
Reference: Gilbarg-Trudinger, thm 9.11 and thm 6.2.
\end{comment}

\section{Introduction and First Results}
Through out this note, let \(M\) denote a compact orientable Riemannian 2-manifold with
given conformal structure, \(N\) a compact Riemannian manifold, with dimension \(\ge
2\). We always consider an isometric embedding \(N\hookrightarrow\mathbb{R}^k\) for
some \(k\), the embedding exists by Nash embedding theorem. Denote by \(\dd{\mu_g}\)
the volume form on \(M\).\ \(u\) is always assumed to be a map from \(U\subset M\) to
\(N\), where \(U\) is some open subset of \(M\). Sometimes we also use letter \(u\) to
denote its composition with \(N\hookrightarrow\mathbb{R}^k\).

\begin{definition}
    A map \(u\in W^{1,2}(M,N)\) is called \emph{weakly harmonic} if it is critical
    point of Dirichlet energy \[
        E(u)=\int_{M}|\nabla u|^2\dd{\mu}
    .\] One can easily derive the Euler-Lagrange equation that \(u\) satisfy: \[
        \lap^{M\to N} u=0
    .\] Where in local coordinates, \[
        \lap_g^{M\to N} u=g^{ij}\left(\pdv{u^\gamma}{x^i}{x^j}
        -\Gamma_{ij}^k\pdv{u^\gamma}{x^k}
        +\Gamma_{\alpha\beta}^\gamma\pdv{u^\alpha}{x^i}\pdv{u^\beta}{x^j}\right)
        \pd{\xi^\gamma}
    .\] \(u\) is considered to satisfy the equation in \(\mathcal{D}'\) sense.
\end{definition}
Let \(\II^N\) be the second fundamental form of \(N\hookrightarrow\mathbb{R}^k\),
let \(A=\op{tr}_g\otimes\II^N\). Then if \(u\) is viewed as \(M\to N\hookrightarrow
\mathbb{R}^k\), the equation writes as \[
    -\lap_g^{M\to\mathbb{R}^k} u+A(\nabla u,\nabla u)=0
.\] Note that the second fundamental form eats 2 tangent vector on \(N\) and gives
a normal vector of \(N\) inside \(\mathbb{R}^k\).

For simplicity, we denote \(\lap\) to be \(\lap_g^{M\to \mathbb{R}^k}\),
\(\lap'\) to be \(\lap_g^{M\to N}\) from now on.

\begin{lemma}
    \(E\) is conformal invariant.
\end{lemma}
\begin{proof}
    If \(\tilde{g}=\lambda g,\lambda\in C^\infty(M)\), then
    \begin{align*}
        \tilde{E}(u)&=\int_{M}|\nabla u|_{\tilde{g}}^2\dd{\tilde{\mu}}
        =\int_{M}\sum_\alpha\tilde{g}^{ij}u^{\alpha}_i u^\alpha_j
        \sqrt{\det\tilde{g}}\dd{x} \\
        &=\int_{M}\sum_\alpha \frac{1}{\lambda}g^{ij}u_i^\alpha u_j^\alpha
        \sqrt{\lambda^{\dim M}\det g}\dd{x} \\
        &=\int_{M}|\nabla u|^2\dd{\mu_g}
    .\end{align*}
\end{proof}
\begin{remark}
    This fact is heavily relying on \(\dim M=2\), and will play a central role in
    may theorem later. So there is a fundamental difficulty to generalize the results
    to higher dimension.
\end{remark}

For weakly harmonic maps, we have regularity theorem:
\begin{theorem}
    If \(u\) is weakly harmonic and continuous, then \(u\in C^\infty(M,N)\). Such map
    \(u\) is called harmonic.
\end{theorem}
\begin{proof}
    We for now refer the reader to thm 9.4.1 of Jost's book \emph{Riemannian
    Geometry}.
\end{proof}

\begin{lemma}
    If \(u\) is a harmonic conformal immersion, then \(u(M)\) is a minimal
    immersed surface.
\end{lemma}
\begin{proof}
    Choose metric on \(M\) such that \(u\) is isometric.\ \(u\) is still critical since
    \(E\) is conformal invariant. Then the lemma follows by definition of a minimal 
    surface.
\end{proof}

Note that on Riemannian surfaces, conformal structures can be identified with complex
structures. Give \(M\) a complex structure compatible to the conformal structure,
with coordinate \(z=x+iy\). Define quadratic differential
\begin{equation}\label{eq:quad-diff}
    \phi=\left(|u_x|^2-|u_y|^2-2i\left<u_x,u_y\right> \right)\dd{z}^2
.\end{equation}
\(u\) is called \emph{weakly conformal} if \(\phi\equiv 0\). One can easily verify
\(\phi\equiv 0\) if \(u\) is conformal.
\begin{lemma}
    If \(u\) is harmonic, then \(\phi\) is holomorphic.
\end{lemma}
\begin{proof}
    Only need to prove \(\overline{\partial}\phi=0\). Notice \[
        |u_x|^2-|u_y|^2-2i\left<u_x,u_y\right> =4\left<\partial u,\partial u\right> 
    .\] So \[
        \overline{\partial}\phi=8\left<\overline{\partial}\partial u,\partial u\right> 
        =-8\left<\lap u,\partial_z u\right> =0
    \] since \(\lap u=A(\nabla u,\nabla u)\perp TN\ni \nabla u\)
\end{proof}

We introduce a black-boxed criterion to \emph{branched immersions}. We need not to
understand this theorem until we get to the final result.
\begin{theorem}
    \(u\) is harmonic and weakly conformal implies that \(u\) is a branched immersion.
\end{theorem}

\begin{lemma}
    There is no non-trivial holomorphic quadratic differential on \(\mathbb{S}^2\).
\end{lemma}
\begin{proof}
    Suppose there is one, write it as \[
        \phi=f(z)\dd{z}^2
    ,\] where \(f\) is a entire function, \(z\) is stereographic coordinate. Let
    \(w=\frac{1}{z}\), then in \(w\)-coordinate, \[
        \phi=\frac{1}{w^4}f(\frac{1}{w})\dd{w}^2
    ,\] which should be holomorphic in \(w\). So \(f(\frac{1}{w})\to 0\) as \(w\to 0\).
    By Liouville theorem, \(f\) must be \(0\).
\end{proof}
\begin{corollary}\label{thm:harmonic-is-conformal}
    If \(u\colon \mathbb{S}^2\to N\) is harmonic. Then \(u\) is a
    smooth conformal branched minimal immersion.
\end{corollary}

For \(M\) genus larger than 0, things become more complicated, we state a sufficient
condition here without proof.
\begin{theorem}
    If \(u\) is a smooth critical map of \(E\) with respect not only to variation of
    \(u\), but also to variation of conformal structure on \(M\), then \(u\) is a
    conformal branched immersion.
\end{theorem}

\begin{remark}\hfill
\begin{itemize}
\item \(u\) critical in energy \(E\) implies \(u\) critical in area and in domain
    diffeomorphism. This is analog to geodesics.
\item For non-zero genus surface, we need \(u\) also to be critical in variation of
    complex structures on \(M\) to get minimality.
\end{itemize}
\end{remark}

\section{The Perturbed Problem}
Instead of \(E\), we consider a perturbed energy to gain extra regularity and to
obtain convergence properties. Normalize \(g\) such that \(M\) has volume 1. Let \[
    E_\alpha(u)=\int_{M}(1+|\nabla u|^2)^{\alpha}\dd{\mu_g}
.\] If \(\alpha=1\), clearly \(E_\alpha=1+E\) is equivalent to \(E\). The natural
mapping space is \(W^{1,2\alpha}\) instead of \(W^{1,2}\). Note that by Morrey's
inequality, we have \[
    \|u\|_{C^{0,1-\frac{n}{p}}}\le C\|\nabla u\|_{L^p}
.\] Take \(n=2,p=2\alpha\) for \(\alpha>1\), we got \[
    W^{1,2\alpha}(M,N)\subset C^{0,1-\frac{1}{\alpha}}(M,N)\subset C^0(M,N)
.\] Further, we claim that
\begin{theorem}
    The homotopy type of the mapping spaces \(W^{1,2\alpha}(M,N)\), \(C^0(M,N)\) and
    \(C^\infty(M,N)\) are the same.
\end{theorem}
We claim that \(W^{1,2\alpha}(M,N)\) is a \(C^2\) separable Banach manifold for
\(\alpha>1\).
\begin{definition}
    A \(C^1\) functional \(I\) on a separable Banach manifold \(L\) satisfies the
    \emph{Palais-Smale condition} if every sequence \(\{u_k\}\subset L\) such that:
    \begin{itemize}
    \item \(I(u_k)\) is bounded, and 
    \item \(I'(u_k)\to 0\) in \(L\)
    \end{itemize}
    has a convergent subsequence in \(L\).
\end{definition}
\begin{theorem}\label{thm:ps-critical}
    If \(I\) is a \(C^2\) function on a separable \(C^2\) Banach manifold \(L\), which
    satisfies the Palais-Smale condition, then 
    \begin{enumerate}[(1)]
    \item \(I\) takes minimum in every component of \(L\),
    \item \(I\) satisfies a Morse theory. \ie\ If there are no critical values of
        \(I\) in interval \([a,b]\), then there exists a deformation retraction \[
            \rho\colon I^{-1}(-\infty,b]\longrightarrow I^{-1}(-\infty,a]
        .\] 
    \end{enumerate}
\end{theorem}
\begin{proof}
    Reference to \emph{Lusternik-Schnirelman Theory on Banach Manifolds} by
    Richard S. Palais (1966).
\end{proof}
\begin{theorem}
    \(E_\alpha\) is a \(C^2\) functional on \(W^{1,2\alpha}(M,N)\), satisfying the
    Palais-Smale condition under metric induced from \(W^{1,2\alpha}(M,\mathbb{R}^k)\)
    when \(\alpha>1\).
\end{theorem}
\begin{proof}
    First we have compact embedding \[
        W^{1,2\alpha}\hookrightarrow{} C^{0,1-\frac{1}{\alpha}}
        \hookrightarrow C^{0,1-\frac{1}{\alpha}-\eps}
    .\] Then boundedness of \(E_\alpha\) gives uniform boundedness under
    \(W^{1,2\alpha}\) norm. Note that boundedness of any \(L^p\) norm is guaranteed by
    compactness of \(M,N\). Hence we can find a strongly convergent subsequence in
    \(C^{0,1-\frac{1}{\alpha}-\eps}\) norm, in particular, in \(C^0\) norm.
    \begin{align*}
        E_\alpha'(u)\vphi&=2\alpha\int_{M}(1+|\nabla u|^2)^{\alpha-1}
        \left<\nabla u,\nabla\vphi\right> \dd{\mu_g}\\
        &=-2\alpha\int_{M}\big(\left<\nabla(1+|\nabla u|^2)^{\alpha-1},\nabla u\right>
        +(1+|\nabla u|^2)^{\alpha-1}\lap' u\big)\cdot\vphi\dd{\mu_g} \\
        &=2\alpha\int_{M}\big(-\nabla^*((1+|\nabla u|^2)^{\alpha-1}\nabla u)
        +(1+|\nabla u|^2)^{\alpha-1}A(\nabla u,\nabla u)\big)\cdot\vphi\dd{\mu_g}
    .\end{align*}
    Note that this \(\vphi\) can be chosen to be any vector valued function on \(M\),
    since the term in bracket is parallel to \(TN\). Consider
    \begin{align*}
        (E_\alpha'(u)-E_\alpha'(v))(u-v)=&2\alpha\int_{M}
        ((1+|\nabla u|^2)^{\alpha-1}\nabla u-(1+|\nabla v|^2)^{\alpha-1}\nabla v)
        \cdot (\nabla u-\nabla v) \\
        &+(1+|\nabla u|^2)^{\alpha-1}(A(\nabla u,\nabla u)-A(\nabla v,\nabla v))
        \cdot (u-v)\dd{\mu_g}
    .\end{align*}
    We claim that \[
        ((1+|x|^2)^{\alpha-1}x-(1+|y|^2)^{\alpha-1}y)\cdot (x-y)
        \ge |x-y|^2+|x-y|^{2\alpha}
    .\] Then we have estimate 
    \begin{align*}
        &\|E_\alpha'(u)-E_\alpha'(v)\|\|u-v\|_{W^{1,2\alpha}}
        +C\|A\|_{L^\infty}(\|\nabla u\|_{L^{2\alpha}}^{2\alpha}
        +\|\nabla v\|_{L^{2\alpha}}^{2\alpha})\|u-v\|_{L^\infty} \\
        \ge &\|\nabla u-\nabla v\|_{L^2}^2
        +\|\nabla u-\nabla v\|_{L^{2\alpha}}^{2\alpha}
    .\end{align*}
    By uniform convergence, we see that \(u_k\) converges in \(W^{1,2\alpha}\).
\end{proof}
\begin{remark}
    If \(\alpha=1\), \(E\) cannot satisfy any kind of Palais-Smale condition.
    We will see why later.
\end{remark}

Now we prove the regularity of critical maps of \(E_\alpha\). It is true for all
\(\alpha>1\), but here we give a simple proof for case \(\alpha-1\) small.
\begin{theorem}\label{thm:regularity-alpha}
    The critical maps of \(E_\alpha\) in \(W^{1,2\alpha}(M,N)\) are \(C^\infty\) if
    \(\alpha>1\).
\end{theorem}
\begin{proof}
    The Euler-Lagrange equation of \(E_\alpha\) writes as \[
        -\nabla^*((1+|\nabla u|^2)^{\alpha-1}\nabla u)+(1+|\nabla u|^2)^{\alpha-1}
        A(\nabla u,\nabla u)=0
    .\] By Morrey 1.11.1 and the fact that \(u\in C^{0,1-\frac{1}{\alpha}}\), we
    have that \(\nabla u\in W^{1,2}(M,N)\). Now we can expand and divide to get
    \begin{equation}\label{eq:e-l-alpha}
        -\lap u-2(\alpha-1)\frac{\left<\left<\nabla^2 u,\nabla u\right>_{g,h},
        \nabla u\right>_g}{1+|\nabla u|^2}+A(\nabla u,\nabla u)=0
    .\end{equation}
    We use the following notation to reduce expression size \[
        (\nabla^2 u,\nabla u)\cdot\nabla u
        :=\left<\left<\nabla^2 u,\nabla u\right>_{g,h},\nabla u\right>_g
        =g^{ik}g^{jl}h_{\alpha\beta}u_{ij}^\alpha u_k^\beta u_l^\gamma \pd{\xi^\gamma}
    .\] When \(\alpha-1\) is small, we have nice inverse for operator \[
        L_s\colon W^{2,p}(M,N)\longrightarrow L^p(M,N),\quad
        L_s v=\lap v+2(\alpha-1) \frac{(\nabla^2 v,\nabla u)\nabla u}
        {1+|\nabla u|^2}
    .\] \(\nabla u\in W^{1,2}\) implies that \(\nabla u\in L^p\) for any
    \(p<2^*=\infty\). Hence we have \(u\in W^{2,p}\), then \(u\in C^{1,\beta}\)
    by Morrey's lemma. Now the equation can be view as a linear equation with H\"older
    continuous coefficients. Schauder estimate (Morrey thm 5.6.3) gives
    \(u\in C^{2,\beta}\), then we can bootstrap to get \(u\in C^\infty\).
\end{proof}
We have nearly prove the existence of non-trivial critical maps of \(E_\alpha\),
\(\alpha>1\). The only gap is \[
    N_0=\{u\equiv \xi\in N\}\cong N
,\] set of trivial critical maps, for which \(E_\alpha\) takes its minimum value 1.

\begin{prop}
    Let \(\alpha>1\). In every connected component of \(W^{1,2\alpha}(M,N)\),
    the minimum value of \(E_\alpha\) is realised by a smooth \(u_\alpha\),
    which also minimizes \(E_\alpha\) in its connected component in \(C^\infty(M,N)\).
    There exists \(B>0\) independent of \(\alpha\) such that \[
        \min E_\alpha \le (1+B^2)^\alpha
    \] in that component.
\end{prop}
\begin{proof}
    Since \(E_\alpha\) satisfies (PS), it takes minimum at some \(u_\alpha\) in 
    each connected component of \(W^{1,2\alpha}(M,N)\). By regularity theorem
    \(u_\alpha\) is smooth. In each component of \(C^\infty(M,N)\), choose a map \(v\)
    and let \(B=\|\nabla v\|_{L^\infty}\). Then \[
        \min E_\alpha\le E_\alpha(v)\le (1+B^2)^\alpha
    \] in that component.

    Note that this \(B\) may depend on the component.
\end{proof}
Now we need to analyze the structure of \(N_0\subset W^{1,2\alpha}(M,N)\). Note that
for \(\xi\in N_0\), \[
    T_\xi W^{1,2\alpha}(M,N)\cong W^{1,2\alpha}(M,T_\xi N)
.\] And \(T_\xi N_0\) is those constant variation vector fields, \[
    T_\xi N_0=\{V\in W^{1,2\alpha}(M,T_\xi N):\dd{V}=0\}.
.\] Note that we identify point \(\xi\in N\) with constant map to \(\xi\).

We can construct a normal bundle to \(N_0\) as follows:
\begin{gather*}
    \mathcal{N}_\xi=\{V\in W^{1,2\alpha}(M,T_\xi N):\int_{M}V\dd{\mu_g}=0\}, \\
    \mathcal{N}=\bigcup_{\xi\in N_0}\mathcal{N}_\xi\subset
    \eval{TW^{1,2\alpha}(M,N)}_{N_0}.
\end{gather*}
Let \(\exp\colon TN\to N\) be the exponential map on \(N\). Define \[
    \op{Exp}\,\colon TW^{1,2\alpha}(M,N)\longrightarrow W^{1,2\alpha}(M,N),\quad
    \op{Exp}(u,V)=(x\mapsto \exp(u(x),V(x)))
.\] 
\begin{lemma}
    \(\eval{\Exp}_{\mathcal{N}}\to W^{1,2\alpha}(M,N)\) is a diffeomorphism from
    a neighbourhood of zero section to a neighbourhood of \(N_0\subset
    W^{1,2\alpha}(M,N)\).
\end{lemma}
\begin{proof}
    We'd like to apply inverse function theorem, \[
        \eval{\dd{\Exp}}_{(\xi,0)}(\nu,W)
        =\big(x\mapsto\eval{\dd\exp}_{(\xi,0)}(\nu,W(x))\big)
        =(x\mapsto \nu+W(x))
    \] for \(\nu\in T_\xi N,W\in \mathcal{N}_\xi\). From our construction of
    \(\mathcal{N}\), \[
        \eval{\dd{\Exp}}_{(\xi,0)}\colon T_\xi N\oplus\mathcal{N}_\xi
        \longrightarrow T_\xi W^{1,2\alpha}(M,N)\cong W^{1,2\alpha}(M,T_\xi N)
    \] is an isomorphism. Note that for \(W\) small in \(W^{1,2\alpha}\) norm,
    it must be small in \(L^\infty\) norm. Hence the inverse function theorem applies
    and we are done.
\end{proof}
\begin{lemma}
    \(\alpha>1\), \(u\in W^{1,2\alpha}(M,\mathbb{R}^k)\), then there exists
    \(C_\alpha>0\) such that \[
        \|u-\int_M u\|_{L^\infty}\le C_\alpha \|\nabla u\|_{L^{2\alpha}}.
    .\]
\end{lemma}
\begin{proof}
    Suppose such \(C_\alpha\) does not exist, then for any \(k\), there exist \(u_k\)
    such that \[
        \|u_k-\int_{M}u_k\|_{L^\infty}\ge k \|\nabla u_k\|_{L^{2\alpha}}
    .\] Further we can assume \(\int_{M}u_k=0\) and \(\|u_k\|_{L^\infty}=1\).
    Then \[
        \|\nabla u_k\|_{L^{2\alpha}}\longrightarrow 0
    .\] Then there is a weakly convergent subsequence since \(\|u_k\|_{W^{1,2\alpha}}\)
    is uniformly bounded. Still denote the subsequence by \(u_k\), we have \[
        u_k\xrr{W^{1,2\alpha},w}u
    .\] By Morrey's lemma, this gives \[
        u_k\xrr{C^{0,1-\frac{1}{\alpha}}}u
    .\] In particular, the convergence is uniform. Hence  \[
        \|u_k\|_{L^{2\alpha}}\longrightarrow \|u\|_{L^{2\alpha}}
    .\] Note we have \[
        \liminf_{k\to \infty}\|u_k\|_{W^{1,2\alpha}}\ge \|u\|_{W^{1,2\alpha}}
    .\] Hence \(\|\nabla u\|_{L^{2\alpha}}=0\), and \(u\) is constant.
    This contradicts to assumptions on \(u_k\). 
\end{proof}
\begin{theorem}\label{thm:low-energy-retraction}
    Given \(\alpha>1\) there exists a \(\delta>0\) depending on \(\alpha\) and a
    deformation retraction \[
        \sigma\colon E_\alpha^{-1}([1,1+\delta])\longrightarrow E_\alpha^{-1}(1)
        \cong N_0
    .\] 
\end{theorem}
\begin{proof}
    For \(u\in E_\alpha^{-1}([1,1+\delta])\), we have \(\int_{M}|\dd{u}|^{2\alpha}
    \dd{\mu}<\delta\), then by above lemma, \[
        \sup \left|u-\int_{M}u\dd{\mu}\right|\le C(\alpha)\delta^{\frac{1}{2\alpha}}
    .\] Hence for sufficiently small \(\delta\), image of \(u\) lie in a normal
    neighbourhood on \(N\). There is unique \(\xi\in N_0\) and \(v\colon M\to
    \mathcal{N}_{\xi}\) \st\ \(u=\Exp(\xi,v)\). We also have that \(\|v\|_{L^\infty}\)
    and \(\|\nabla v\|_{L^{2\alpha}}\) can be made arbitrarily small for small
    \(\delta\).

    Consider the natural choice of the retraction \(\sigma\): \[
        \sigma_t(u)=\Exp(\xi,tv)
    .\] \(\sigma\) is continuous as long as \(\delta\) is sufficiently small.

    Now we prove that \(\dv{}{t}E_\alpha(\sigma_t(u))\ge 0\), and hence \(\sigma\) is a
    deformation retraction. Let \(u_t=\sigma_t(u)\), then
    \begin{gather*}
        \dv{}{t}(u_t)=\dv{}{t}(\Exp(\xi,tv))=\eval{\dd\Exp}_{(\xi,tv)}(\xi,v) \\
        \nabla u_t=t\eval{\dd{\Exp_\xi}}_{tv}(\nabla v)
        \implies \nabla v=\frac{1}{t}\eval{\dd{\Exp_\xi}}_{tv}^{-1}(\nabla u_t)
        =\frac{1}{t}\nabla u_t+O(\|v\|_{L^\infty})\nabla u_t
    \end{gather*}
    Since \(\eval{\dd\Exp}_{(\xi,0)}\) is identity, we calculate
    \begin{align*}
        \dv{}{t}E_\alpha(u_t)&=2\alpha\int_{M}(1+|\nabla u_t|^2)^{\alpha-1}
        \left<\nabla u_t,{\color{orange}\nabla v}\right> \dd{\mu_g} \\
        &\ge \frac{2\alpha}{t}\int_{M}(1+|\nabla u_t|^2)^{\alpha-1}|\nabla u_t|^2
        (1-C't\|v\|_{L^\infty})\dd{\mu_g} \\
        &\ge 0\text{ for small }\delta
    .\end{align*}
\end{proof}

Let \(\Omega(M,N)\) be base preserving mapping space from \(M\) to \(N\). We have
natural fibration \[
    \Omega(M,N)\longrightarrow C^0(M,N)\longrightarrow N
.\] 
\begin{theorem}
    If \(\Omega(M,N)\) is not contractible, then there exists a \(B>0\), such that
    for all \(\alpha>1\), \(E_\alpha\) has a critical value in interval 
    \((1,(1+B^2)^\alpha]\).
\end{theorem}
\begin{proof}
    The fibration above have a section \(N\to N_0\subset C^0(M,N)\), hence the exact
    sequence of homotopy groups split, \[
        \pi_k(C^0(M,N))=\pi_k(N)\oplus \pi_k(\Omega(M,N))
    .\] If \(C^0(M,N)\) is not connected, by \cref{thm:ps-critical} we can choose a
    connect component not contain \(N_0\), which contains a critical \(u_\alpha\).
    Otherwise choose a non-zero homotopy class \(\gamma\in \pi_k(\Omega(M,N))\).
    Note that \[
        \gamma \colon \mathbb{S}^k\to \Omega(M,N)\subset C^0(M,N)
    \] is not homotopic to any map \(\mathbb{S}^k\to N_0\). Let \[
        B=\sup_{p\in \mathbb{S}^k,x\in M}|\dd{\gamma(p)}(x)|
    .\] Then \(E_\alpha(\gamma(p))\le (1+B^2)^{\alpha}\) for any \(p\in\mathbb{S}^k\).

    Suppose that \(E_\alpha\) has no critical value in \((1,(1+B^2)^\alpha]\).
    \Cref{thm:ps-critical} shows that there is deformation retraction \[
        \rho\colon E_\alpha^{-1}([1,(1+B^2)^\alpha])
        \longrightarrow E_\alpha^{-1}([1,1+\delta])
    \] for any \(\delta>0\). Choose \(\delta\) as in \cref{thm:low-energy-retraction},
    we have deformation retraction \[
        \sigma\circ \rho\colon E_\alpha^{-1}([1,(1+B^2)^\alpha])
        \longrightarrow E_\alpha^{-1}(1)=N_0
    .\] But \(\sigma\circ \rho\circ \gamma\colon \mathbb{S}^k\to N_0\) is homotopic
    to \(\alpha\). This gives a contradiction.
\end{proof}

\begin{theorem}\label{thm:alpha-existence}
    If \(M=\mathbb{S}^2\) and the universal covering of \(N\) is not contractible,
    then there exists a \(B>0\) and a critical map of \(E_\alpha\) with value in
    \((1,(1+B^2)^\alpha)\) for \(\alpha>1\).
\end{theorem}
\begin{proof}
    If \(\tilde{N}\) is not contractible, then \[
        \pi_{k+2}(N)=\pi_k(\Omega(\mathbb{S}^2,N))\neq 0
    \] for some \(k\). Apply last theorem.
\end{proof}

\section{Main Estimate and Regularity}
In this section, we will obtain several local estimates and regularity theorems.
We shall consider \(\alpha\ge 1\) for the estimates below.

Cover \(M\) with small disks with radius \(R\), such that on each disk the metric
differs from euclidean one by \(\eps\) order. Locally we can conformally expand
the disk to be a unit disk. We have \[
    E_\alpha(u)=\int_D(1+|\nabla u|_{g}^2)^{\alpha}\dd{\mu_g}
    =R^{2(1-\alpha)}\int_D(R^2+|\nabla u|_{\tilde{g}}^2)^{\alpha}\dd{\mu_{\tilde{g}}}
.\] The Euler-Lagrange equation is now
\begin{equation}\label{eq:e-l-disk}
    -\lap u-2(\alpha-1)\frac{(\nabla^2 u,\nabla u)\cdot\nabla u}{R^2+|\nabla u|^2}
    +A(\nabla u,\nabla u)=0
.\end{equation}
\begin{theorem}
    Let \(u\colon D\to N\) be critical map of \(E_\alpha\), suppose \(\alpha-1>0\) is
    small (may depending on \(p\)), then for all smaller disks \(D'\subset D\),
    we have estimate \[
        \|\nabla u\|_{W^{1,p}}'\le C(p,D',{\color{orange}\|\nabla u\|_{L^4}})
        \cdot \|\nabla u\|_{L^4}
    .\] Where the primed norm means norm on \(D'\) and others are norm on \(D\).
\end{theorem}
\begin{proof}
    Without loss of generality, we choose the origin of \(\mathbb{R}^k\) such that
    \(\int u=0\). Then \(\|\nabla u\|_{L^p}\) can bound the norm \(\|u\|_{W^{1,p}}\)
    by Poincaré inequality. Let \(\vphi\) be smooth function supported on \(D^\circ\)
    and be identical 1 on \(D'\). Replace \(u\) by \(\vphi u\), we have estimate
    \begin{align*}
        &\left|\lap (\vphi u)+2(\alpha-1) \frac{(\nabla^2(\vphi u),\nabla u)\cdot
        \nabla u}{R^2+|\nabla u|^2}-A(\nabla(\vphi u),\nabla u)\right| \\
        \le & \left|\left<\nabla\vphi,\nabla u\right>+u\lap\vphi\right|
        +2(\alpha-1)\left|u\nabla^2\vphi+\nabla u\otimes \nabla\vphi\right|
        +\left|A(u\nabla\vphi,\nabla u)\right|\\
        \le &\ C_1(\vphi,\|A\|_{\infty},\|u\|_{L^\infty})(|\nabla u|+|u|)
    .\end{align*}
    Take \(L^p\) norm on both side, \[
        \|\lap(\vphi u)\|_{L^p}
        \le 2(\alpha-1)\|\vphi u\|_{W^{2,p}}
        +\|A\|_{\infty}\||\nabla(\vphi u)||\nabla u|\|_{L^p}
        +C_1 \|u\|_{W^{1,p}}
    .\] 
    Let \(c_p\) be the norm of \(\lap^{-1}\colon L^p\to W^{2,p}\cap H_0^1\) on
    \(D\), then we get
    \begin{equation}\label{eq:est-7}
        (c_p^{-1}-2(\alpha-1))\|\vphi u\|_{W^{2,p}}
        \le \|A\|_{\infty}\||\nabla(\vphi u)||\nabla u|\|_{L^p}
        +C_1 \|u\|_{W^{1,p}}
    .\end{equation}
    Take \(p=2\), we have
    \begin{align*}
        (c_2^{-1}-2(\alpha-1))\|\vphi u\|_{W^{2,2}}
        &\le \|A\|_{\infty}\|\nabla(\vphi u)\|_{L^4}\|\nabla u\|_{L^4}
        +C_1\|u\|_{W^{1,2}} \\
        &\le C_2(\vphi,\|A\|_{\infty},\|\nabla u\|_{L^4})\cdot\|\nabla u\|_{L^4}
    .\end{align*}
    Note we use implicitly the fact \(N\) is compact. This provides the bound for
    \(\|u\|_{W^{2,2}}''\) on any further smaller disk \(D''\), hence for any 
    \(\|u\|_{W^{1,p}}''\). For sufficiently small \(\alpha-1\), \(c_p-2(\alpha-1)>0\).
    Apply in\cref{eq:est-7} again we get desired estimate for
    \(\|\nabla u\|_{W^{1,p}}''\).
\end{proof}

\begin{theorem}[Main Estimate]\label{thm:main-est}
    There exists \(\eps>0\) and \(\alpha_0>1\) such that if \(u\colon D\to N\)
    is a {\color{red}smooth} critical map of \(E_\alpha\), \(1\le\alpha\le\alpha_0\) and \(E(u)<\eps\), then \[
        \|\nabla u\|_{W^{1,p}}'\le C(p,D')\|\nabla u\|_{L^2}
    \] for any smaller disk \(D'\subset D\), uniform in \(1\le \alpha\le\alpha_0\).
\end{theorem}
\begin{proof}
    Note \(E(u)\) is just \(\|\nabla u\|_{L^2}^2\), so we only need to prove that \[
        \|\nabla u\|_{L^4}''\le C(D'')\|\nabla u\|_{L^2}
    .\] Again we assume \(\int u=0\), apply in\cref{eq:est-7} with \(p=\frac{4}{3}\),
    the bad quadratic term can be estimated by H\"older inequality: \[
        \||\nabla(\vphi u)||\nabla u|\|_{L^{4/3}}
        \le \|\nabla(\vphi u)\|_{L^4}\|\nabla u\|_{L^2}
    .\] Note that we have Sobolev embedding \(W^{2,4/3}(D,\mathbb{R}^k)
    \subset W^{1,4}(D,\mathbb{R}^k)\), \ie\ \[
        \|\nabla(\vphi u)\|_{L^4}\le C_3\|\vphi u\|_{W^{2,4/3}}
    .\] Then in\cref{eq:est-7} gives
    \begin{equation}\label{eq:est-7'}
    (c_{4/3}^{-1}-2(\alpha-1))\|\vphi u\|_{W^{2,4/3}}
    \le C_3\|A\|_{\infty}\|\nabla u\|_{L^2}\|\vphi u\|_{W^{2,4/3}}
    +C_1\|u\|_{W^{1,4/3}}
    .\end{equation}
    \(\|\nabla u\|_{L^{4/3}}\le C\|\nabla u\|_{L^2}\) since \(N\) is
    compact. If in addition \[
        c_{4/3}^{-1}-2(\alpha-1)-C_3\|A\|_{\infty}\sqrt{E(u)}>0
    ,\] we get the desired estimate on \(\|\nabla u\|_{L^4}''\le C_3\|\vphi u\|
    _{W^{2,4/3}}\).
\end{proof}
We can do similar estimate globally, without the boundary term containing \(\vphi\).
Global version of in\cref{eq:est-7'} takes form \[
    (c_{4/3}^{-1}-2(\alpha-1))\|u\|_{W^{2,4/3}}
    \le C\|A\|_{\infty}\sqrt{E(u)}\|u\|_{W^{2,4/3}}
,\] where the constants and norm are global version.

Clearly if \(\sqrt{E(u)}\) is too small, \(u\) must be a constant
\(u\equiv\int u\).
\begin{theorem}\label{thm:energy-gap}
    There exists \(\eps>0\) and \(\alpha_0>1\) such that if \(1\le\alpha\le\alpha_0\),
    \(E(u)<\eps\) and \(u\) is critical map of \(E_\alpha\), then \(u\) is constant.
\end{theorem}

Now we turn to consider the original energy \(E(u)\), we'd like to prove a slightly
stronger regularity theorem for harmonic maps. Let \(D\) be a small geodesic disk
on \(M\), with normal coordinate. Notation \(D(R)\) will mean a disk of radius \(R\).
\begin{theorem}\label{thm:removable-singularity}
    If \(u\colon D\setminus\{0\}\to N\) is harmonic and has finite energy, then
    \(u\) extends to a smooth harmonic map \(u\colon D\to N\).
\end{theorem}

Note that \(E\) is conformal invariant, so we can always do a conformal expansion
to make \(D\) a unit disk. By shrinking \(D\) to neighbourhood of 0, we can assume
\(E(u\big|_{D})<\eps\). We choose \(\eps\) later. Moreover, we can assume the
metric is flat in the coordinate.

We need several estimation lemmas to prove the theorem.
\begin{lemma}\label{lem:sup-est}
    There is an \(\eps>0\) such that if \(u\) is smooth harmonic on \(D(2)\setminus\{0\}\) and
    \(\int_{D(2)}|\nabla u|^2\dd{\mu}<\eps\), then \[
        |\nabla u(x)||x|
        \le C\left(\int_{D(2|x|)}|\nabla u|^2\dd{\mu}\right)^{\frac{1}{2}}
        =C\|\nabla u\|_{L^2(D(2|x|))}
    \] for any \(x\in D=D(1)\).
\end{lemma}
\begin{proof}
    Choose \(\eps\) from the main estimate, with \(p=4\) and \(\alpha=1\). Fix
    \(y\in D\), let \(\tilde{u}(x)=u(y+|y|x)\). Then \[
        \int_{D}|\nabla\tilde{u}|^2\dd{\mu}
        =\int_{D(y,|y|)}|\nabla u|^2\dd{\mu}
        \le E(u\big|_{D(2)})<\eps
    .\] Apply \cref{thm:main-est} to \(\tilde{u}\colon D\to N\), and use Sobolev
    embedding \(W^{1,4}\hookrightarrow L^\infty\), we have \[
        \sup_{x\in D(1/2)}|\nabla \tilde{u}(x)|
        \le C\|\nabla \tilde{u}\|_{W^{1,4}(D(1/2))}
        \le C'\|\nabla \tilde{u}\|_{L^2(D)}
    .\] Translate this to \(u\), we get \[
        |\nabla u(y)||y|=|\nabla\tilde{u}(0)|
        \le C'E\left(\tilde{u}\big|_{D}\right)^{\frac{1}{2}}
        \le C'E\left(u\big|_{D(2|y|)}\right)^{\frac{1}{2}}
    .\] Which is what we wanted to show.
\end{proof}
\begin{lemma}\label{lem:circle-integral}
    Let \(u\colon D\setminus\{0\}\to N\subset \mathbb{R}^k\) be a smooth harmonic
    map with finite energy, then \[
        \int_{0}^{2\pi}|u_\theta|^2\dd{\theta}
        =r^2 \int_{0}^{2\pi}|u_r|^2\dd{\theta}
    .\] Where \(z=re^{i\theta}\) is local complex coordinate on \(M\).
\end{lemma}
\begin{proof}
    Consider the quadratic differential \(\phi=w(z)\dd{z}^2\). By
    \cref{lem:sup-est}, \[
        |w(z)|=|u_x|^2-|u_y|^2-2i\left<u_x,u_y\right>\le 2|\nabla u(z)|^2
        \le \frac{C}{|z|^2}
    .\] Therefore \(w(z)\) has a pole of order at most 2 at \(z=0\).
    On the other hand, \[
        \int_{D}|w(z)|\dd{\mu}\le 2\int_{D}|\nabla u|^2\dd{\mu}<\infty
    .\] The order of the pole is at most 1. Notice \[
        \Re(z^2 w(z))=|z|^2|u_r(z)|^2-|u_\theta(z)|^2
    .\] Then \[
        0=\Im\int_{|z|=r_0}zw(z)\dd{z}
        =\Re \int_{r=r_0}z^2w(z)\dd{\theta}
        =\int_{0}^{2\pi}r_0^2|u_r(r_0,\theta)|^2-|u_\theta(r_0,\theta)|^2\dd{\theta}
    .\] This proves the lemma.
\end{proof}
\begin{proof}[Proof of \cref{thm:removable-singularity}]
    First assume \(\int_{D(2)}|\nabla u|^2\dd{\mu}<\eps\) be a comformal expansion.
    Our goal is to prove \(u\) satisfy slightly higher integrability.

    We construct an approximation of \(u\): Let \[
        q(2^{-m})=\frac{1}{2\pi}\int_{0}^{2\pi}u(2^{-m},\theta)\dd{\theta}
    \] be the average of \(u\) on circle of radius \(2^{-m}\). Define \(q\) to depend
    only on \(r\) and be piecewise linear in \(\log r\). Then \(q\) is harmonic
    for \(r\in (2^{-m-1},2^{-m})\) and absolutely continuous in \(D\).
    Now for \(r\in [2^{-m-1},2^{-m}]\), \[
        |q(r)-u(r,\theta)|\le |q(2^{-m-1}-q(2^{-m}))|+|u(r,\theta)-q(2^{-m})|
    .\] Apply again \cref{lem:sup-est}, \[
        \sup_{2^{-m-1}\le |x|,|y|\le 2^{-m}}|u(x)-u(y)|
        \le C 2^{-m}\cdot\sup_{2^{-m-1}\le |x|\le 2^{-m}}|\nabla u(x)|
        \le C'\left(\int_{|x|\le 2^{-m}}|\nabla u|^2\right)^{\frac{1}{2}}
    .\] Then we can assume \[
        |q(r)-u(r,\theta)|
        \le C''\left(\int_{|x|\le 2^{-m}}|\nabla u|^2\dd{\mu}\right)^{\frac{1}{2}}
        \le C''\sqrt{\eps}
    .\] Next we estimate \(H^1\) norm of \(q-u\). By divergence theorem,
    \begin{align*}
        \int_{D}|\nabla q-\nabla u|^2\dd{\mu}
        =\sum_{m=0}^\infty & r \int_{0}^{2\pi}(q(r)-u(r,\theta))\cdot
        ({\color{green}q'(r)\color{orange}-u_r(r,\theta)})
        \dd{\theta}\Big|_{2^{-m-1}}^{2^{-m}} \\
        -& \int_{D\setminus\{r=2^{-m}:m\ge 0\}} (q-u)\lap (q-u)\dd{\mu}
    .\end{align*} The green term disappears since \(q\) is the average of \(u\). The
    orange term cancel with succeeding and preceding terms since \(u_r\) is continuous.
    Note that the green term cannot cancel in this way since \(q(r)\) is not
    continuous. Easy to see the limit term tends to 0 as \(m\to \infty\).
    For the Laplacian term, notice \(\lap (q-u)=A(\nabla u,\nabla u)\) on the
    integration domain. Hence we can estimate the term by \[
        \|A\|_{\infty}\|q-u\|_{L^\infty}\|\nabla u\|_{L^2}^2
        \le C\|A\|_{\infty}\sqrt{\eps}\|\nabla u\|_{L^2}^2
    .\] Choose a small \(\eps\) \st\ \(C\|A\|_{\infty}\sqrt{\eps}<\delta\). Then
    we have
    \begin{equation}\label{eq:q-u-H1}
        \int_{D}|\nabla (q-u)|^2\dd{\mu}
        \le \left(\int_{r=1}|q-u|^2\dd{\theta}\right)^{\frac{1}{2}}
        \left(\int_{r=1}|u_r|^2\dd{\theta}\right)^{\frac{1}{2}}
        +\delta \int_{D}|\nabla u|^2\dd{\mu}
    .\end{equation}
    By \cref{lem:circle-integral}, \[
        \int_{0}^{2\pi}|u_r|^2\dd{\theta}
        =\frac{1}{r^2}\int_{0}^{2\pi}|u_\theta|^2\dd{\theta}
        =\frac{1}{2}\int_{0}^{2\pi}|\nabla u|^2\dd{\theta}
    .\] So the left hand side of (\ref{eq:q-u-H1}) is \[
        \int_{D}|\nabla (q-u)|^2\dd{\mu}
        =\int_{D}|q_r-u_r|^2+r^{-2}|u_\theta|^2\dd{\mu}
        \ge \frac{1}{2}\int_{D}|\nabla u|^2\dd{\mu}
    .\] By 1-dimensional Poincaré inequality, \[
        \int_{r=1}|q-u|^2\dd{\theta}\le \int_{r=1}|u_\theta|^2\dd{\theta}
        =\int_{r=1}|u_r|^2\dd{\theta}=\frac{1}{2}\int_{r=1}|\nabla u|^2\dd{\theta}
    .\] Put these together, we obtain \[
        (1-2\delta)\int_{D}|\nabla u|^2\dd{\mu}\le \int_{r=1}|\nabla u|^2\dd{\theta}
    .\] This can be easily to translate to disks of any radius, precisely, \[
        (1-2\delta)\int_{D(R)}|\nabla u|^2\dd{\mu}
        \le R\int_{r=R}|\nabla u|^2\dd{\theta}
    .\] This yields \[
        \dv{}{R}\left(\frac{1}{R^{1-2\delta}}\int_{D(R)}|\nabla u|^2\dd{\mu}\right)\ge 0
    .\] Integrate we have \[
        \int_{D(r)}|\nabla u|^2\dd{\mu}\le r^{1-2\delta}\int_{D}|\nabla u|^2\dd{\mu}
    .\] Apply \cref{lem:sup-est} again we get, for \(0<|y|<\frac{1}{2}\), \[
        |\nabla u(y)||y|
        \le C\left(\int_{D(2|y|)}|\nabla u|^2\dd{\mu}\right)^{\frac{1}{2}}
        \le C(2|y|)^{\frac{1-2\delta}{2}}\|u\|_{L^2(D)}
    .\] This implies \(u\in W^{1,2\beta}\) for some \(\beta>1\).
    We claim that \(u\) satisfy the Euler-Lagrange equation weakly on entire \(D\),
    then following similar argument as for \(E_\alpha\) critical maps gives
    regularity of \(u\).
\end{proof}

The last claim essentially used the following lemma:
\begin{lemma}
    Let \(D\) be the unit disk in \(\mathbb{R}^n\) for \(n\ge 2\), suppose \(u\)
    is a harmonic map on \(D\setminus\{0\}\), and has finite energy, then \(u\) is
    weakly harmonic on \(D\).
\end{lemma}
\begin{proof}
    We claim that \(H_0^1(D\setminus\{0\},\mathbb{R}^k)=H_0^1(D,\mathbb{R}^k)\), then
    for any \(\vphi\in H^1(D,\mathbb{R}^k)\), choose \(\vphi_m\in H_0^1(D\setminus
    \{0\},\mathbb{R}^k)\) such that \(\vphi_m\xrightarrow{H^1}\vphi\), we have \[
        \left<-\lap' u,\vphi\right> =\left<-\lap u+A(\nabla u,\nabla u),
        \vphi\right> =\lim_{m\to\infty}\left<-\lap u-A(\nabla u, \nabla u),
        \vphi_m\right> =0
    .\] For the claim, we split the case into \(n=2\) and \(n\ge 3\).

    Case \(n\ge 3\): Let \(\chi(r)\) increasing, smooth, be 0 around \(r=0\) and
    \(1\) for \(r\ge 1\), \(\chi_\eps(r)=\chi(\frac{r}{\eps})\). Then let
    \(\vphi_\eps=\chi_\eps\vphi\).

    Case \(n=2\): Let \[
        \chi_\eps(r)=\begin{cases}
            0, & r\le \eps^2 \\
            1-\frac{\log \frac{r}{\eps}}{\log\eps} & \eps^2\le r\le \eps \\
            1, & r\ge \eps
        \end{cases}
    .\] Then let \(\vphi_\eps=\chi_\eps\vphi\).
\end{proof}

\section{More Regularity}
In this section, we relax the dimension condition of \(M\) to be greater or equal to 2,
and we'd like to sketch the proof of a more general regularity theorem.

\noindent\textbf{Reference}:
\begin{itemize}\itshape{}
\item Richard Schoen, Karen Uhlenbeck. A Regularity Thoery For Harmonic Maps (1982)
\item Lin Fanghua. Revisit the Regularity Theory of Schoen-Uhlenbeck (in Chinese) (2017)
\end{itemize}

\begin{theorem}[Monotonicity Formula]\hfill\\
    Let \(u\) be a \(W^{1,2}\) map from \(M\) to \(N\subset \mathbb{R}^k\),
    \(E(R)\) be the energy of \(u\) restricted on geodesic ball \(B(0,R)\),
    centered at some \(x_0\in M\). Then 
    \begin{equation}\label{eq:monotonicity}
    \begin{split}
        \dv{}{R}E(R)&=\frac{n-2}{R}E(R)+\int_{\partial B(0,R)}|u_r|^2\dd{\mu_g}
        -\frac{1}{R}\int_{B(0,R)}ru_r\cdot \lap u\dd{\mu_g} \\
        &+\frac{1}{2R}\int_{B(0,R)}r(\lap r-\frac{n-1}{r})|\nabla u|^2\dd{\mu_g}\\
        &+\frac{1}{R}\int_{B(0,R)}|u_\theta|^2
        -\frac{1}{R}\int_{B(0,R)}r\Hess r(u_\theta,u_\theta)\dd{\mu_g}.
    \end{split}\end{equation}
    Where \((r,\theta)\) is the normal spherical coordinate of the geodesic ball,
    \(\nabla u=u_r\dd{r}+u_{\theta^i}\dd{\theta^i}\), and \(u_\theta\) be the
    angular part \(u_{\theta^i}\dd{\theta^i}\).
    Note that \(\pd{r}\log\sqrt{\det g}=\lap r-\frac{n-1}{r}\), and
    \(\II^{\partial B(0,r)\hookrightarrow M}=-\Hess r\big|_{\partial B(0,r)}\).
\end{theorem}
\begin{corollary}
    If we further assume \(u\) is weakly harmonic, \ie\ \(\lap u=A(\nabla u,
    \nabla u)\), then there exists a small \(R_0>0\), and bounded function \(h(r)\),
    such that \[
        \dv{}{r}(\frac{e^{h(r)}}{r^{n-2}}E(r))\ge 0
    \] for \(r<R_0\). If \(M\) is compact, the constants can be chosen uniform for any
    \(x_0\in M\).
\end{corollary}

Let \(F(x,\eps)\) be \(\displaystyle\frac{e^{h(r)}}{r^{n-2}}E(u\big|_{B(x,\eps)})\).
Consider the set \[
    \mathfrak{S}=\{x\in M:\lim_{\eps\to 0}F(x,\eps)\ge c\}
.\] The limit exists by monotonicity. Let \(\Theta(x)=\lim_{\eps\to 0}F(x,\eps)\)
be the density function.
\begin{theorem}
    \(\Theta\) is upper semi-continuous, and hence \(\mathfrak{S}\) is a closed set.
\end{theorem}
Cover \(\mathfrak{S}\) by small balls \(B(x_i,2\eps)\), where \(x_i\in\mathfrak{S}
\), \(2\eps<R_0\) in monotonicity formula. We have estimate \[
    \sum_i \eps^{n-2}F(x_i,\eps)\le Ce^{h(\eps)}E(u\big|_{\bigcup B(x_i,2\eps)})
.\] The constant \(C\) comes from a covering lemma argument. Then \[
    \sum_i \mu(B(x_i,\eps))\le c^{-1}C' E(u\big|_{\bigcup B(x_i,2\eps)})
.\] The right hand side tends to zero as \(\eps\to 0\), by dominated convergence
theorem. We conclude that \(\mathcal{H}^{n-2}(\mathfrak{S})=0\). Hence, with a
possible \((n-2)\)-dimensional exception set, we have \(\Theta(x)=0\). This translate
to \[
    \int_{B(x_0,\eps)}|\nabla u|^2\dd{\mu_g}\le C\eps^{n-2}
    \quad\text{for small }\eps
.\] Next, let me just sketch the idea of proof for now.

First we need a alternative version of Morrey's lemma:
\begin{lemma}[Morrey]
    Suppose \(\Omega\subset \mathbb{R}^n\) open, \(u\in W^{1,p}(\Omega)\), and for any
    sphere \(B(x_0,r)\subset \Omega\), \[
        \int_{B(x_0,r)}|\nabla u|^p \dd{x}\le C_0r^{n-p+p\alpha},
        \quad p\le n,\alpha>0
    .\] Then we have interior estimate \[
        \|u\|_{C^{0,\alpha}}'\le C_1(C_0,n)
    .\] To be more precise, we have \[
        |u(x)-u(y)|\le C'R^{\alpha},\ \forall\,y\in \overline{B}(x,R)
    .\] 
\end{lemma}
\begin{proof}
    Following similar idea of the other Morrey's lemma, assume \(y\in \partial
    B(x,R)\), then
    \begin{align*}
        |u(x)-u(y)|&\le |u(x)-u(z)|+|u(z)-u(y)| \\
        &\le \frac{C}{R^n}\int_{B(x,R)\cap B(y,R)}
        |u(x)-u(z)|+|u(z)-u(y)|\dd{z} \\
        &\le\frac{C}{R^n}\left(\int_{B(x,R)}\dd{z}\int_x^z|\nabla u|\dd{\ell}
        +\int_{B(y,R)}\dd{z}\int_{y}^{z}|\nabla u|\dd{\ell}\right) \\
        &=\frac{C}{R^n}(I_1+I_2)
    .\end{align*}
    Use polar coordinate at \(x\),
    \begin{align*}
        I_1&=\int_{r\le R}\int_{\mathbb{S}^{n-1}}r^{n-1}\dd{r}\dd{\theta}
        \int_{0}^{r}|\nabla u(x+\ell\theta)|\dd{\ell} \\
        &=\int_{r\le R}r^{n-1}\dd{r}\int_{\ell\le r}
        \int_{\mathbb{S}^{n-1}}|\nabla u(x+\ell\theta)|\dd{\theta}\dd{\ell} \\
        &=\int_{r\le R}r^{n-1}\dd{r}\int_{B(x,r)}\frac{|\nabla u|}{|z|^n}\dd{z}
    .\end{align*}
    Let \(\vphi(r)=\int_{B(x,r)}|\nabla u|\dd{z}\), then \(\vphi'(r)=r^{n-1}
    \int_{\mathbb{S}^{n-1}}|\nabla u (x+r\theta)|\dd{\theta}\), and \[
        \vphi(r)\le C\left(\int_{B(x,r)}|\nabla u|^{p}\right)^{\frac{1}{p}}
        (r^n)^{1-\frac{1}{p}}\le Cr^{n-1+\alpha}
    .\] Then \[
        \int_{\ell\le r}\int_{\mathbb{S}^{n-1}}|\nabla u(x+\ell\theta)|\dd{\ell}
        \dd{\theta}=\int_{0}^{r}\frac{\vphi'(\ell)}{\ell^{n-1}}\dd{\ell}
        =\eval{\frac{\vphi(\ell)}{\ell^{n-1}}}_{\ell=0}^r
        +(n-1)\int_{0}^{r}\frac{\vphi(\ell)}{\ell^n}\dd{\ell}\le Cr^\alpha
    .\] Hence \(I_1\le CR^{n+\alpha}\). We can do similar estimate for \(I_2\).
    In the end we have \[
        |u(x)-u(y)|\le CR^{\alpha},\ \forall\,y\in \partial B(x,R)
    .\] This gives the H\"older continuity.
\end{proof}
Then, we need to squeeze a little more decay from what we get above, to be precise,
we'd like to prove for any \(x_0\notin \mathfrak{S}\), \[
    \frac{1}{\eps^{n-2}}\int_{B(x_0,\eps)}|\nabla u|^2\dd{\mu_g}\le C\eps^{\alpha}
    \quad\text{for some }\alpha>0
.\] The key to the theorem is a uniform decay estimate for small energy:
\begin{lemma}
    There exists \(\eps_0>0\) such that if \(u\) is energy minimizing and 
    \(E(u\big|_{D(1)})\le \eps_0\), we have \[
        \frac{E(r)}{r^{n-2}}\le \frac{1}{2}E(1)+C
    .\] For some \(r_0<1\) and any \(r\le r_0\).
\end{lemma}
Using this lemma, we can iterate and get desired estimate. The condition can be
weaken that without small energy assumption. However I have not figure out yet how
Schoen and Uhlenbeck did that.


Now let's fix the gap we left to Morrey 1.11.1 before. Instead prove the original one,
here we prove for a somehow simper case that easier to deal with.

\noindent\textbf{Reference:}
\begin{itemize}
    \item Chapter 9 of \itshape{}
    J\"urgen Jost. Riemannian Geometry (7th edition)-Springer (2017)
\end{itemize}

Suppose from now on \(\Omega\subset\mathbb{R}^n\) open region with ``good'' boundary,
\(u\in H^1(\Omega,\mathbb{R}^k)\) is a continuous weak solution of system \[
    \int_{\Omega}a^{ij}\pdv{u^{\alpha}}{x^i}\pdv{\vphi^{\alpha}}{x^j}
    =\int_{\Omega}G^{\alpha}(x,u,\nabla u)\vphi^\alpha(x)
\] for any \(\vphi\in H_0^1\cap L^\infty(\Omega,\mathbb{R}^k)\).

\begin{theorem}[A priori Estimate]\hfill\\
    Suppose \(u\in C^0\cap W^{1,4},\cap H^3(B(0,R),\mathbb{R}^k)\), then we have
    interior estimate \[
        \|\nabla^2 u\|_{L^2}'+\|\nabla u\|_{L^4}'^2
        \le C_0R^{\frac{n}{2}}+C_1\|\nabla u\|_{L^2}
    .\] Where primed norm is on \(B(0,\frac{R}{2})\).
\end{theorem}

\section{Convergence Theorems and Bubbling}
Throughout this section, \(u_\alpha\) means a sequence \(u_{\alpha_j},j\ge 0\) with
\(\alpha_j\to 1\).

\begin{prop}
    Let \(u_\alpha\) be a sequence of critical maps of \(E_\alpha\) as \(\alpha\to 1\),
    with \(E_\alpha(u_\alpha)\le B\). Then there exists a subsequence, still denoted
    by \(u_\alpha\), converges to \(s\) weakly in \(W^{1,2}(M,\mathbb{R}^k)\),
    and \(E(u)\le \liminf_{\alpha\to 1}E(u_\alpha)\)
\end{prop}
\begin{remark}\hfill
\begin{itemize}
\item The weak convergence guarantees \(u(x)\in N\) for a.e.\ \(x\) since 
    \(W^{1,2}\hookrightarrow L^p\) compactly.
\item This does not guarantee that \(u\) is continuous nor \(E(u)>0\).
\end{itemize}
\end{remark}

Now, as we mentioned before, we can choose a small \(R\) and a cover of \(M\) by disks
of radius \(R\), such that the metric on each disk is uniformly close to the flat one.
Further, we can assume each point lie in at most \(h\) disks, where \(h\) only depend
on dimension.

Expand the disks to unit size, the \(\alpha\)-energy writes as \[
    E_\alpha(u)=\int_{D}(R^2+|\nabla u|^2)^\alpha\dd{\tilde{\mu}}
.\] Define \(\tilde{E}_\alpha=E_\alpha-R^{2\alpha}\tilde{\mu}(D)\).

\begin{lemma}\label{lem:small-energy-converge}
    Let \(u_\alpha\colon D(R)\to N\) be sequence of critical maps of \(E_\alpha\),
    weakly convergent in \(W^{1,2}\) to \(u\). Then there exists \(\eps>0\) such that
    if \(E(u_\alpha)<\eps\) uniformly, then the convergence is in
    \(C^1(D(\frac{R}{2}),N)\), and \(u\colon D(\frac{R}{2})\to N\) is smooth harmonic
    map.
\end{lemma}
\begin{proof}
    We may wlog assume \(R=1\) by conformally expanding \(D(R)\). Apply main estimate
    with \(p=4\) and \(D'=D(\frac{1}{2})\), we have \[
        \|\nabla u_\alpha\|_{W^{1,4}}'\le C\sqrt{\eps}
    .\] By compact Sobolev embedding \(W^{2,p}\hookrightarrow C^1\), every subsequence
    of \(u_\alpha\) has a further subsequence converges weakly in \(W^{2,4}\), and then
    strongly in \(C^1\). But \(u_\alpha\to u\) weakly in \(W^{1,2}\), the limit must
    be \(u\) uniquely. Hence \(u_\alpha\to u\) in \(C^1(D')\). Take limit in the
    Euler-Lagrange equation we see \(u\) is harmonic.
\end{proof}

\begin{prop}
    Let \(U\subset M\) open, \(u_\alpha\) be sequence critical maps of \(E_\alpha\)
    and \(u_\alpha\to u\) weakly in \(W^{1,2}\). \(E_\alpha(u_\alpha)\le B\) uniformly
    in \(\alpha\). Let \(U_m=\{x\in U:D(x,2^{-m+1})\subset U\}\), then there exists
    a subsequence and a finite number of points \(x_{1,m},\ldots,x_{l,m}\), where
    \(l\) depends on \(B,N\) but not on \(m\), such that \[
        u_\alpha\longrightarrow u\quad\text{ in }
        C^1(U_m\setminus\bigcup_{i}D(x_i,2^{-m}),N)
    .\] 
\end{prop}
\begin{proof}
    Cover \(U_m\) by disks \(D(x_i,2^{-m+1})\subset U\), such that each point is
    covered at most \(h\) times. Then \[
        \sum_i \int_{D(x_i,2^{-m+1})}|\nabla u_\alpha|^2\dd{\mu}\le Bh
    .\] Hence for each \(\alpha\), there are at most \(\frac{Bh}{\eps}\) disks \st\ 
    \(E\left(u_\alpha\big|_{D(x_i,2^{-m})}\right)>\eps\).
    We can choose a subsequence such that there is eventually \(l<\frac{Bh}{\eps}+1\)
    bad disks. Then apply \cref{lem:small-energy-converge} on each good disk, we see
    there is a subsequence that converges to \(u\) in \(C^1\) except on
    \(\bigcup_{i}D(x_i,2^{-m})\).
\end{proof}

\begin{theorem}\label{thm:alpha-convergence}
    Under the same assumption on \(u_\alpha\), there exists a subsequence and finitely
    many points \(\{x_1,\ldots,x_l\}\) where \(l\) does not depend on \(U\), such that
    \(u_\alpha\to u\) in \(C^1(U\setminus\{x_1,\ldots,x_l\},N)\).
    Moreover, this \(u\) is harmonic and smooth.
\end{theorem}
\begin{proof}
    Consider a sequence for \(U_m\) in last proposition, we can do a diagonal argument
    to obtain a subsequence that converges to \(u\) in \(C^1\) on \[
        \bigcup_{m\ge 0}\left(U_m\setminus \bigcup_{i=1}^l D(x_{i,m},2^{-m})\right)
        =U\setminus\bigcap_{m\ge 0}\bigcup_{i=1}^l D(x_{i,m},2^{-m})
        =U\setminus\{\text{At most }l\text{ points}\}
    .\] By \cref{thm:removable-singularity}, we see \(u\) is harmonic and smooth.
\end{proof}
\begin{remark}
    For now, there is no guarantee that \(u\) is not trivial nor we can extend the
    convergence to exception points.
\end{remark}

\begin{lemma}\label{lem:bounded-du-converge}
    Under same assumption on \(u_\alpha\) as in \cref{thm:alpha-convergence}, suppose
    there exists \(\delta>0\) such that  \[
        \sup_{D(x_i,\delta)}|\nabla u_\alpha|\le B<\infty
    .\] Then \(u_\alpha\to u\) in \(C^1(D(x_i,\delta),N)\).
\end{lemma}
\begin{proof}
    In a sufficiently small disk \(D(x_i,R)\subset D(x_i,\delta)\), we have \[
        \int_{D(x_i,R)}|\nabla u_\alpha|^2\dd{\mu}\le \pi R^2B^2<\eps
    .\] Then apply \cref{lem:small-energy-converge}.
\end{proof}

\begin{theorem}\label{thm:energy-loss}
    Let \(u_\alpha\) be a sequence of critical maps of \(E_\alpha\) with \(E_\alpha(
    u_\alpha)\le B\) uniformly in \(\alpha\to 1\) and \(u_\alpha\to u\) in \(C^1\) on
    \(M\setminus\{x_1,\ldots,x_l\}\) but not on \(M\setminus\{x_2,\ldots,x_l\}\).
    Then there exists a nontrivial harmonic map \(\tilde{u}\colon \mathbb{S}^2\to N\)
    such that \[
        \tilde{u}(\mathbb{S}^2)\subset \bigcap_{m\ge 0}\bigcap_{\alpha\to 1}
        \bigcup_{\beta\le \alpha}u_\beta(D(x_1,2^{-m}))
    .\] Moreover, \[
        E(u)+E(\tilde{u})\le \liminf_{\alpha\to 1}E(u_\alpha)
    .\] 
\end{theorem}
\begin{proof}
    Let \(b_\alpha=\sup_{D(x_1,2^{-m})}|\nabla u_\alpha|\) and let \(x_\alpha\) be
    the point where the maximum is taken. By choosing a subsequence we may assume
    \(b_\alpha\to \infty\), otherwise we can apply \cref{lem:bounded-du-converge}
    to remove the singularity.

    Define \(\tilde{u}_\alpha(x)=u_\alpha(x_\alpha+\frac{x}{b_\alpha})\), then 
    \(\tilde{u}_\alpha\colon D(0,2^{-m}b_\alpha)\to N\) is critical map of
    \(E_\alpha\) and \(|\nabla\tilde{u}_\alpha|\le 1\) in \(D(x_i,2^{-m}b_\alpha)\).
    Note that these disks have radius going to \(\infty\) as \(\alpha\to 1\), and
    the metrics on the disks converges to the Euclidean metric.
    Combine \cref{thm:alpha-convergence} and \cref{lem:bounded-du-converge}, we see 
    by choosing a subsequence \(\tilde{u}_\alpha\to \tilde{u}\) in \(C^1\) on
    \(D(R)\) for any \(R<\infty\). Where \(\tilde{u}\colon D(R)\to N\) is 
    harmonic and smooth. Moreover, note that \(|\nabla \tilde{u}_\alpha(0)|=1\), so
    \(|\nabla\tilde{u}(0)|=1\). Thus \(\tilde{u}\) cannot be constant. By a diagonal
    argument, we got a subsequence \(\tilde{u}_\alpha\to \tilde{u}\) in \(C^1(
    \mathbb{R}^2,N)\).
    \begin{align*}
        E(\tilde{u})&+E\left(u\big|_{M\setminus D(x_1,2^{-m})}\right) \\
        &\le\liminf_{\alpha\to 1}
        E\left(\tilde{u}_\alpha\big|_{D(x_1,2^{-m}b_\alpha)}\right)
        +\liminf_{\alpha\to 1}E\left(u_\alpha\big|_{M\setminus D(x_1,2^{-m})}\right)\\
        &\le \liminf_{\alpha\to 1}E(u_\alpha)
    .\end{align*}
    Note that choice of \(x_\alpha\) and \(b_\alpha\) is eventually not depend on
    \(m\), so is \(\tilde{u}_\alpha\). Hence we can let \(m\to \infty\) and get \[
        E(\tilde{u})+E(u)\le \liminf_{\alpha\to 1}E(u_\alpha)
    .\] Note that we have been choosing subsequence multiple times during the proof,
    but we can in the first place choose a subsequence such that \(\lim_{\alpha\to 1}
    \) exits and equal to the \(\liminf\). Then the \(\liminf\) at last is valid.
    Apply \cref{thm:removable-singularity}, we see \(\tilde{u}\) can be extend
    to a harmonic map on \(\mathbb{S}^2\).
\end{proof}

\begin{theorem}\label{thm:main-converge}
    Let \(u_\alpha\) be a sequence of critical maps of \(E_\alpha\) for
    \(\alpha\to 1\), and \(u_\alpha\to u\) weakly in \(W^{1,2}(M,\mathbb{R}^k)\).
    Then either \(u_\alpha\to u\) in \(C^1(M,N)\), or there exists a non-trivial
    harmonic map \(\tilde{u}:\mathbb{S}^2\to N\) with \[
        \tilde{u}(\mathbb{S}^2)\subset \bigcap_{\alpha\to 1}
        \overline{\bigcup_{\beta<\alpha}u_\beta(M)}
    .\]
    Moreover, \[
        E(u)+E(\tilde{u})\le \liminf_{\alpha\to 1}E(u_\alpha)
    .\] 
\end{theorem}

In theorems above, we saw a phenomenon which is called ``bubbling'', \ie\ in limit
process of critical maps, energy may concentrate to a point and get lost in limit
map. But one can collect these energy by expanding neighbourhoods of the concentrate
point to create a map on sphere carrying those energy, looks like one blow a bubble
at the point. However, the argument above cannot regain all the energy. Some energy
may still get lost on the ``neck''. To collect them, we need a more careful argument.

\noindent\textbf{Reference:}
\begin{itemize}\itshape{}
\item Thomas H. Parker. Bubble tree convergence for harmonic maps (1996)
\end{itemize}

\section{The Final Results}
In this section, we put all above together to prove the final results.

\begin{lemma}
    If \(\pi_2(N)=0\), \(u,\tilde{u}\colon M\to N\) which agree outside a disk \(D\),
    then \(\tilde{u}\) is homotopic to \(u\).
\end{lemma}
\begin{proof}
    Let \(f\colon \mathbb{S}^2\to N\) be \(u\) on north hemisphere, \(\tilde{u}\) on 
    south hemisphere. By assumption \(f\) is homotopic to constant, \ie\ \(f\) extends
    to \(F\colon D^3\to N\). This \(F\) gives homotopy from \(u\) to \(\tilde{u}\).
\end{proof}

\begin{theorem}\label{thm:no-pi2}
    If \(N\) is compact and \(\pi_2(N)=0\), then there exists a minimizing harmonic
    map in every homotopy class in \(C^0(M,N)\).
\end{theorem}
\begin{proof}
    Let \(u_\alpha\colon M\to N\) be minimizing map for \(E_\alpha\) in a fixed
    homotopy class with \(E_\alpha(u_\alpha)<(1+B^2)^\alpha\). 
    By \cref{thm:alpha-convergence}, let \(U=M\), we can choose a subsequence such
    that \(u_\alpha\to u\) in \(C^1(M\setminus\{x_1,\ldots,x_l\},N)\) with \(u\colon
    M\to N\) harmonic. We claim this convergence is \(C^1\) on whole \(M\).

    Fix a small \(R\), such that \(D(x_i,R)\) are disjoint. Fix \(i\), choose
    coordinate such that \(x_i\) is origin. Let \(\chi\) be a smooth increasing
    function which is \(1\) for \(r\ge 1\) and is \(0\) for \(r\le \frac{1}{2}\).
    Define
    \begin{equation}\label{eq:def-modify-u}
        \hat{u}_\alpha(x)=\exp_{u(x)}\left(\chi\left(\frac{|x|}{R}\right)
        \exp^{-1}_{u(x)}(u_\alpha(x))\right)
    .\end{equation}
    Then \(\hat{u}_\alpha\colon D(R)\to N\) agrees with \(u_\alpha\) on the
    boundary of the disk and equals to \(u\) around the center. The definition is
    valid since \(u_\alpha\) is \(C^1\) close to \(u\) for \(\frac{1}{2}\le r\le 1\).
    Then \(\hat{u}_\alpha\to u\) in \(C^1(D(R),N)\), so \[
        \lim_{\alpha\to 1}E_\alpha(\hat{u}_\alpha\big|_{D(R)})-1=E(u\big|_{D(R)})
    .\] Since \(\pi_2(N)=0\), \(u_\alpha\) and \(\hat{u}_\alpha\) are homotopic.
    But \(u_\alpha\) is minimizer, hence \(E_\alpha(u_\alpha\big|_{D(R)})\le 
    E_\alpha(\hat{u}_\alpha\big|_{D(R)})\), \[
        \limsup_{\alpha\to 1}E_\alpha(u_\alpha\big|_{D(R)})
        \le E(u\big|_{D(R)})\le \pi R^2\|u\|_{W^{1,\infty}}^2
    .\] Choose \(R\) small and apply \cref{lem:small-energy-converge}, we see
    the original convergence is \(C^1\) with no exceptional point.
    Since \(u_\alpha\) minimize \(E_\alpha\), \(u\) must minimize \(E\) in
    the same homotopy class.
\end{proof}

\begin{theorem}\label{thm:conj-class}
    Every conjugacy class of homomorphism from \(\pi_1(M)\) to \(\pi_1(N)\)
    is induced by a minimizing harmonic map from \(M\) to \(N\).
\end{theorem}
\begin{proof}
    Essentially same as last theorem. Note when we replace \(u_\alpha\) by \(\hat{u}_
    \alpha\), we do not change their induced homomorphism on \(\pi_1\).
\end{proof}
\begin{remark}\hfill
\begin{itemize}
\item If \(\pi_2(N)=0\), then \cref{thm:no-pi2} implies \cref{thm:conj-class} since
    in this case conjugacy classes of homomorphisms from \(\pi_1(M)\) to \(\pi_1(N)\)
    are canonically identified with components of \(C^0(M,N)\).
\item These two theorems had been obtained independently by Lemaire and Schoen-Yau
    using other methods earlier.
\end{itemize}
\end{remark}

\begin{theorem}
    If the universal covering of \(N\) is not contractible, then there exists
    a non-trivial harmonic map \(u\colon \mathbb{S}^2\to N\).
\end{theorem}
\begin{proof}
    Apply \cref{thm:alpha-existence} to get nontrivial critical maps \(u_\alpha\) of
    \(E_\alpha\) with \(1+\eps<E_\alpha(u_\alpha)<(1+B^2)^{\alpha}\). Then there is a
    subsequence converges to \(u\) in \(C^1(\mathbb{S}^2\setminus\{x_1,\ldots,x_l\}
    ,N)\) and \(u\) is harmonic. If \(u\) is nontrivial, we are done. If \(u\) is
    constant, by \cref{thm:energy-loss} there exists another nontrivial harmonic map
    \(\tilde{u}\).
\end{proof}
\begin{remark}
    The condition on \(N\) cannot be dropped. If \(N\) has non-positive curvature,
    there will be no nontrivial harmonic map from \(\mathbb{S}^2\) to \(N\) and
    \(\tilde{N}\) is trivial.
\end{remark}

Combine this theorem with \cref{thm:harmonic-is-conformal} we get:
\begin{theorem}
    If the universal covering of \(N\) is not contractible, then there exists a
    non-trivial smooth conformal branched minimal immersion \(u\colon \mathbb{S}^2
    \to N\).
\end{theorem}

Next, we prove a refined theorem for case \(\pi_2(N)\neq 0\).
\begin{lemma}\label{lem:no-trivial-bubble}
    Let \(u_\alpha\colon \mathbb{S}^2\to N,\alpha\to 1\) be a sequence of non-trivial
    \(E_\alpha\)-critical maps, which converges to \(u\) in \(C^1\) except at north
    pole \(p\). Then \(u\) is non-constant.
\end{lemma}
\begin{remark}
    This lemma claims that a trivial bubble must have multiple bubbles on it.
\end{remark}
\begin{proof}
    Let \((r,\theta)\in [0,\pi]\times [0,2\pi]\) be the standard spherical
    coordinate on \(\mathbb{S}^2\), \(S^+,S^-\) be northern and southern hemisphere
    respectively. We omit \(\alpha\) in \(u_\alpha\) in calculation for simplicity.
    Consider variation \[
        v(r,\theta)=u_r(r,\theta)\sin r
    .\] We have \[
        0=E_\alpha'(u)=2\alpha\int_{\mathbb{S}^2}(1+|\nabla u|^2)^{\alpha-1}
        \left<\nabla u,\nabla v\right> \dd{\mu_g}
    .\] We calculate the inner product term
    \begin{align*}
        \left<\nabla u,\nabla v\right> 
        &=u_r\cdot (u_r\sin r)_r+\frac{1}{\sin^2 r}
        (u_\theta\cdot (u_r\sin r)_{\theta}) \\
        &=\frac{1}{2}\left(\pd{r}|\nabla u|^2\right)\sin r
        +|\nabla u|^2\cos r
    .\end{align*}
    Then
    \begin{align*}
        0&=2\alpha\int_{\mathbb{S}^2}(1+|\nabla u|^2)^{\alpha-1}\left(
        \frac{1}{2}\left(\pd{r}|\nabla u|^2\right)\sin r+|\nabla u|^2\cos r\right)
        \dd{\mu} \\
        &=\int_{0}^{2\pi}\int_{0}^{\pi}\pd{r}((1+|\nabla u|^2)^\alpha-1)\sin^2 r
        +2\alpha(1+|\nabla u|^2)^{\alpha-1}|\nabla u|^2\sin r\cos r\dd{r}\dd{\theta}\\
        &=2\int_{\mathbb{S}^2}\left(1-(1+|\nabla u|^2)^\alpha+
        \alpha(1+|\nabla u|^2)^\alpha|\nabla u|^2\right)\cos r\dd{\mu_g}
    .\end{align*}
    Note that by Taylor's expansion of \((1+r)^\alpha\), for \(\lambda>0\), \[
        1-(1+\lambda)^\alpha+\alpha(1+\lambda)^{\alpha-1}\lambda\sim
        \frac{1}{2}\alpha(\alpha-1)(1+\lambda)^{\alpha-2}\lambda^2
    .\] More precisely, \[
    \frac{1-(1+\lambda)^\alpha+\alpha(1+\lambda)^{\alpha-1}\lambda}
    {(\alpha-1)(1+\lambda)^{\alpha-2}\lambda^2}
    \text{ lie with in }1\text{ and }\frac{\alpha}{2}
    .\] Divide the integral to northern and southern parts, we get for \(\alpha<2\),
    \[
        \frac{\alpha}{2}\int_{r\le \frac{\pi}{2}}(1+|\nabla u_\alpha|^2)^{\alpha-2}
        |\nabla u_\alpha|^4 \cos r\dd{\mu}
        \le \int_{r\ge \frac{\pi}{2}}(1+|\nabla u_\alpha|^2)^{\alpha-2}
        |\nabla u_\alpha|^4(-\cos r)\dd{\mu}
    .\] Assume \(u\) is constant, by \(C^1\) convergence away from \(p\),
    \begin{equation}
        \text{RHS }\to \int_{\mathbb{S}^-}(1+|\nabla u|^2)^{\alpha-2}|\nabla u|^4
        (-\cos r)\dd{\mu}=0\quad\text{as }\alpha\to 0
    .\end{equation}
    However,
    \begin{align*}
        \text{LHS}&\ge \frac{\alpha}{2}\cos 1\int_{r\le 1}
        (1+|\nabla u_\alpha|^2)^{\alpha-2}|\nabla u_\alpha|^4\dd{\mu} \\
        &\ge c_1 \int_{r\le 1,|\nabla u_\alpha|\ge\delta}
        (1+|\nabla u_\alpha|^2)^{\alpha-1}|\nabla u_\alpha|^2\cdot
        \frac{\delta^2}{1+\delta^2}\dd{\mu} \\
        &\ge \frac{c_1\delta^2}{1+\delta^2}\int_{r\le 1,|\nabla u_\alpha|\ge \delta}
        |\nabla u_\alpha|^2\dd{\mu} \\
        &\ge\frac{c_1\delta^2}{1+\delta^2}\left(E(u_\alpha\big|_{D(1)})
        -\op{Vol}(D)\delta^2\right)
    .\end{align*}
    Note we have energy gap \(E(u_\alpha\big|_{D(1)})\ge \eps\) uniform in \(\alpha\).
    Choose \(\delta\) sufficiently small, we conclude a contradiction.
\end{proof}

Now note that the components of \(C^0(\mathbb{S}^2,N)\) can be canonically identified
with \(\pi_2(N)\) acted by \(\pi_1(N)\). For a homotopy class \(\Gamma\) we define \[
    \mathcal{E}_\Gamma=\inf_{u\in \Gamma\cap W^{1,\infty}}\{E(u)\}
    =\lim_{\alpha\to 1}\{\inf \tilde{E}_\alpha(u):u\in \Gamma\cap W^{1,\infty}\}
.\] Note that \(\mathcal{E}_\Gamma=0\) iff \(\Gamma\) is trivial and otherwise
\(\mathcal{E}_\Gamma\ge \eps>0\) (energy gap).

\begin{lemma}
    Let \(\Gamma\in \pi_0(C^0(\mathbb{S}^2,N))\). Then either \(\Gamma\) contains
    a minimizing harmonic map or for any \(\delta>0\), there exists non-trivial
    homotopy class \(\Gamma_1+\Gamma_2\supset \Gamma\) and  \[
        \Eps_{\Gamma_1}+\Eps_{\Gamma_2}<\Eps_\Gamma+\delta
    .\] 
\end{lemma}
\begin{proof}
    Choose a sequence \(u_\alpha\in\Gamma\) minimizing \(E_\alpha\) such that
    \(u_\alpha\) converge weakly in \(W^{1,2}(M,\mathbb{R}^k)\) to \(u\in W^{1,2}
    (M,N)\), and the convergence is \(C^1\) except at \(\{x_1,\ldots,x_l\}\).
    We can assume \[
        \lim_{m\to \infty}\limsup_{\alpha\to 1}
        \tilde{E}_\alpha(u_\alpha\big|_{D(x_i,2^{-m})})\ge \eps
    ,\] for otherwise we can apply \cref{lem:small-energy-converge} to remove the
    singularity. If now \(l=0\), we are done with \(u\). If not, choose small
    disk \(D(R)\) around \(x_1\), use same construction as in \cref{eq:def-modify-u}
    to get \(\hat{u}\). Define \[
        v_\alpha(x)=\begin{cases}
            \hat{u}_\alpha\circ f(x), & x\in \mathbb{S}^2\setminus D(R) \\
            u_\alpha(x), & x\in D(R)
        \end{cases}
    .\] Where \(f\colon \mathbb{S}^2\setminus D(R)\to D(R)\) is the conformal
    reflection. Let \(\Gamma_1\) and \(\Gamma_2\) be the homotopy class of \(\hat{u}
    _\alpha\) and \(v_\alpha\) respectively. Note \(\hat{u}_\alpha\) agrees with
    \(u_\alpha\) on \(\mathbb{S}^2\setminus D(R)\). We have \(\Gamma\subset 
    \Gamma_1+\Gamma_2\). From conformality of \(f\),
    \begin{align*}
        \lim_{\alpha \to 1} \tilde{E}_\alpha(\hat{u}_\alpha)&=\lim_{\alpha\to 1}
        \tilde{E}_\alpha(u_\alpha\big|_{\mathbb{S}^2\setminus D(R)})
        +E(u\big|_{D(R)}) \\
        \lim_{\alpha \to 1} \tilde{E}_\alpha(v_\alpha)&=\lim_{\alpha\to 1}
        \tilde{E}_\alpha(u_\alpha\big|_{D(R)})+E(u\big|_{D(R)})
    \end{align*}
    If \(\pi R^2\|\nabla u\|_{L^\infty}\le\delta\), we can choose
    \(\alpha\) close to 1 such that 
    \begin{align*}
        \tilde{E}_\alpha(\hat{u}_\alpha)&\le
        \tilde{E}_\alpha(u_\alpha\big|_{\mathbb{S}^2\setminus D(R)})+2\delta \\ 
        \tilde{E}_\alpha(v_\alpha)&\le \tilde{E}_\alpha(u_\alpha\big|_{D(R)})+2\delta
    .\end{align*}
    And then \[
        \Eps_{\Gamma_1}+\Eps_{\Gamma_2}\le \tilde{E}_\alpha(\hat{u}_\alpha)
        +\tilde{E}_\alpha(v_\alpha)\le \tilde{E}_\alpha(u_\alpha)+4\delta
        <\Eps_\Gamma+5\delta
    .\] Choose \(\delta<\frac{\eps}{10}\). Since \(\tilde{E}_\alpha(v_\alpha)\ge
    \tilde{E}_\alpha(v_\alpha\big|_{D(R)})\ge\eps\), \[
        \Eps_{\Gamma_1}\le \tilde{E}_\alpha(\hat{u}_\alpha)
        \le \Eps_{\Gamma}+5\delta-\eps<\Eps_\Gamma
    .\] Hence \(\Gamma_1\neq \Gamma\) and \(\Gamma_2\neq 0\). Now we need to prove
    \(\Gamma_1\neq 0\). Choose \(\alpha\) close to 1, \[
        \tilde{E}_\alpha(\hat{u}_\alpha)>\tilde{E}_\alpha(u_\alpha\big|_{\mathbb{S}^2
        \setminus D(R)})\ge E(u\big|_{\mathbb{S}^2\setminus D(R)})-\delta
        >E(u)-2\delta
    .\] If \(u\) is non-trivial, \(E(u)>\eps\) and RHS \(>8\delta\). If \(u\) is
    trivial, by \cref{lem:no-trivial-bubble} there must be \(x_2\neq x_1\) where
    \(C^1\) convergence fails. Then \(\lim_{\alpha\to 1}\tilde{E}_\alpha(u_\alpha
    \big|_{D(x_2,R)})>\eps\) and \(\tilde{E}_\alpha(\hat{u}_\alpha)>
    \tilde{E}_\alpha(u_\alpha\big|_{D(x_2,R)})\) for small \(R\). In either case,
    \(\tilde{E}_\alpha(\hat{u}_\alpha)>5\delta\) and \(\Gamma_2\neq \Gamma\), and
    finally \(\Gamma_1\neq 0\).
\end{proof}

\begin{theorem}
    There exists a set of free homotopy classes \(\Gamma_i\subset\pi_0(C^0(
    \mathbb{S}^2,N))\), such that elements \(\gamma\in \Gamma_i\) form a generating 
    set for \(\pi_1(N)\acts \pi_2(N)\), and each \(\Gamma_i\) contains a minimizing
    harmonic map \(u_i\colon \mathbb{S}^2\to N\).
\end{theorem}
\begin{proof}
    Let \(\Gamma_i\) be all homotopy class containing minimizing harmonic maps,
    \(H\) be the subgroup generated by \(\bigcup_{i}\Gamma_i\). Suppose \(H\) is
    proper subgroup, then there is \(\Gamma\notin H\) such that if \(\Eps_{\Gamma'}
    \le \Eps_{\Gamma}-\frac{\eps}{2}\), then \(\Gamma'\in P\).

    By assumption \(\Gamma\) does not contain a minimizing harmonic map, by previous
    lemma there exists \(\Gamma_1,\Gamma_2\) \st\ \(\Gamma_1+\Gamma_2\supset \Gamma\)
    and \(\Eps_{\Gamma_1}+\Eps_{\Gamma_2}<\Eps_\Gamma+\frac{\eps}{2}\).
    But \(\Gamma_1,\Gamma_2\) are non-trivial so \(\Eps_{\Gamma_j}\ge \eps,j=1,2\).
    Hence \(\Eps_{\Gamma_j}<\Eps_\Gamma-\frac{\eps}{2}\) for \(j=1,2\), \ie\ 
    \(\Gamma_j\in H\). This leads to a contradiction.
\end{proof}

Combine this theorem with \cref{thm:harmonic-is-conformal} we get final result:
\begin{theorem}
    There exists a set of free homotopy classes in \(\pi_0(C^0(\mathbb{S}^2,N))\)
    generating \(\pi_1(N)\acts \pi_2(N)\), each of which contains a conformal
    branched immersion of \(\mathbb{S}^2\) with least area in the same class.
\end{theorem}

% \color{green} Further results to remove ``branched'' and ``immersion'' go here.

\end{document}

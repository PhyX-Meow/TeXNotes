% !TeX program = xelatex
\documentclass[UTF8,12pt]{article}
\usepackage[xcharter]{phyxmeow}
\geometry{a4paper,scale=0.8}
\allowdisplaybreaks{}

\numberwithin{theorem}{section}
\numberwithin{equation}{section}

\newcommand{\hooklongrightarrow}{\lhook\joinrel\longrightarrow}
\newcommand{\gap}{\hspace{0pt}}

\addbibresource{Yau-Seminar.bib}

\title{Variational Method in Dirichlet's Problem}
\author{XueHaotian}

\usepackage{datetime}
\newdate{date-1}{27}{10}{2022}
\newdate{date-2}{03}{11}{2022}
\date{\displaydate{date-1}}

\begin{document}
\maketitle

\section{Introduction}
Goal of this report is to solve the following so-called \emph{Dirichlet Problem} on
bounded region \(\Omega\subset \mathbb{R}^n\):
\begin{problem}\label{porb:boundary-value}
    Suppose \(\Omega\subset \mathbb{R}^n\) is a open set, given some function 
    \(f\) defined on \(\partial\Omega\), find a function \(u\) satisfying \[
        \begin{cases}
            -\triangle u=0 & \text{in }\Omega \\
            \eval{u}_{\partial\Omega}=f &
        \end{cases}
    \]
\end{problem}
\begin{remark}
    The problem stated above is not good enough, because when we talk solving an
    equation, we need to clarify in what means, and under what conditions.
\end{remark}
We first assume \(n\ge 2\) since one dimensional problem is not interesting, only
linear solutions exists. Now the problem is, what ``\(\eval{u}_{\partial\Omega}=f\)''
means? More precisely, we need to clarify
\begin{enumerate}[(1)]
    \item What \(\partial\Omega\) should be?
    \item What \(f\) should be?
    \item What the restriction \(\eval{u}_{\partial\Omega}\) mean?
\end{enumerate}

Here in this report, we only deal with classic case, where \(\partial\Omega\) is
\(C^1\) submanifold, \(f\) is continuous function on \(\partial\Omega\) and the
restriction means \(u\) is continuous to the boundary and equal to \(f\) on it.
For other case, e.g.1 Smooth boundary with \(f\in H^{\frac{1}{2}}(\partial\Omega)\),
one can reference to Pin Yu's book for an existence and uniqueness theorem using 
Riesz's representation theorem. E.g.2 \(n=2\), then the boundary condition can be
weakened, this case may be covered in later report by Lin Yiran,
reference:~\cite{courant_dirichlets_1977}.

Now under above assumption, the problem is still not good, boundary condition is
difficult to deal with. We consider to transform boundary condition to zero, at
cost of harmonic condition breaking. Let \(\tilde{u}=u-u_0\), then \[
    \begin{cases}
        -\triangle \tilde{u}=\triangle u_0=\colon g \\
        \eval{\tilde{u}}_{\partial\Omega}=f-\eval{u_0}_{\partial\Omega}
    \end{cases}
\] We'd like to choose \(u_0\) smooth \st\ \(\eval{u_0}_{\partial\Omega}=f\).
Consider a partition of unity, subordinate to a slice chart cover on \(\partial
\Omega\), we can construct \(u_0\) piecewise then add them together. On one of these
chart, with a smooth parametrization, we can assume \(f\) is compactly supported
function on \(\{x_n=0\}\), and to find \(u_0\) smooth in upper-halfspace with
boundary value \(f\). Write \(\xi=(x_1,\ldots,x_{n-1}), y=x_n\), consider Poisson
kernel \[
    P(y,\xi)=\frac{\Gamma(\frac{n}{2})}{\pi^{\frac{n}{2}}}
    \frac{y}{(|\xi|^2+y^2)^{\frac{n}{2}}}
.\] Let \(u_0=P(y,\xi)*_{\xi}f(\xi)\), easily to see \(u_0\) is smooth in \(\{y>0\}\)
and continuous to boundary.

Now problem turns into:
\begin{problem}\label{prob:boundary-zero}
    Suppose \(\Omega\subset \mathbb{R}^n\) is an open set, given function \(f\)
    defined in \(\Omega\), find a function \(u\) satisfying \[
        \begin{cases}
            -\triangle u=f & \text{in }\Omega \\
            \eval{u}_{\partial\Omega}=0
        \end{cases}
    \] 
\end{problem}
Still we need assumptions to \(f\) and \(u\). Though \(f\) here should be smooth
if we come from last problem, we only assume \(f\in L^2(\Omega)\) for proving
existence and uniqueness, then consider smooth \(f\) for regularity. \(u\) here is
assumed to lie in so-called \emph{Sobolev-Hilbert space} \(H_0^1(\Omega)\), which is
intuitively a generalization of boundary zero functions to \(L^2\) functions with
a \(L^2\) derivative, in next section we'll see what this means in detail.
No condition is assumed on \(\Omega\) here.

\section{Basics of Sobolev Spaces}
All functions in this section is considered as locally integrable functions, and
equal should be understood as equal almost everywhere. Measure is always assumed to
be standard Lebesgue measure on \(\mathbb{R}^n\). And all capital \(C\) as a number
in equations means a constant which is not important for inequalities, they may not
equal to each other.
% Maybe a brief intro for distributions here?
\begin{definition}[Sobolev-Hilbert space on \(\mathbb{R}^n\)]
    Given an non-negative real number \(s\), consider all tempered distribution
    (Here may just consider \(L^2\) functions) \(u\) satisfying:
    \begin{enumerate}[(a)]
        \item \(\hat{u}\in L_{\text{loc}}^1\) is locally integrable;
        \item \((1+|\xi|^2)^{\frac{s}{2}}\hat{u}(\xi)\) is \(L^2\) function.
    \end{enumerate}
    Where \(\hat{u}\) is Fourier transformation of \(u\).
    For such \(u\), define its Sobolev norm \[
        \|u\|_{H^s}:=\left(\int_{\mathbb{R}^n} (1+|\xi|^2)^s|\hat{u}(\xi)|^2
        \dd{\xi}\right)^{\frac{1}{2}}
    .\] The set of all such \(u\) is called Sobolev-Hilbert space of index \(s\),
    denoted by \(H^s(\mathbb{R}^n)\). Obviously it's a complex linear space,
    with norm induced by inner product \[
        \left<u,v\right>_{H^s}=\int_{\mathbb{R}^n}(1+|\xi|^2)^s
        \overline{\hat{u}(\xi)}\hat{v}(\xi)\dd{\xi}
    .\]
\end{definition}
Notice \(H^0(\mathbb{R}^n)\) is just \(L^2(\mathbb{R}^n)\), which followed
immediately by Plancherel's formula. If we consider measure \[
    \mu_s=(1+|\xi|^2)^{s}\dd{\xi}
.\] We see \(u\in H^s(\mathbb{R}^n)\iff \hat{u}\in L^2(\mathbb{R}^n,\dd{\mu_s})\).
With this observation we can prove \(H^s(\mathbb{R}^n)\) is a Hilbert space. For
detail one may refer to Analysis123, page 827.
% Proof may be filled later.

Also, here we only defined Sobolev spaces of non-negative index, we can also define
\(H^{-s}\) in similar way, and a beautiful theorem states \(H^{-s}\) is naturally
the dual space of \(H^s\). For detail, again, one may refer to Analysis123.

\hfill

We have following properties of \(H^s\):
\begin{prop}[Sobolev space of integer index]\hfill\par
    Assume \(k\ge 0\) is integer, then \[
        H^k(\mathbb{R}^n)=\{u\in L^2(\mathbb{R}^n):\partial^\alpha
        u\in L^2(\mathbb{R}^n), \forall\,\text{multiple index }|\alpha|\le k\}
    .\] 
\end{prop}

\begin{prop}
    For any \(s\ge 0\), compactly supported smooth functions \(C_0^\infty
    (\mathbb{R}^n)\) is dense subspace of \(H^s(\mathbb{R}^n)\) under Sobolev norm.
\end{prop}
\begin{proof}
    First use \((1-\triangle)^{\frac{s}{2}}\colon H^0\to H^s\) isomorphism,
    and \(\mathcal{S}\) dense in \(L^2\) to reduce case to prove
    \(C_0^\infty\subset \mathcal{S}\) dense under \(H^s\) norm. Then do it the hard
    way. Detail fill later.
\end{proof}

\begin{theorem}[Sobolev embedding into \(L^\infty\)]\hfill\par
    Suppose \(s>\frac{n}{2}\), then we have continuous inclusion \[
        H^s(\mathbb{R}^n)\hooklongrightarrow L^\infty(\mathbb{R}^n)
    .\] \ie\ \(\exists\,C\) depends only on \(s\) \st\ \[
        \|u\|_{L^\infty}\le C\|u\|_{H^s}
    .\] Further, \(u\) is continuous (in the sense it equals a continuous function
    a.e.) and tends to 0 as \(|x|\to \infty\).
\end{theorem}
\begin{proof}
    To prove \(u\) is \(L^\infty\), we turn to prove \(\hat{u}\) is a \(L^1\)
    function, then \(u\) is bounded by \(\|\hat{u}\|_{L^1}\) and is actually
    a \(C_{\circ}\) function.
    \begin{align*}
        \int_{\mathbb{R}^n}|\hat{u}|\dd{\xi}
        &=\int_{\mathbb{R}^n}\frac{1}{(1+|\xi|^2)^{\frac{s}{2}}}\cdot 
        (1+|\xi|^2)^{\frac{s}{2}}|\hat{u}|\dd{\xi} \\
        &\le \left(\int_{\mathbb{R}^n}\frac{1}{(1+|\xi|^2)^s}\dd{\xi}\right)
        ^{\frac{1}{2}}\left(\int_{\mathbb{R}^n}(1+|\xi|^2)^s |\hat{u}|^2\dd{\xi}
        \right)^{\frac{1}{2}} \\
        &=C\|u\|_{H^s}
    .\end{align*}
    Notice \((1+|\xi|^2)^{-s}\) is integrable since \(s>\frac{n}{2}\).
\end{proof}

\begin{corollary}
    Suppose \(s>\frac{n}{2}+k\), for \(k\) non-negative integer, then \[
        H^s(\mathbb{R}^n)\hooklongrightarrow C^k(\mathbb{R}^n)
    .\] Moreover the inclusion is continuous if we choose norm on
    \(C^k(\mathbb{R}^n)\) be sum of \(L^\infty\) norm of \(\le k\) order derivatives.
\end{corollary}

\begin{prop}[Sobolev space of infinite index]\hfill\par
    Notice by definition we have \(H^{s'}\hookrightarrow H^{s}\) for \(s<s'\).
    Let \(H^\infty(\mathbb{R}^n)=\bigcap_{s\ge 0}H^s(\mathbb{R}^n)\), then we have \[
        \mathcal{S}(\mathbb{R}^n)\subset H^{\infty}(\mathbb{R}^n)\subset 
        C^\infty(\mathbb{R}^n)
    .\] Where left hand side is the Schwarz function space.
\end{prop}

\begin{theorem}
    If \(s>\frac{n}{2}\), then \(H^s(\mathbb{R}^n)\) is an algebra, \ie\ for any
    \(u,v\in H^s(\mathbb{R}^n)\), we have \(uv\in H^s(\mathbb{R}^n)\), and
    there is constant depending only on \(s\) such that \[
        \|uv\|_{H^s}\le C\|u\|_{H^s}\|v\|_{H^s}
    .\] 
\end{theorem}
\begin{remark}
    (Will not report in seminar) For proof please refer to Analysis123, Page 834.
\end{remark}

\begin{lemma}[Sobolev inequality for homogeneous norm]\hfill\par
    For \(u\in H^s(\mathbb{R}^n)\), define homogeneous Sobolev norm \[
        \|u\|_{\dot{H}^s}=\left(
        \int_{\mathbb{R}^n}|\xi|^{2s}|\hat{u}(\xi)|^2\dd{\xi}
        \right)^{\frac{1}{2}}
    .\] Suppose \(s<\frac{n}{2}\), \(\frac{1}{p^*}=\frac{1}{2}-\frac{s}{n}\),
    then \[
        \|u\|_{L^{p^*}}\le C\|u\|_{\dot{H}^s}
    .\] 
\end{lemma}
\begin{proof}
    We split \(u\) into lower and higher frequencies. Let \[
        \hat{u}_l=\hat{u}\cdot \Id_{|\xi|<R},
        \quad \hat{u}_h=\hat{u}\cdot \Id_{|\xi|\ge R}
    .\] Then by definition \(u_l=\mathcal{F}^{-1}\hat{u}_l\), we have
    \begin{align*}
        |u_l|
        &\le \int_{\mathbb{R}^n}|\hat{u}_l|\dd{\xi}
        \le \left(\int_{|\xi|<R}|\xi|^{-2s}\right)^{\frac{1}{2}}
        \left(\int_{\mathbb{R}^n}|\xi|^{2s}|\hat{u}_l|^2\right)^{\frac{1}{2}}\\ 
        &= C_1R^{\frac{n}{2}-s}\|u_l\|_{\dot{H}^s} \\
        &\le C_1R^{\frac{n}{2}-s}\|u\|_{\dot{H}^s}
    .\end{align*}
    Use the fact that \[
        (\|u\|_{L^{p^*}})^{p^*}=\int_{\mathbb{R}^n}|u|^{p^*}\dd{x}
        =p^*\int_{0}^{\infty}\lambda^{p^*-1}m(\{|u|>\lambda\})\dd{\lambda}
    .\] Notice \[
        m(\left\{|u|>\lambda\right\})\le m(\left\{|u_l|>\frac{\lambda}{2}\right\})
        +m(\left\{|u_h|>\frac{\lambda}{2}\right\})
    .\] Fix \(\lambda\), choose \(R\) \st\ \[
        C_1R^{\frac{n}{2}-s}\|u\|_{\dot{H}^s}=\frac{\lambda}{2}
    .\] Now \(\{|u_l|>\frac{\lambda}{2}\}=\emptyset\), we conclude \[
        m(\left\{|u|>\lambda\right\}) \le m(\left\{|u_h|>\frac{\lambda}{2}\right\})
    .\] By Markov's inequality, \[
        m(\{|u|>\lambda\})\le \frac{1}{\lambda^p}\|u\|_{L^p}^p
    .\] Then \begin{align*}
        \int_{\mathbb{R}^n}|u|^{p^*}\dd{x}
        &\le p^*\int_{0}^{\infty}\lambda^{p^*-1}\frac{1}{(\lambda/2)^2}
        \|u_h\|_{L^2}^2\dd{\lambda} \\
        &=4p^*\int_{0}^{\infty}\lambda^{p^*-3}\int_{|\xi|\ge R(\lambda)}
        |\hat{u}|^2\dd{\xi}\dd{\lambda} \\
        &=4p^*\int_{\mathbb{R}^n}|\hat{u}|^2\dd{\xi}\int_{0}^{2C_1|\xi|^{\frac{n}
        {2}-s}\|u\|_{\dot{H}^s}}\lambda^{p^*-3}\dd{\lambda} \\
        &=\cdots =C_2\|u\|_{\dot{H}^s}^{p^*}
    .\end{align*}
\end{proof}

\begin{theorem}[Sobolev embedding into \(L^p\)]\hfill\par
    Suppose \(s<\frac{n}{2},\forall\,p\in [2,p^*]\), we have \[
        \|u\|_{L^p}\le C\|u\|_{H^s}
    .\] \ie\ inclusion \(H^s\hookrightarrow L^p\) is continuous.
\end{theorem}
\begin{proof}
    By interpolation H\"older inequality, we have
    \begin{align*}
        \|u\|_{L^p}
        &\le \|u\|_{L^2}^{1-\theta}\|u\|_{L^{p^*}}^{\theta}
        \quad \text{where }\frac{1}{p}=\frac{1-\theta}{2}+\frac{\theta}{p^*} \\
        &\le C\color{cyan}{\|u\|_{L^2}^{1-\theta}\|u\|_{\dot{H}^s}^\theta} \\
        &\le C(\|u\|_{L^2}+\|u\|_{\dot{H}^s}) \\
        &= C\int_{\mathbb{R}^n}(1+|\xi|^{2s})|\hat{u}|^2\dd{\xi} \\
        &\le C\int_{\mathbb{R}^n}(1+|\xi|^2)^s\|\hat{u}\|^2\dd{\xi} \\
        &= C\|u\|_{H^s}
    .\end{align*}
\end{proof}

\begin{theorem}[Gagliardo-Nirenberg inequality]\label{thm:G-N}
    \hfill\par
    Let \(p^*=\frac{2n}{n-2},n\ge 2\), suppose \(2\le p<p^*=\infty\) for \(n=2\),
    \(2\le p\le p^*\) for \(n\ge 3\), then \[
        \|u\|_{L^p}\le C\|u\|_{L^2}^{1-\theta}\|\nabla u\|_{L^2}^\theta,
        \quad\text{ where }\theta=\frac{n(p-2)}{2p}.
    \]
\end{theorem}
\begin{proof}
    By looking carefully at proof of last theorem. Notice \(\|u\|_{\dot{H}^1}
    =\|\nabla u\|_{L^2}\)
\end{proof}

\begin{theorem}[Generalized Gagliardo-Nirenberg inequality]\label{thm:general-G-N}
    \hfill\par
    Let \(1\le p,q,r\le \infty\), \(l>k\ge 0\), \(\frac{k}{l}\le \theta\le 1\),
    satisfying \[
        \frac{1}{p}-\frac{k}{n}=\frac{1-\theta}{q}+\theta(\frac{1}{r}-\frac{l}{n})
    .\] Then \[
        \|\nabla^k u\|_{L^p}\le C\|u\|_{L^q}^{1-\theta}\|\nabla^l u\|_{L^r}^{\theta}
    ,\] where \(C\) only depends on \(p,q,r,k,l\), not on \(u\), with one exception:
    If \(r>1\) and \(l-k-\frac{n}{r}\) is a non-negative integer, then \(\theta<1\)
    is needed.
\end{theorem}
\begin{remark}
    Let \(k=0,l=1,q=2,r=2\) in~\cref{thm:general-G-N}, we get exactly former
    Gagliardo-Nirenberg inequality~\cref{thm:G-N}.
\end{remark}
We will NOT prove the generalized version here, please refer
to~\cite{nirenberg_elliptic_2011} and~\cite{fiorenza_detailed_2021} for detailed
proof and explanation.

\hfill

Now we turn to Sobolev spaces defined on a bounded region.
\begin{definition}[Sobolev space on bounded region]\hfill\par
    Suppose \(\Omega\subset \mathbb{R}^n\) is open (may not be bounded),
    we can consider all \(L^2\) function on \(\Omega\), this is a subspace of
    \(L^2(\mathbb{R}^n)\), we define \(H^s(\Omega)\) be \(L^2(\Omega)\cap 
    H^s(\mathbb{R}^n)\). Then define ``Sobolev space with zero boundary value''
    \(H_0^s(\Omega)\) be the closure of \(C_0^\infty(\Omega)\subset H^s(\Omega)\)
    under \(H^s\) norm.
\end{definition}
\begin{theorem}
    \(H^s(\Omega)\) and \(H_0^s(\Omega)\) are Hilbert spaces under above definition.
\end{theorem}

The mainly difference of \(H_0^1\) versus \(H^1\) is we can control \(L^2\) norm
of a function with information of its derivative.
\begin{theorem}[Poincaré inequality]\label{thm:poincare} \hfill\par
    Suppose \(\Omega\subset \mathbb{R}^n\) is bounded region, then there exists a
    constant \(C\) only depend on \(\Omega\) \st\ \(\forall\,u\in H_0^1(\Omega)\),
    we have \[
        \|u\|_{L^2}\le C\|\nabla u\|_{L^2}
    .\] 
\end{theorem}
\begin{proof}
    Fix a \(R>0\) \st\ \(\Omega\subset\{|x_n|\le R\}\). Since \(C_0^\infty(\Omega)\)
    is dense in \(H_0^1(\Omega)\), only need to prove for \(u\in C_0^\infty(\Omega)
    \). Denote \(x'=(x_1,\ldots,x_{n-1})\), we have
    \begin{align*}
        |u(x)|^2
        &=|u(x',x_n)|^2=\left|\int_{-R}^{x_n}\pdv{u}{x_n}\gap(x',t)
        \dd{t}\right|^2 \\
        &\le \left(\int_{-R}^{R}|\partial_n u(x',t)|\cdot 1\dd{t}\right)^2 \\
        &\le 2R\int_{-\infty}^{\infty}|\partial_n u(x',t)|^2\dd{t}
    .\end{align*}
    Notice RHS is not depend on \(x_n\), integrate w.r.t \(x_n\) from \(-R\) to
    \(R\), we get \[
        \int_{\mathbb{R}}|u(x',x_n)|^2\dd{x_n}\le 
        4\mathbb{R}^2 \int_{-\infty}^{\infty}|\partial_n u(x',t)|\dd{t}
    .\] Now integrate w.r.t \(x'\), we have \[
        \int_{\mathbb{R}}|u|^2\dd{x}\le 4R^2\int_{\mathbb{R}^{n-1}}
        \int_{-\infty}^{\infty}|\partial_n u(x',t)|^2\dd{t}\dd{x'}
        \le 4R^2 \int_{\mathbb{R}^n}|\nabla u|^2\dd{x}
    .\] 
\end{proof}

\begin{remark}
    From the proof we see inequality still holds if only assumed \(\Omega\) can be
    contained inside a pair of parallel hyperplanes.
\end{remark}

\begin{corollary}
    We can define a new inner product on \(H_0^1(\Omega)\): \[
        \left<u,v\right>_{H_0^1}=\left<\nabla u,\nabla v\right>_{L^2}
    .\] Then the norm induced is equivalent to original one, making \(H_0^1(\Omega)\)
    a Hilbert space.
\end{corollary}

Now we state an important theorem on Compactness of Sobolev embedding into 
\(L^p\) spaces.
\begin{theorem}[Rellich]\label{thm:cpt-Lp}\hfill\par
    Suppose \(\Omega\subset \mathbb{R}^n\) is a bounded region, \(n\ge 2\),
    then for any \(1\le p<p^*=\frac{2n}{n-2}\), the inclusion \[
        H_0^1(\Omega)\hooklongrightarrow L^p(\Omega)
        \quad\text{is compact.}
    \] 
\end{theorem}
\begin{proof}
    We first prove for case \(p=2\). Assume \(\overline{\Omega}\subset (0,2\pi)^n\),
    for general \(\Omega\) only need to do a linear transformation.

    Choose a compact
    \(K\) \st\ \[
        \Omega\subset \overline{\Omega}\subset K^\circ \subset K\subset (0,2\pi)^n
    .\] Let \(\chi,\eta\in C_0^\infty((0,2\pi)^n)\) such that \(\eval{\chi}_\Omega
    =1\) and \(\eval{\eta}_{\mathrm{supp(\chi)}}=1\). Now we prove \[
        T_{\chi}\colon H^1(\mathbb{R}^n)\longrightarrow L^2(\mathbb{R}^n),
        \ u\longmapsto \chi u
    \] is compact operator. Let \(u_\eta=\eta\cdot u\), then \(T_\chi(u)=T_\chi
    (u_{\eta})\). Consider Fourier series of \(u_{\eta}\), we have
    \begin{align*}
        T_{\chi}(u)(x)
        &=\chi(x)\sum_{k\in \mathbb{Z}^n}c_k e^{ik\cdot x}
        =\chi(x)\sum_{k\in \mathbb{Z}^n}\left(\frac{1}{(2\pi)^n}
        \int_{(0,2\pi)^n}u_\eta(y)e^{-ik\cdot y}\dd{y}\right)e^{ik\cdot x} \\
        &=\sum_{|k|\le N} c_k\chi(x)e^{ik\cdot x}
        +\chi(x)\sum_{|k|>N}c_k e^{ik\cdot x} \\
        &=:S_N(u)+R_N(u)
    .\end{align*}
    Obviously \(S_N\) has finite rank, hence is compact. We claim \(R_N\to 0\)
    as linear operators, then \(T_\chi\) is compact as limit of compact operators.

    Since \(\nabla u_\eta=\eta \nabla u+u \nabla\eta\), so \(\|\nabla u_\eta\|_{L^2}
    \le C\|u\|_{H^1}\). We then estimate \(R_N(u)\): 
    \begin{align*}
        \|R_N(u)\|_{L^2}^2
        &=\sum_{|k|>N}|c_k|^2\le \frac{1}{N^2}\sum_{|k|>N} |k|^2|c_k|^2 \\
        &\le \frac{1}{N^2}\|\nabla u_\eta\|_{L^2}^2
        \le \frac{C}{N^2}\|u\|_{H^1}^2
    .\end{align*}
    So \(\|R_N\|\le \frac{C}{N}\to 0\) as \(N\to \infty\).

    Then consider composition \[
        H_0^1(\Omega)\hooklongrightarrow H^1(\mathbb{R}^n)
        \xlongrightarrow{T_\chi}L^2(\Omega)\subset L^2(\mathbb{R}^n),
    \] it gives exactly inclusion \(H_0^1(\Omega)\hookrightarrow L^2(\Omega)\).
    Hence the inclusion is compact.

    Now for general \(p\), if \(1\le p<2\), the inclusion is compact since we can
    split it into \[
        H_0^1(\Omega)\hooklongrightarrow L^2(\Omega) \hooklongrightarrow L^p(\Omega)
    .\] If \(2<p<p^*\), for any bounded sequence \(\{u_k\}\subset H_0^1(\Omega)\),
    choose a \(L^2\) convergent subsequence, also denoted by \(\{u_k\}\), by
    Gagliardo-Nirenberg inequality we have \[
        \|u_k-u_l\|_{L^p}\le C\|u_k-u_l\|_{L^2}^{1-\theta}\|\nabla u_k
        -\nabla u_l\|_{L^2}^\theta
    .\] The first term is Cauchy and second is bounded, so \(\{u_k\}\) is Cauchy
    sequence in \(L_p(\Omega)\), and hence convergent.
\end{proof}

\begin{remark}
    Compactness of Sobolev embedding is very useful in finding solution of a PDE.
    We'll see how it works in next section. There are other embedding theorems
    of Sobolev spaces, one for general Sobolev spaces will be posted after this
    remark, and the other one for inclusion into \(L^\infty\) will be proved
    in~\cref{sec:morrey}.
\end{remark}

\begin{theorem}[Kondrachov]\hfill\par
    Suppose \(\Omega\subset \mathbb{R}^n\) is a bounded region,
    if \(k>l,\ k-\frac{n}{p}>l-\frac{n}{q}\), then embedding \[
        W^{k,p}(\Omega)\hooklongrightarrow W^{l,q}(\Omega)
        \quad\text{is compact.}
    \]
\end{theorem}

\section{Existence and Uniqueness}
Back to~\cref{prob:boundary-zero}, we reformulate it to
\begin{problem}\label{prob:h01}
    Suppose \(\Omega\subset \mathbb{R}^n\) is a bounded region, \(f\in L^2(\Omega)\),
    solve equation \[
        -\triangle u=f
    \] for \(u\in H_0^1(\Omega)\). Notice the boundary condition is translated into
    \(u\in H_0^1(\Omega)\).
\end{problem}

Define energy functional \[
    E(u)=\frac{1}{2}\int_{\Omega}|\nabla u|^2\dd{x}-\int_{\Omega}fu\dd{x}
.\] We claim that critical point of the energy functional must be solution
of original equation. To see this, suppose \(u\) is critical point of \(E\),
consider a variation \(u_t=u+t\vphi\), \(\vphi\in C_0^\infty(\Omega)\). Then
\begin{align*}
    \eval{\pd{t}}_{t=0}E(u+t\vphi)
    &=\lim_{t\to 0}\int_{\Omega}\frac{1}{t}
    \left(\frac{1}{2}(|\nabla u+t\nabla\vphi|^2-|\nabla u|^2)
    -(f\cdot (u+t\vphi)-fu)\right)\dd{x} \\
    &=\int_{\Omega}\left<\nabla u,\nabla\vphi\right>-f\vphi\dd{x} \\
    &=\int_{\Omega}\vphi\triangle u-f\vphi\dd{x}
.\end{align*}
Hence we have \[
    \left<-\triangle u-f,\vphi \right> =0,\quad
    \forall\,\vphi\in C_0^\infty(\Omega)
.\] \ie\ \(-\triangle u\xlongequal{\mathcal{D}'}f\).

Then we'd like to find a concrete solution, we use the method of minimizing
sequence. First notice \[
    E(u)\ge \frac{1}{2}\|\nabla u\|_{L^2}^2-\|u\|_{L^2}\|f\|_{L^2}
    \ge \frac{1}{2}\|\nabla u\|_{L^2}^2-C\|\nabla u\|_{L^2}\|f\|_{L^2},
\] which is a quadratic function in \(\|\nabla u\|_{L^2}\), with principal
term positive. Hence \(E\) has a lower bound, denote \(E_0=\inf_{u\in H_0^1
(\Omega)} E(u)\). So we can choose a sequence \(\{u_k\}\subset H_0^1(\Omega)\)
\st\ \[
    E(u_k)\to E_0.
\] \(\{u_k\}\) is bounded, otherwise \(\|\nabla u_k\|_{L^2}
\) unbounded implies \(E(u_k)\to \infty\). Recall \(H_0^1(\Omega)\) is Hilbert
space. So we can choose a weakly convergent subsequence, also denoted by
\(\{u_k\}\), \st\ \[
    u_k\xlongrightarrow{H_0^1\text{, w}}u\in H_0^1(\Omega)
.\] Since \(H_0^1(\Omega)\hookrightarrow L^2(\Omega)\) compactly, we know \[
    u_k\xlongrightarrow{L^2} u
.\] Passing to subsequence we can assume the convergent is of fast \(L^2\).
So we can also have \(u_k\to u\) a.e.

By property of weak convergence, we have \[
    \|u\|_{H_0^1}\le \liminf_{k\to \infty}\|u_k\|_{H_0^1}
.\] And \[
    \left<u_k,f\right> \to \left<u,f\right> 
    \quad\text{by strong convergence in }L^2
.\] Hence \[
    E(u)\le \liminf_{k\to \infty}E(u_k)=E_0
.\] This proves \(u\) is exactly a minimizer of energy, by variation argument
above we conclude that \(u\) is a solution to~\cref{prob:h01}.

For uniqueness, if there is two solutions \(u_1,u_2\in H_0^1(\Omega)\), then
\(u_1-u_2\) solves equation \(-\triangle u\equiv 0\). Multiply by \(u\) and integrate
by parts, we have \[
    \int_{\Omega}\left<\nabla u,\nabla u\right>\dd{x}
    =\int_{\Omega}|\nabla u|^2\dd{x} =0
.\] So \(\nabla u=0\) a.e. Which shows \(u\) is constant on each connect component
of \(\Omega\). But if constant \(c\in H_0^1(\Omega_1)\), apply Poincaré inequality,
we get \[
    c^2 m(\Omega_1)=\|c\|_{L^2}^2\le C\|\nabla c\|_{L^2}^2=0
.\] So only constant function in \(H_0^1(\Omega)\) is 0. Thus we proved uniqueness.

\section{Morrey's Lemma}\label{sec:morrey}
To obtain regularity of solution to~\cref{prob:h01}, we need following important
estimation by Morrey:
\begin{lemma}[Morrey]
    Let \(u\in C^\infty(B(x,R))\), then \[
        |u(x)-u(y)|\le CR^{1-\frac{n}{p}}\left(\int_{B(x,R)}|\nabla u|^p
        \right)^{\frac{1}{p}},
    \] where \(n<p<\infty\) and \(y\in B(x,\frac{R}{2})\).
\end{lemma}
\begin{proof}
    Let \(w\) be a unit vector, then \[
        |u(x+rw)-u(x)|\le \int_{0}^{r}|\nabla u(x+sw)|\dd s
    .\] Integrate w.r.t \(w\), 
    \begin{align*}
        \int_{|w|=1}|u(x+sw)-u(x)|\dd{\sigma}
        &\le \int_{0}^{r}\int_{\mathbb{S}^{n-1}}|\nabla u(x+sw)|\dd{w}\dd{t} \\
        &=\int_{0}^{r}\int_{\mathbb{S}^{n-1}}\frac{|\nabla u(x+sw)|}{s^{n-1}}
        \cdot s^{n-1}\dd{w}\dd{s} \\
        &=\int_{B(x,r)}\frac{|\nabla u|}{|x-y|^{n-1}}\dd{y}         
    .\end{align*}
    Multiply by \(r^{n-1}\) and integrate w.r.t \(r\) from 0 to \(R\), we get \[
        \int_{B(x,R)}|u(x)-u(y)|\dd{y}\le \frac{R^n}{n}
        \int_{B(x,R)}\frac{|\nabla u|}{|x-y|^{n-1}}\dd{y}
    .\] Choose \(p>n\), use H\"older's inequality, we have
    \begin{align*}
        \int_{B(x,R)}|u(x)-u(y)|\dd{y}
        &\le CR^n\|\nabla u\|_{L^p(B(x,R))}\left(
        \int_{B(x,R)}\frac{\dd{y}}{|x-y|^{(n-1)\frac{p}{p-1}}}
        \right)^{\frac{p-1}{p}} \\
        &\le CR^n\cdot R^{1-\frac{n}{p}}\|\nabla u\|_{L^p(B(x,R))}
    .\end{align*}
    Now assume \(|x-y|=\frac{R}{2}\), then \[
        |u(x)-u(y)|\le \frac{C}{R^n}\int_{B(x,\frac{R}{2})\cap B(y,\frac{R}{2})}
        |u(x)-u(z)|+|u(u)-u(z)|\dd{z}
    .\] Apply above estimate to two terms in RHS, we get
    \begin{align*}
        |u(x)-u(y)|&\le CR^{1-\frac{n}{p}}\|\nabla u\|_{L^p(B(x,R))} \\
        &\le C|x-y|^{1-\frac{n}{p}}\|\nabla u\|_{L^p}
    .\end{align*}
\end{proof}
\begin{remark}
    By last inequality, we can rewrite Morrey's lemma as \[
        \frac{|u(x)-u(y)|}{|x-y|^{1-\frac{n}{p}}}\le C\|\nabla u\|_{L^p},
    \] provided by \(u\in C^\infty(\mathbb{R}^n)\).
\end{remark}
As an application, we prove \(H^s(\Omega)\hookrightarrow L^\infty(\Omega)\) is
compact for \(s>\frac{n}{2}\) on bounded region \(\Omega\).

\begin{definition}[H\"older space]
    Let \(\Omega\) be a bounded region in \(\mathbb{R}^n\), recall for functions on
    \(\Omega\) of class \(C^k\), we have its \(C^k\)-norm defined by \[
        \|f\|_{C^k}=\sum_{0\le |\alpha|\le k}\|\partial^\alpha f\|_{L^\infty},
    \] and \(C^k(\overline{\Omega})\) become a Banach space under this norm, where
    the notation means all function of class \(C^k\) inside \(\Omega\) and all
    \(\le k\) order derivatives are continuous on \(\overline{\Omega}\).

    Now let \(0<\rho\le 1\), define \[
        \|f\|_{C^{k,\rho}}=\|f\|_{C^k}+\sup_{|\beta|=k}
        \frac{|\partial^\beta f(x)-\partial^\beta f(y)|}{|x-y|^{\rho}}
    .\] One can verify all \(f\in C^k(\overline{\Omega})\) \st\ 
    \(\|f\|_{C^{k,\rho}}<\infty\) form a Banach space under the norm, the space is
    denoted by \(C^{k,\rho}(\overline{\Omega})\).
\end{definition}
\begin{remark}\hfill
\begin{itemize}
    \item 
    Under this definition, the conclusion of Morrey's lemma can be translated as
    \[
        \|u\|_{C^{0,1-\frac{n}{p}}}\le C\|\nabla u\|_{L^p},
    \] for any \(u\in C^\infty(\mathbb{R}^n)\) and \(p>n\).
    \item 
    Suppose \(k+\rho>l+\eta\), one can easily prove \(C^{k,\rho}\hookrightarrow
    C^{l,\eta}\) is compact by Arzelà-Ascoli theorem.
\end{itemize}
\end{remark}

\begin{theorem}[Compactness of Sobolev embedding into \(L^\infty\)]\label{thm:cpt-Linf}
    \hfill\par
    Let \(\Omega\subset \mathbb{R}^n\) be a bounded region, \(s>\frac{n}{2}\), and
    \(s-\frac{n}{2}\) is not an integer, then inclusion \(H_0^s(\Omega)
    \hookrightarrow L^\infty(\Omega)\) is compact.
\end{theorem}
\begin{proof}
    For any \(u\in H_0^s(\Omega)\), extend \(u\) by 0 we get \(u\) defined on
    \(\mathbb{R}^n\), and in \(H^s(\mathbb{R}^n)\). Choose a compactly supported
    good integral kernel family \(\chi_\eps\), we have \(u_\eps=\chi_\eps*u\in 
    C_0^\infty(\mathbb{R}^n)\), and converges to \(u\) in \(H^s(\mathbb{R}^n)\).
    We have proved \(H^s(\mathbb{R}^n)\hookrightarrow C^0_\circ(\mathbb{R}^n)
    \hookrightarrow L^\infty(\mathbb{R}^n)\) is continuous, so the convergent
    is actually uniform.

    Apply Morrey's lemma, we have \[
        \|u_\eps\|_{C^{0,1-\frac{n}{p}}}\le C\|\nabla u_\eps\|_{L^p}
    .\] Let \(\eps\to 0^+\), we may take \(\eps\) in last inequality away.

    Now apply Gagliardo-Nirenberg inequality~\cref{thm:general-G-N}, let
    \(k=1,r=2,\theta=1\), we get \[
        \|\nabla u\|_{L^p}\le \|\nabla^{l+1}u\|_{L^2},
    \] with condition \(\frac{1}{p}=\frac{1}{2}-\frac{l}{n}\). Let \(l=s-1\) and
    rewrite the condition, we get \[
        \begin{cases}
            p=\frac{2n}{n-2l}\implies l<\frac{n}{2}; \\
            l=\frac{n}{2}-\frac{n}{p}\implies l>\frac{n}{2}-1.
        \end{cases}
    \] Put them together, we get \[
        \|u\|_{C^{0,1-\frac{n}{p}}}\le C\|\nabla^s u\|_{L^2}\le C\|u\|_{H^s},
    \]
    where \(\frac{n}{2}<s<\frac{n}{2}+1,\ p=\frac{2n}{n-2s+2}\).
    This gives continuous inclusion \(H_0^s(\Omega)\hookrightarrow
    C^{0,1-\frac{n}{p}}(\overline{\Omega})\).

    Notice for any \(0<\rho\le 1\), \(C^{0,\rho}(\overline{\Omega})\hookrightarrow
    L^{\infty}(\overline{\Omega})\) is compact, which
    immediately followed from Arzelà–Ascoli theorem. So we have \(H_0^s
    \hookrightarrow L^\infty(\overline{\Omega})\) is also compact. Since functions
    in \(H_0^s(\Omega)\) are continuous and take 0 value on boundary, the image
    of the inclusion is contained in \(L^\infty(\Omega)\), we get desired
    compactness.

    For case \(s>\frac{n}{2}+1\), write \(s=s_0+k\) for \(\frac{n}{2}<s_0<\frac{n}{2}
    +1\), \(k\) is non-negative integer. Just use the fact that \(H_0^s(\Omega)
    \hookrightarrow H_0^{s_0}(\Omega)\) continuously.
\end{proof}
\begin{remark}
    Write \(s=s_0+k\) as above, the conclusion can be improved to \(H_0^s(\Omega)
    \hookrightarrow C^k(\overline{\Omega})\) compactly. For an alternate proof, one
    can refer to Theorem 1.50 in~\cite{bahouri_fourier_2011}.
\end{remark}

By similar method, we can generalize Morrey's lemma to complete Riemannian manifolds.
\begin{theorem}
    
\end{theorem}

\section{Regularity of Solution}
In this section, we assume in~\cref{prob:h01}, \(f\in C^\infty(\overline{\Omega})\).
Our goal is to prove the solution \(u\in C^\infty(\Omega)\). Note we only prove for
regularity inside \(\Omega\), boundary regularity is further complicated and depends
on regularity of the boundary itself. There is multiple ways to obtain regularity.
First one is to apply Morrey's lemma:

Choose a ``good'' integration kernel \(\chi_\eps\in C_0^\infty\), \st\ \(u_\eps=
u*\chi_\eps\xrightarrow{H^1} u\). Note the convolution is in \(\mathbb{R}^n\),
the support of \(u_\eps\) may no longer \(\subset\Omega\). We have \(u_\eps \in 
C_0^\infty\), and \[
    \|\nabla^l u_\eps\|_{L^2}=\|\nabla^{l-2}(\nabla^2 u)*\chi_\eps\|_{L^2}
    =\|\nabla^{l-2}f_\eps\|_{L^2},\quad \text{where }f_\eps=f*\chi_\eps
.\] Notice for any \(\psi\in C_0^\infty(\mathbb{R}^n)\), \[
    \int_{\mathbb{R}^n}(\triangle \psi)^2=\sum_{i,j}\int_{\mathbb{R}^n}\psi_{ii}
    \psi_{jj}=-\sum_{i,j}\int_{\mathbb{R}^n}\psi_i\psi_{ijj}
    =\sum_{i,j}\int_{\mathbb{R}^n}\psi_{ij}\psi_{ij}
    =\int_{\mathbb{R}^n}|\mathrm{Hess}\,\psi|^2
.\] 

First take \(p=\infty,k=0,q=r\) in Gagliardo-Nirenberg inequality, and assume
\(mq>n\), then \[
    \|u\|_{L^\infty}\le C\|\nabla^l u\|_{L^q}^\theta \|u\|_{L^q}^{1-\theta}
.\] Where \(\theta=\frac{n}{mq}\). This proves \(u_\eps\) is uniformly bounded.

Then take \(\theta=1,r=2,k=1\), we have \[
    \|\nabla u\|_{L^p}\le \|\nabla^{l+1}u\|_{L^2},\quad \frac{1}{p}-\frac{1}{n}
    =\frac{1}{2}-\frac{l+1}{n}
.\] To make \(p>n\), let \(l>\frac{n}{2}-1\), and \(p>0\) gives
\(l<\frac{n}{2}\). By Morrey's lemma \[
    \frac{|u(x)-u(y)|}{|x-y|^{1-\frac{n}{p}}}\le CR^{1-\frac{n}{p}}\|\nabla u\|_{L^p}
\] for smooth \(u\).

Put these together, we have
\begin{align*}
    \frac{|u_\eps(x)-u_\eps(y)|}{|x-y|^{1-\frac{n}{p}}}
    &\le C\|\nabla u\|_{L^{p_0}},\quad \text{where }p_0=\frac{2n}{n-2l} \\
    &\le C\|\nabla^{l+1}u\|_{L_2} \\
    &=C\|\nabla^{l-1}f_\eps\|_{L^2}<\infty
.\end{align*}
Further \[
    \frac{|\nabla^k u_\eps(x)-\nabla^{k}u_\eps(y)|}{|x-y|^{1-\frac{n}{p}}}
    \le C\|\nabla^{l+k-1}f_\eps\|_{L_2}<\infty
.\] Hence \(\nabla^{k}u_\eps\) is equiv-continuous and uniformly bounded in \(\eps\).
By A-A thm, we have \[
    \nabla^k u_\eps\xlongrightarrow{\text{uni.}}\nabla^k u\in C^0
.\] Hence \(u\in C^\infty\) inside \(\Omega\).

\hfill

Another approach is to ``bootstrap'', write \(\tilde{u}=\chi u\), where \(\chi\)
is \(C_0^\infty\) on \(\Omega\) and is \(1\) around some point \(p\), a smooth
bump function.
We have equation for \(\tilde{u}\) become \[
    -\triangle \tilde{u}=\chi f-u\triangle\chi-2\left<\nabla\chi,\nabla u\right> 
.\] First \(u\in L^2, \nabla u\in L^2\implies \) RHS \(\in L^2\).
So \(|\xi|^2\hat{\tilde{u}}\in L^2\), which shows \(\tilde{u}\in H^2\). Then
RHS \(\in H^2\implies \tilde{u}\in H^3\), repeatedly apply the argument we have \[
    \tilde{u}\in \bigcap_{k\ge 0}H^k\subset C^{\infty}(\mathbb{R}^n)
.\] So \(u\) is smooth at \(p\), and hence at any \(p\in \Omega\).

\newpage
\printbibliography{}

\end{document}

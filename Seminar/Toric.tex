% !TeX program = xelatex
% \DocumentMetadata{pdfversion=1.7}
\documentclass[12pt]{article}
\usepackage[fourier,xy]{phyxmeow-common}
\geometry{a4paper,margin=1in}
\allowdisplaybreaks{}

\theoremstyle{plain}\newtheorem{theorem}{Theorem}
\theoremstyle{definition}\newtheorem{definition}[theorem]{Definition}
\theoremstyle{definition}\newtheorem{example}[theorem]{Example}
\theoremstyle{plain}\newtheorem{axiom}[theorem]{Axiom}
\theoremstyle{plain}\newtheorem{assertion}[theorem]{Assertion}
\theoremstyle{plain}\newtheorem{corollary}[theorem]{Corollary}
\theoremstyle{plain}\newtheorem{lemma}[theorem]{Lemma}
\theoremstyle{plain}\newtheorem{proposition}[theorem]{Proposition}
\theoremstyle{plain}\newtheorem{prop}[theorem]{Proposition}
\theoremstyle{plain}\newtheorem{conjecture}[theorem]{Conjecture}
\theoremstyle{plain}\newtheorem{conj}[theorem]{Conjecture}
\theoremstyle{plain}\newtheorem{problem}[theorem]{Problem}
\theoremstyle{remark}\newtheorem{notation}[theorem]{Notation}
\theoremstyle{definition}\newtheorem*{question}{Question}
\theoremstyle{definition}\newtheorem*{answer}{Answer}
\theoremstyle{definition}\newtheorem*{goal}{Goal}
\theoremstyle{plain}\newtheorem*{application}{Application}
\theoremstyle{plain}\newtheorem*{exercise}{Exercise}
\theoremstyle{remark}\newtheorem*{remark}{Remark}
\theoremstyle{remark}\newtheorem*{note}{\small{Note}}
\numberwithin{equation}{section}
\numberwithin{theorem}{section}
\numberwithin{figure}{section}

\addbibresource{Toric.bib}

\title{Introduction to Toric Manifolds}
\author{Xue Haotian}

\begin{document}
\maketitle

\section*{Abstract}
This article will cover the following aspects of Toric Geometry:
\begin{enumerate}[(1)]
\item How to construct a toric manifold from a Delzant polytope, inverse
    to Atiyah-Guillemin-Sternberg convexity theorem.
\item Construct the same manifold in a complex fashion, compatible to previous one,
    this makes the toric manifold a K\"ahler one.
\item Deriving an explicit formula of the K\"ahler metric from information of the
    polytope.
\item Calculation of Ricci and scalar curvature of the constructed metric.
\item Relation of blowing-up of manifold and cutting-off a corner of the polytope.
\end{enumerate}

\section{Preliminaries}
First recall the following basic definitions on Lie group action:
\begin{definition}
    Let \(M\) be a smooth manifold, \(G\) be a Lie group acting on \(M\) smoothly.
    The action is called
    \begin{itemize}
    \item \emph{Transitive}. If for any \(p,q\in M\), \(\exists\,g\in G\) such that
        \(q=g\cdot p\).
    \item \emph{Effective}. If for any \(g\neq 1\), \(\exists\,p\in M\) such that
        \(g\cdot p\neq p\). \ie\ Non-identity element must move some point.
    \item \emph{Free}. If for any \(g\neq 1,p\in M\), \(g\cdot p\neq p\).
        \ie\ Non-identity element must move \emph{every} point.
    \end{itemize}
\end{definition}

Now let \((M^{2n},\omega)\) be a compact symplectic manifold with a Lie group action
by a compact \(G\). Recall
\begin{definition}
    The action is called \emph{Hamiltonian} if there exists a moment map \[
        \Phi\colon M\longrightarrow \mathfrak{g}^*
    \] satisfying Hamilton equation \[
        \iota_{\xi_{\#}}\omega=-\dd \left<\Phi,\xi\right> 
    .\] Where \(\mathfrak{g}\) is Lie algebra of \(G\), \(\xi\in \mathfrak{g}\),
    and \(\xi_{\#}\) is the vector field on \(M\) generated by \(\xi\).
\end{definition}

From now on, all \(G\) action is considered as Hamiltonian, and \(G\) is some torus
\(\mathbb{T}^k=\mathbb{R}^k/\mathbb{Z}^k\).

\begin{lemma}
    Assume \(G\) acts effectively on \(M\), then \(\dim M\ge 2\dim G\).
\end{lemma}
\begin{proof}
    Let \(\mathcal{O}_p\) be a free orbit (of \(p\)) under action of \(G\). All such
    \(p\) is open dense provided by \(G\) Abelian and the action is proper. Then by
    the equivariance of moment map, we see \(\Phi\) is constant on \(\mathcal{O}_p\).
    Hence \(\eval{\dd{\Phi}}_{T\mathcal{O}_p}=0\). By Hamilton equation, we have \[
        \omega(X,Y)=-Y\left<\Phi,\xi\right> =0,\quad
        \text{for any }X=\xi_\#,Y=\eta_\#\in T\mathcal{O}_p
    .\] Hence \(\omega\) vanishes on a \(\dim G\) dimensional subspace.
    But \(\omega\) is non-degenerate, this forces \(\dim M\ge 2\dim G\).
\end{proof}

\begin{lemma}[Symplectic Quotient]\label{thm:symplectic-quotient}
    Let \(Z=\Phi^{-1}(0)\), if \(G\) acts on \(Z\) freely, then 0 is a regular
    value of \(\Phi\) and \(Z\) become an embedded submanifold. Moreover
    there is a natural symplectic structure \(\omega_B\) on quotient \(B=Z/G\) \st\ \[
        \pi^*\omega_B=\iota^*\omega
    .\] Where \(\iota\colon Z\to M\) is inclusion, \(\pi\colon Z\to B\) is quotient
    map.
\end{lemma}
\begin{proof}
    To prove 0 is regular value, we need to prove \(\dd{\Phi}\) has full rank
    everywhere on \(Z\). Note that \(\dd{\Phi}\) has full rank if and only if for
    any \(0\neq \xi\in \mathfrak{g}\), \(\left<\dd{\Phi},\xi\right> \neq 0\). But by
    Hamiltonian equation, this is just \[
        -\iota_{\xi_{\#}}\omega
    ,\] which is not zero functional since \(\omega\) is non-degenerate.

    Now we just define \[
        \omega_B(X,Y)=\omega(\tilde{X},\tilde{Y})
    .\] Where \(X,Y\in T_p B\), \(\tilde{X},\tilde{Y}\) are any lift of them in
    \(TZ\subset TM\). \(\omega_B\) is well-defined since \(\omega\) is 0 on orbit
    direction. Easy to verify it is non-degenerate and satisfies desired neutrality.
\end{proof}

\subsection{Blowing-up of K\"ahler Manifolds}
Identify \(\mathbb{CP}^{n-1}\) with lines in \(\mathbb{C}^n\). Let \(X=\{(L,v):
L\in \mathbb{CP}^{n-1},v\in L\}\). We have: \[
\xymatrix{
    \mathbb{CP}^{n-1} \ar[r]^\iota & X \ar[r]^\pi & \mathbb{C}^n
}\] Where \(\iota\) is zero section and \(\pi\) is projection onto fiber. The total
space \(X\) is called the \emph{blowing-up} of \(\mathbb{C}^n\) at origin.
Note that \(X\) can be viewed as deleting \(0\in \mathbb{C}^n\) then attach an
\(\mathbb{CP}^{n-1}\) at \(0\).

Let \(\omega_0\) on \(\mathbb{C}^n\) be the standard symplectic form \[
    \omega_0=\sum_{k=1}^{n}\dd{x_k}\wedge \dd{y_k}
    =\frac{\sqrt{-1}}{2}\sum_{k=1}^n \dd{z_k}\wedge \dd{\overline{z}_k}
.\] There is a natural \(U(n)\)-action on all three spaces and \(\omega_0\) is
invariant under the action.

\begin{definition}
    A symplectic form \(\omega\) on \(X\) is a \emph{blow-up} of \(\omega_0\) if
    it is \(U(n)\)-invariant and \(\omega-\pi^*\omega_0\) is compactly supported
    (in a neighbourhood of zero locus).
\end{definition}
Denote the set of all such forms by \(\mathcal{B}\).

\begin{definition}
    Two blow-up symplectic forms \(\omega_1,\omega_2\) are called \emph{equivalent}
    if there is a \(U(n)\)-equivariant diffeomorphism of \(X\) mapping \(\omega_1\) to
    \(\omega_2\).
\end{definition}

\begin{theorem}\label{thm:equiv-blow-up}
    \(\omega_1\) and \(\omega_2\) are equivariant \(\iff \iota^* \omega_1
    =\iota^* \omega_2\).
\end{theorem}

Now let \(\omega_{\text{FS}}\) be the standard Fubini-Study metric on
\(\mathbb{CP}^{n-1}\). Denote \(\mathcal{B}^{\eps}\) be the set of all \(\omega\in
\mathcal{B}\) \st\ \(\iota^* \omega=\eps \omega_{\text{FS}}\). Then by
\cref{thm:equiv-blow-up}, \(\omega_1\) and \(\omega_2\) are equivalent iff they lie
in the same \(\mathcal{B}^{\eps}\).

\begin{theorem}
    Given \(r>0\) there exists an \(\eps>0\) and \(\omega\in \mathcal{B}^{\eps}\) \st\ 
    \(\omega=\pi^*\omega_0\) out of ball \(|z|<r\). Further we have estimate
    \(\eps\sim \frac{r}{|\log r|}\).
\end{theorem}

\begin{prop}
    Given Hamiltonian \(G\)-action on standard \((\mathbb{C}^n,\omega_0)\), there is
    a way to choose coordinate \st\ the action is unitary, \ie\ \(G\) acts as a
    subgroup of \(U(n)\).
\end{prop}
Now Let \(M\) be a 2n dim symplectic manifold and \(p\in M\). By the Darboux theorem,
locally \[
    (M,p)\cong (\mathbb{C}^n,0)
.\] Further if \(M\) has a Hamiltonian \(G\)-action, the identification can be made
\(G\)-equivariant.

\subsection{Legendre Transform}
\begin{prop}
    Consider a smooth function \(f\colon \mathbb{R}\to \mathbb{R}\). Suppose
    \(f\) is strictly convex, \ie\ \(f''>0\), then the following are equivalent:
    \begin{enumerate}[(1)]
    \item \(f'(x_0)=0\) for some \(x_0\).
    \item \(f\) has local minimum at \(x_0\).
    \item \(f\) has unique local minimum.
    \item \(f(x)\to +\infty\) as \(|x|\to \infty\).
    \end{enumerate}
\end{prop}
We can generalize this proposition to \(n\)-dim.
\begin{theorem}\label{thm:stable-convex}
    Let \(F\colon \mathbb{R}^n\to \mathbb{R}\) be smooth, strictly convex, then the
    following are equivalent:
    \begin{enumerate}[(1)]
    \item \(\dd{F}=0\) at some \(x\).
    \item \(F\) has local minimum at \(x\).
    \item \(F\) has unique local minimum.
    \item \(F(x)\to +\infty\) as \(|x|\to \infty\).
    \end{enumerate}
\end{theorem}
The proof of these two facts are easy Analysis exercise.

Now denote \(V=\mathbb{R}^n\), identify \(T_x V\cong V\), \(T_x^* V\cong V^*\). Then \[
    \dd{F}\colon V\longrightarrow V^*\times V
.\] By projection onto first factor one gets a map, called \emph{Legendre transform}
of \(F\) \[
    L_F\colon V\longrightarrow V^*
.\] In coordinates, let \(x=x^i e_i\), \(\alpha=\alpha_i e_i^*\), then \[
    L_F(x)=\alpha\iff \alpha_i=\pdv{F}{x^i}
.\] The strict convexity translate into \(\mathrm{Hess}\,F>0\). In particular, this
shows that \(\dd{L_F}\) is non-singular everywhere. So \(L_F\) is local diffeomorphism.
\begin{definition}
    \(F\) is called \emph{stable} if it satisfy one of the four condition in
    \cref{thm:stable-convex}.
\end{definition}
For any \(l\in V^*\), \(F-l\) is also strictly convex with same Hessian.
\begin{definition}
    Define \(V_s^*\) be all linear functional \(l\in V^*\) \st\ \(F-l\) is stable,
    called the \emph{stability set} of \(F\).
\end{definition}
Notice that \(F-l\) is stable if and only if \(\dd(F-l)=0\) at some \(x\), \ie\ 
\(\eval{\dd{F}}_x=l\). Hence \(V_s^*\) is just image of \(L_F\).

\begin{theorem}
    \(V_s^*\) is open and convex, and \(L_F\) is diffeomorphism from \(V\) to
    \(V_s^*\).
\end{theorem}
\begin{proof}
    Easily followed from \cref{thm:stable-convex}
\end{proof}

\begin{theorem}
    If there exists a positive definite quadratic form \(\mathcal{Q}\) on \(V\),
    \(\eps>0\), \(C\) constant such that \[
        F(x)\ge \mathcal{Q}(x,x)^{\frac{1}{2}+\eps}+C
    .\] Then \(V_s^*=V^*\).
\end{theorem}
\begin{remark}
    Such \(F\) is said to have \emph{supra-linear growth}.
\end{remark}

Given \(F_1,F_2\) both strictly convex, easy to see \(F_1+F_2\) is strictly convex.
Let \(V_i^*\) be stability set of \(F_i\).
\begin{theorem}\label{thm:stable-set-sum}
    \(V_1^*+V_2^*\) is the stability set of \(F_1+F_2\).
\end{theorem}
\begin{proof}
    \(V_{1+2}^*\subset V_1+V_2\) is clear. Then for \(l_1\in V_1^*,l_2\in V_2^*\),
    \(F_1-l_1+F_2-l_2\) is stable by \cref{thm:stable-convex}, condition (4).
    This shows \(l_1+l_2\in V_{1+2}^*\).
\end{proof}

\begin{prop}\label{eg:legendre-1}
    Let \(l_1,\ldots,l_N\in V^*\), \(c_i\) be positive, \[
        F(x)=\sum_{i=1}^{N}c_i e^{l_i(x)}
    .\] Then \(\img L_F\) is the cone spanned by \(l_1,\ldots,l_N\).
\end{prop}
\begin{proof}
    First assume \(\{l_i\}\) span \(V^*\), otherwise let \(W=\Span\{l_1,\ldots,l_N\}\).
    Then consider \[
        \tilde{F}\colon V/W^\perp\to W
    .\] Further we assume \(N=n\), then with a change of basis we can assume \[
        F(x)=\sum_{i=1}^{N}c_i e^{x_i}
    .\] In this case \(\img L_F\) is clearly the positive orthant. General case
    followed by \cref{thm:stable-set-sum}.
\end{proof}
\begin{prop}\label{eg:legendre-2}
    For same \(l_i\), \(c_i\), let \[
        F(x)=\log\left(\sum_{i=1}^{N}c_i e^{l_i(x)}\right)
    .\] Then \(\img L_F\) is the convex closure (which is a polytope) of \(l_1,\ldots
    ,l_N\).
\end{prop}

Now we look into the concept of Legendre dual.
\begin{definition}
    Given \(F\colon:\mathbb{R}^n\to R\) convex and lower semi-continuous, \(F
    \not\equiv \infty\).
    The \emph{Legendre dual} of \(F\) is defined as \[
        F^*(l)=\sup_{x\in V}\{l(x)-F(x)\}
    .\] In particular, the dual is defined for strictly convex smooth \(F\).
\end{definition}

\begin{theorem}[Young's inequity]
    For any \(x\in \mathbb{R}^n,l\in V_s^*\), \[
        F(x)+F^*(l)\ge l(x)
    .\] 
\end{theorem}

\begin{theorem}
    \(F^{**}=F\), and \(L_{F^*}=L_F^{-1}\).
\end{theorem}

\begin{remark}
    The definition and theorems can be generalized to infinite dimensional
    normed linear space.
\end{remark}

As a example,
\begin{prop}
    Let \(F(x)=\frac{|x|^p}{p}\), \(p'\) \st\ \(\frac{1}{p}+\frac{1}{p'}=1\), then \[
        F^*(y)=\frac{|y|^{p'}}{p'}
    .\] 
\end{prop}

At last, we state a fact:
\begin{theorem}
    Let \(\Gamma\) be the graph of \(L_F\) in \(\mathbb{R}^{2n}\), equipped with 
    symplectic form \[
        \omega=\sum_{i=1}^n \dd{x_i}\wedge \dd{\alpha_i}
    .\] Then \(\Gamma\) is a Lagrangian submanifold, \ie\ \[
        \eval{\omega}_{\Gamma}=0
    .\] 
\end{theorem}

\section{Delzant Polytope and Symplectic Construction}

\subsection{The Construction}
\begin{definition}
    A polytope \(\Delta\subset (\mathbb{R}^n)^*\) is called \emph{Delzant} if it
    is convex and satisfy the following conditions:
    \begin{itemize}
    \item Each vertex \(p\) has exactly \(n\) edges.
    \item These edges can be written as \(p+tv_i\), \(v_i\in (\mathbb{Z}^n)^*\) are
        primitive integral vector. \ie\ the coordinate representation \[
            v_i^j\in \mathbb{Z},\text{ and }\gcd(v_i^1,\ldots,v_i^n)=1
        .\] 
    \item \(\{v_1,\ldots,v_n\}\) form a \(\mathbb{Z}\)-basis of \((\mathbb{Z}^n)^*\).
    \end{itemize}
\end{definition}

Given a Delzant polytope we can construct a symplectic manifold \(M\) with a
Hamiltonian \(\mathbb{T}^n\) action by the following steps:

\subsubsection*{Step 1. Description of the Polytope.}
For a convex polytope \(\Delta\), we can always write it as \[
    \Delta=\bigcap_{i=1}^N \{\left<\alpha,u_i\right> \ge \lambda_i\}
.\] Where \(\alpha\) is coordinate on \((\mathbb{R}^n)^*\), \(u_i\) are inward
normal vector of codim 1 faces. Further, we can choose all \(u_i\)'s to be integral and
primitive. Note that by Delzant condition, we have exactly \(n\) faces crossing
at each vertex, and corresponding \(u_i\)'s also form a \(\mathbb{Z}\)-basis.

\subsubsection*{Step 2. SES of Lie groups.}
Now let \(\omega_0\) be the standard symplectic form on \(\mathbb{C}^N\) \[
    \omega_0=\sum_{i=1}^N \dd{x^i}\wedge \dd{y^i}=\frac{\sqrt{-1}}{2}
    \sum_{i=1}^N\dd{z_i}\wedge \dd{\overline{z}_i}
.\] Let \(\mathbb{T}^N\) acts on \(\mathbb{C}^N\) by rotating each coordinate, \ie\ \[
    (\theta_1,\ldots,\theta_N)\cdot (z_1,\ldots,z_N)
    =(e^{\sqrt{-1}\theta_1}z_1,\ldots,e^{\sqrt{-1}\theta_N}z_N)
.\] Consider linear map \[
    \pi_*\colon \mathbb{R}^N \longrightarrow \mathbb{R}^n,\quad
    e_i \longmapsto u_i
,\] where \(\{e_1,\ldots,e_N\}\) is standard basis of \(\mathbb{R}^N\).
View \(\mathbb{R}^k\) as Lie algebra of \(\mathbb{T}^k\), since \(u_i\) are integral,
this induces a group quotient \[
    \pi\colon \mathbb{T}^N\longrightarrow \mathbb{T}^n
.\] Let \(K=\ker \pi\) and \(k=\mathrm{Lie}(K)\). We have short exact
sequence on both group level and Lie algebra level: \[
\xymatrix{
    0 \ar[r] & K \ar[r]^\iota & \mathbb{T}^N \ar[r]^\pi & \mathbb{T}^n \ar[r] & 0
}\] \[
\xymatrix{
    0 \ar[r] & k \ar[r]^{\iota_*} & \mathbb{R}^N \ar[r]^{\pi_*}
    & \mathbb{R}^n \ar[r] & 0
}\] And on dual Lie algebra: \[
\xymatrix{
    0 & \ar[l] k^* & \ar[l]_{\iota^*} (\mathbb{R}^N)^* & \ar[l]_{\pi^*}
    (\mathbb{R}^n)^* & \ar[l] 0
}\] 

\subsubsection*{Step 3. Obtain Moment Maps.}
Let \(\Phi_0\) be the moment map of \(\mathbb{T}^N\acts\mathbb{C}^N\), choose it
to be \[
    \Phi_0(z)=(\frac{1}{2}|z_1|^2+\lambda_1,\ldots,\frac{1}{2}|z_N|^2+\lambda_N)
.\] Then let \(\Phi_K\) be the moment map of \(K\acts \mathbb{C}^N\), such that \[
    \Phi_K=\iota^*\circ \Phi_0
.\] Define \(Z\) be the pre-image of 0 under \(\Phi_K\).
\begin{theorem}\label{thm:free-K-action}
    \(Z\) is a compact subset of \(\mathbb{C}^N\), with \(K\acts Z\) freely.
\end{theorem}
\begin{proof}
    First notice that the condition \(Z=\Phi_K^{-1}(0)\) is translated into \[
        \Phi_0(Z)\subset \ker\iota^*=\img \pi^*
    .\] Note that \(\pi^*\) is injective, let \(\alpha\) be a point in
    \((\mathbb{R}^n)^*\), we have \[
        \pi^*(\alpha)=(e_i\longmapsto \left<\alpha,u_i\right>)=\left<\alpha,u_i\right> 
        e^{i}
    .\] On the other hand, \[
        \Phi_0(z)=(\frac{1}{2}|z_i|^2+\lambda_i)e^{i}
    .\] So we have \[
        \left<\alpha,u_i\right> =\frac{1}{2}|z_i|^2+\lambda_i\ge \lambda_i
    .\] \ie\ \(\alpha\in \Delta\). Conversely, for any \(\alpha\in \Delta\),
    we can find suitable \(|z_i|\) \st\ \[
        \frac{1}{2}|z_i|^2=\left<\alpha,u_i\right>-\lambda_i
    .\] Hence \(Z=\Phi_0^{-1}(\pi^*(\Delta))\). Since \(\Phi_0\) is a proper map,
    and \(\pi^*\) is continuous, \(Z\) is compact. Let \[
        \Psi=(\pi^*)^{-1}\circ \Phi_0\colon Z\longrightarrow (\mathbb{R}^n)^*
    .\] Now we prove \(K\acts Z\) freely. For \(\alpha\in \Delta^\circ \),
    \(\Psi^{-1}(\alpha)\) consists of points \(z\) with no \(z_i=0\). In this case,
    whole \(\mathbb{T}^N\) acts on \(\Psi^{-1}(\alpha)\) freely. If \(\alpha\) lie
    on the boundary of \(\Delta\), there will be several \(i\) \st\ \(z_i=0\).
    The worst case is when \(\alpha\) is a vertex of \(\Delta\), where \(z_i=0\iff 
    u_i\) represents a face \(F_i\) through \(\alpha\). In this case, stabilizer of
    \(z\) is \[
        (\theta_1,\ldots,\theta_N),\ \theta_i \neq 0 \iff i\in I
    ,\] where \(I=\{i:\alpha\in F_i\}\). Under \(\pi\), this stabilizer is mapped to \[
        \sum_{i\in I}\theta_i\cdot u_i,\quad\theta_i\in [0,2\pi]
    .\] But \(u_i\) form a \(\mathbb{Z}\)-basis. So \(\pi\) is bijective on the
    stabilizer. Hence \(K\) intersects the stabilizer transversely, \ie\ \(K\) acts
    freely.
\end{proof}

\subsubsection*{Step 4. Take Symplectic Quotient.}
By \cref{thm:symplectic-quotient} and \cref{thm:free-K-action}, \(Z\) is an embedded
submanifold and we have a natural induced symplectic structure \(\omega\) on \[
    M_\Delta=Z/K
.\] The dimension of \(M_\Delta\) is \(2N-(N-n)-(N-n)=2n\).
Denote by \(\Phi\) its moment map under the induced action of \[
    \mathbb{T}^N/K\cong \mathbb{T}^n
.\] By naturality of moment map, we have \[
    \Phi\circ \pi=\Psi\colon Z\longrightarrow (\mathbb{R}^n)^*
.\] Hence \(\img \Phi\) is exactly \(\Psi(Z)=\Delta\).

Recall we already know
\begin{theorem}[Atiyah, Guillemin-Sternberg]
    Let \(\mathbb{T}^k\acts (M^{2n},\omega)\) be an effective Hamiltonian action.
    Let \(\Phi\) be its moment map, then
    \begin{enumerate}[(1)]
    \item The level sets of \(\Phi\) are connected.
    \item The image of \(\Phi\) is convex. Moreover, it is the convex hull
        of image of fixed points under \(\mathbb{T}^k\) action.
    \end{enumerate}
\end{theorem}
In particular, if \(k=n\), we call \(M\) a symplectic toric manifold. Delzant has
shown that the image of the moment map must be a Delzant polytope in this case.
And we have shown that symplectic toric manifold has an one-one correspondence
to Delzant polytope.

\subsection{Examples}
\begin{example}\label{eg:cp1}
    Let \(\Delta\) be
    \begin{center}
    \begin{tikzpicture}[>=latex]
        \draw (0,0) -- (2,0);
        \node [below] at (0,0) {\(0\)};
        \node [below] at (2,0) {\(1\)};
    \end{tikzpicture}
    \end{center}
    Then \[
        u_1=1,\ u_2=-1
    .\] \[
        \lambda_1=0,\ \lambda_2=-1
    .\] We have SES of Lie algebra: \[
    \xymatrix@R-2pc{
        0 \ar[r] & k \ar[r]^{\iota_*} & \mathbb{R}^2 \ar[r]^{\pi_*} & \mathbb{R}
        \ar[r] & 0 \\
        & & (\theta_1,\theta_2) \ar@{|->}[r] & \theta_1-\theta_2 \\
        & \theta \ar@{|->}[r] & (\theta,\theta)
    }\] And for dual Lie algebra: \[
    \xymatrix@R-2pc{
        0 & \ar[l] k^* & \ar[l]_{\iota^*} (\mathbb{R}^2)^* & \ar[l]_{\pi^*}
        \mathbb{R}^* & \ar[l] 0 \\
        & & (\vphi,-\vphi) & \ar@{|->}[l] \vphi \\
        & \vphi_1+\vphi_2 & \ar@{|->}[l] (\vphi_1,\vphi_2)
    }\] Then \[
        \Phi_K(z)=\iota^*\circ \Phi_0(z)=\frac{|z_1|^2}{2}+\frac{|z_2|^2}{2}-1
    .\] So \(Z=\{|z_1|^2+|z_2|^2=2\}\) and \(K\) action is \[
    \theta\cdot (z_1,z_2)=(e^{\sqrt{-1}\theta}z_1,e^{\sqrt{-1}\theta}z_2).
    \] Which gives exactly the Hopf fibration, hence \(M\cong \mathbb{CP}^1\).
\end{example}
\begin{example}\label{eg:cp1xcp1}
    Let \(\Delta\) be
    \begin{center}
    \begin{tikzpicture}[>=latex]
        \draw [->] (-0.5,0) -- (3,0);
        \draw [->] (0,-0.5) -- (0,2.5);
        \draw (2,0) -- (2,2) -- (0,2);
        \node [below left] at (0,0) {\(0\)};
        \node [below] at (2,0) {\(1\)};
        \node [left] at (0,2) {\(1\)};
    \end{tikzpicture}
    \end{center}
    Then \[
        u_1=(1,0),\ u_2=(0,1),\ u_3=(-1,0),\ u_4=(0,-1)
    .\] \[
        \lambda_1=0,\ \lambda_2=0,\ \lambda_3=-1,\ \lambda_4=-1
    .\] We have SES of Lie algebra: \[
    \xymatrix@R-2pc{
        0 \ar[r] & k \ar[r]^{\iota_*} & \mathbb{R}^4 \ar[r]^{\pi_*} & \mathbb{R}^2
        \ar[r] & 0 \\
        & & (\theta_1,\theta_2,\theta_3,\theta_4) \ar@{|->}[r]
        & (\theta_1-\theta_3,\theta_2-\theta_4) \\
        & (\theta_1,\theta_2) \ar@{|->}[r] & (\theta_1,\theta_2,\theta_1,\theta_2)
    }\] And for dual Lie algebra: \[
    \xymatrix@R-2pc{
        0 & \ar[l] k^* & \ar[l]_{\iota^*} (\mathbb{R}^4)^* & \ar[l]_{\pi^*}
        (\mathbb{R}^2)^* & \ar[l] 0 \\
        & & (\vphi_1,\vphi_2,-\vphi_1,-\vphi_2) & \ar@{|->}[l] (\vphi_1,\vphi_2) \\
        & (\vphi_1+\vphi_3,\vphi_2+\vphi_4) & \ar@{|->}[l]
        (\vphi_1,\vphi_2,\vphi_3,\vphi_4)
    }\] Then \[
        \Phi_K(z)=\iota^*\circ \Phi_0(z)=(\frac{|z_1|^2}{2}+\frac{|z_3|^2}{2}-1,
        \frac{|z_2|^2}{2}+\frac{|z_4|^2}{2}-1)
    .\] So \(Z=\{|z_1|^2+|z_3|^2=2\}\cap \{|z_2|^2+|z_4|^2=2\}\) and \(K\) action
    is \[
    (\theta_1,\theta_2)\cdot (z_1,z_2,z_3,z_4)=(e^{\sqrt{-1}\theta_1}z_1,
    e^{\sqrt{-1}\theta_2}z_2, e^{\sqrt{-1}\theta_1}z_3,e^{\sqrt{-1}\theta_2}z_4).
    \] Hence \(M\cong \mathbb{CP}^1\times \mathbb{CP}^1\).
\end{example}
\begin{example}\label{eg:cp2}
    Let \(\Delta\) be
    \begin{center}
    \begin{tikzpicture}[>=latex]
        \draw [->] (-0.5,0) -- (3,0);
        \draw [->] (0,-0.5) -- (0,2.5);
        \draw (2,0) -- (0,2);
        \node [below left] at (0,0) {\(0\)};
        \node [below] at (2,0) {\(1\)};
        \node [left] at (0,2) {\(1\)};
    \end{tikzpicture}
    \end{center}
    Then \[
        u_1=(1,0),\ u_2=(0,1),\ u_3=(-1,-1)
    .\] \[
        \lambda_1=0,\ \lambda_2=0,\ \lambda_3=-1
    .\] We have SES of Lie algebra: \[
    \xymatrix@R-2pc{
        0 \ar[r] & k \ar[r]^{\iota_*} & \mathbb{R}^3 \ar[r]^{\pi_*} & \mathbb{R}^2
        \ar[r] & 0 \\
        & & (\theta_1,\theta_2,\theta_3) \ar@{|->}[r]
        & (\theta_1-\theta_3,\theta_2-\theta_3) \\
        & \theta \ar@{|->}[r] & (\theta,\theta,\theta)
    }\] And for dual Lie algebra: \[
    \xymatrix@R-2pc{
        0 & \ar[l] k^* & \ar[l]_{\iota^*} (\mathbb{R}^3)^* & \ar[l]_{\pi^*}
        (\mathbb{R}^2)^* & \ar[l] 0 \\
        & & (\vphi_1,\vphi_2,-\vphi_1-\vphi_2) & \ar@{|->}[l] (\vphi_1,\vphi_2) \\
        & \vphi_1+\vphi_2+\vphi_3 & \ar@{|->}[l] (\vphi_1,\vphi_2,\vphi_3)
    }\] Then \[
        \Phi_K(z)=\iota^*\circ \Phi_0(z)=\frac{|z_1|^2}{2}+\frac{|z_2|^2}{2}
        +\frac{|z_3|^2}{2}-1
    .\] So \(Z=\{|z_1|^2+|z_2|^2+|z_3|^2=2\}\) and \(K\) action is \[
    \theta\cdot (z_1,z_2,z_3)=(e^{\sqrt{-1}\theta}z_1,e^{\sqrt{-1}\theta}z_2,
    e^{\sqrt{-1}\theta}z_3).
    \] Hence \(M\cong \mathbb{CP}^2\).
\end{example}
\begin{example}[Blow up a point]\label{eg:blow-up}
    Let \(\Delta\) be:

    \begin{center}
    \begin{tikzpicture}[>=latex]
        \draw [->] (-0.5,0) -- (3,0);
        \draw [->] (0,-0.5) -- (0,2.5);
        \draw (2,0) -- (1,1) -- (0,1);
        \draw [dashed] (1,1) -- (0,2);
        \node [below left] at (0,0) {\(0\)};
        \node [below] at (2,0) {\(1\)};
        \node [left] at (0,1) {\(\frac{1}{2}\)};
        \node [left] at (0,2) {\(1\)};
    \end{tikzpicture}
    \end{center}
    Then \[
        u_1=(1,0),\ u_2=(0,1),\ u_3=(-1,-1),\ u_4=(0,-1)
    .\] \[
        \lambda_1=0,\ \lambda_2=0,\ \lambda_3=-1,\ \lambda_4=-\frac{1}{2}
    .\] We have SES of Lie algebra: \[
    \xymatrix@R-2pc{
        0 \ar[r] & k \ar[r]^{\iota_*} & \mathbb{R}^4 \ar[r]^{\pi_*} & \mathbb{R}^2
        \ar[r] & 0 \\
        & & (\theta_1,\theta_2,\theta_3,\theta_4) \ar@{|->}[r]
        & (\theta_1-\theta_3,\theta_2-\theta_3-\theta_4) \\
        & (\theta_1,\theta_4) \ar@{|->}[r] & (\theta_1,\theta_1+\theta_4,
        \theta_1,\theta_4)
    }\] And for dual Lie algebra: \[
    \xymatrix@R-2pc{
        0 & \ar[l] k^* & \ar[l]_{\iota^*} (\mathbb{R}^4)^* & \ar[l]_{\pi^*}
        (\mathbb{R}^2)^* & \ar[l] 0 \\
        & & (\vphi_1,\vphi_2,-\vphi_1-\vphi_2,-\vphi_2) & \ar@{|->}[l]
        (\vphi_1,\vphi_2) \\
        & (\vphi_1+\vphi_2+\vphi_3,\vphi_2+\vphi_4) & \ar@{|->}[l]
        (\vphi_1,\vphi_2,\vphi_3,\vphi_4)
    }\] Then \[
        \Phi_K(z)=\iota^*\circ \Phi_0(z)=(\frac{|z_1|^2}{2}+\frac{|z_2|^2}{2}
        +\frac{|z_3|^2}{2}-1,
        \frac{|z_2|^2}{2}+\frac{|z_4|^2}{2}-\frac{1}{2})
    .\] So \[
        Z=\{|z_1|^2+|z_2|^2+|z_3|^2=2\}\cap \{|z_2|^2+|z_4|^2=1\}
    .\] And \(K\)-action is \[
        (\theta_1,\theta_4)\mapsto (\theta_1,\theta_1+\theta_4,\theta_1,\theta_4)
    .\] We claim this is \cref{eg:cp2} blow up at \((0,1,0)\), which is diffeomorphic
    to \(\mathbb{CP}^2\,\#\,\overline{\mathbb{CP}^2}\).
\end{example}

See \cref{sec:blowing-up} for detailed proof.

% Note for me:
% \Phi_0 : T^N act on C^n
% \Phi_K : K act on C^n
% \Phi : T^n=T^N/K act on M_\Delta=Z/K
% Z=\Phi_K^{-1}(0)=\Phi_0^{-1}(\pi^*(\Delta))
% \omega_0 : std metric on C^n
% \omega : induced metric on M

\section{Complex Quotient}
\subsection{The Construction}
Recall the SES we got in symplectic construction: \[
\xymatrix{
    0 \ar[r] & k \ar[r]^{\iota_*} & \mathbb{R}^N \ar[r]^{\pi_*} & \mathbb{R}^n \ar[r] & 0
}\] Complexify, we get \[
\xymatrix{
    0 \ar[r] & k_{\mathbb{C}} \ar[r]^{\iota_*} & \mathbb{C}^N \ar[r]^{\pi_*} &
    \mathbb{C}^n \ar[r] & 0
}\] 
Where \(\mathbb{R}^N\) embedded in \(\mathbb{C}^N\) as imaginary part.

Quotient by \(2\pi\sqrt{-1}\mathbb{Z}\), we get SES of Lie groups: \[
\xymatrix{
    0 \ar[r] & K_{\mathbb{C}} \ar[r]^{\iota} & \mathbb{T}_{\mathbb{C}}^N
    \ar[r]^{\pi} & \mathbb{T}_{\mathbb{C}}^n \ar[r] & 0
}\] Where \(\mathbb{T}_{\mathbb{C}}=\mathbb{C}/2\pi\sqrt{-1}\mathbb{Z}\).

The action of \(\mathbb{T}^N\) on \(\mathbb{C}^N\) extends to \[
    \mathbb{T}_{\mathbb{C}}^N\acts \mathbb{C}^N:
    (w_1,\ldots,w_N)\cdot z=(e^{w_1}z_1,\ldots,e^{w_N}z_N)
.\] Where \(w_i=\log r_i +\sqrt{-1}\theta_i\).

Let \(F\) be a face of \(\Delta\) \st\ \[
    \left<x,u_i\right> =0,\text{ for }i\in I\subset \{1,\ldots,N\}
.\] Let  \[
    O_F=\{z\in \mathbb{C}^N:z_i=0\iff i\in I\}
.\] And \[
    O_{\Delta}=\bigcup_{F}O_F
.\] 
\begin{theorem}
    \(O_{\Delta}\) is an open subset of \(\mathbb{C}^N\). Moreover, \(K_{\mathbb{C}}\)
    acts freely on \(O_\Delta\), and the quotient \[
        X_{\Delta}=O_\Delta/K_{\mathbb{C}}
    \] is a compact complex manifold.
\end{theorem}
\begin{proof}
    Left to reader.
\end{proof}
Note \(\mathbb{T}_{\mathbb{C}}^n\) orbits in \(X_{\Delta}\) is one-to-one correspond
to faces of \(\Delta\).

Let \(\sigma\colon \mathbb{C}^N\to \mathbb{C}^N\) be complex conjugation, we have \[
    w\cdot \sigma(z)=\sigma(\overline{w}\cdot z)
.\] So \(\sigma\) maps \(K_{\mathbb{C}}\) orbit of \(z\) to orbit of \(\sigma(z)\).
Hence it induces an involution \[
    \sigma\colon X_\Delta\to X_\Delta
.\] Denote \(\Re(X_{\Delta})\) be the fixed points of \(\sigma\), called the \emph{real
part} of \(X_\Delta\).

Compare this to construction in last section, we got \(Z\subset \mathbb{C}^N\) compact
with \(K\)-action before. It's easy to see \(Z\subset O\), and the \(K\)-action
is related to \(K_{\mathbb{C}}\)-action by embedding as imaginary part.

More precisely, we write
\begin{lemma}
    Map \[
        \Psi=(\pi^*)^{-1}\circ \Phi_0\colon Z\to \Delta
    \] is surjective, and \(\Psi(z)\in F \iff z\in O_F\cap Z\).
\end{lemma}

\begin{theorem}\label{thm:orbit-intersect}
    Every \(K_{\mathbb{C}}\) orbit of \(O\) intersects \(Z\), and there intersection
    is exactly an \(K\) orbit.
\end{theorem}
As a corollary, the symplectic quotient and complex quotient give the same result,
\ie\ \(M_\Delta=X_\Delta\). And \(M_\Delta\) has a natural complex structure that
compatible with symplectic structure \(\omega\), hence become a K\"ahler manifold.
We do not distinguish \(M_\Delta\) and \(X_\Delta\) from now on, and use \(M\) and
\(X\) to denote them if there is no ambiguity.

Before proving this theorem, let's look at some examples compare to symplectic case.
\subsection{Examples}

\begin{example}[\(\mathbb{CP}^2\)]
    Let \(\Delta\) be 

    \begin{center}
    \begin{tikzpicture}[>=latex]
        \draw [->] (-0.5,0) -- (3,0);
        \draw [->] (0,-0.5) -- (0,2.5);
        \draw (2,0) -- (1,1) -- (0,2);
        \node [below left] at (0,0) {\(0\)};
        \node [below] at (2,0) {\(1\)};
        \node [left] at (0,2) {\(1\)};
    \end{tikzpicture}
    \end{center}
    Set 
    \begin{align*}
        u_1&= (1,0), &\lambda_1&=0 \\
        u_2&=(0,1), &\lambda_2&=0 \\
        u_3&=(-1,-1), &\lambda_3&=-1
    .\end{align*}
    We have \[
        \pi(\theta_1,\theta_2,\theta_3)=(\theta_1-\theta_3,\theta_2-\theta_3)
    .\] Complexify we get \[
    \xymatrix@R-2pc{
        0 \ar[r] & k_{\mathbb{C}} \ar[r]^{\iota_*} & \mathbb{T}_{\mathbb{C}}^N
        \ar[r]^{\pi_*} & \mathbb{T}_{\mathbb{C}}^n \ar[r] & 0 \\
        & & (w_1,w_2,w_3) \ar@{|->}[r] & (w_1-w_3,w_2-w_3) \\
        & w \ar@{|->}[r] & (w,w,w)
    }\] And on dual Lie algebra \[
    \xymatrix@R-2pc{
        0 & \ar[l] k_{\mathbb{C}}^* & \ar[l]_{\iota^*} (\mathbb{T}_{\mathbb{C}}^N)^* &
        \ar[l]_{\pi^*} (\mathbb{T}_{\mathbb{C}}^n)^* & \ar[l] 0 \\
        & & (\xi_1+\xi_2,-\xi_2,-\xi_1) & \ar@{|->}[l] (\xi_1,\xi_2) \\
        & \xi_1+\xi_2+\xi_3 & \ar@{|->}[l] (\xi_1,\xi_2,\xi_3)
    }\] by Then
    \begin{itemize}
    \item \(F_0=\{x:\left<x,u_i\right> \alpha_i\}\) and
        \(O_0=\{z_1\neq 0,z_2\neq 0,z_3\neq 0\}\).
    \item \(F_1=\bigcup_{i=1}^3\{x:\left<x,u_i\right> =0\}\) and 
        \(O_1=\{(0,z_2,z_3)\}\cup \{z_1,0,z_3\}\cup \{z_1,z_2,0\}\).
    \item \(F_2=\{(0,0),(1,0),(0,1)\}\) and
        \(O_2=\{z_1,0,0\}\cup \{0,z_2,0\}\cup \{0,0,z_3\}\)
    \end{itemize}
    Hence \(O=O_0\cup O_1\cup O_3=\mathbb{C}^3\setminus\{(0,0,0)\}\), and \[
        X_{\Delta}=O/K_{\mathbb{C}}=\mathbb{C}^3\setminus\{0\}/\mathbb{C}^*
        \cong \mathbb{CP}^2
    .\] 
\end{example}

\subsection{Proof of Theorem}
Let \(A\) be the real part of \(K_{\mathbb{C}}\), we have Iwasawa decomposition of
\(K_{\mathbb{C}}\): \[
    K_{\mathbb{C}}=KA
.\] Where \(K\) is a real torus and \(A\) is a vector group. The theorem is
equivalent to the following:
\begin{assertion}\label{ass:A-orbit-intersect}
    Every \(A\) orbit in \(O_{\Delta}\) intersects \(Z\) in exactly one point.
\end{assertion}
Note that the moment map of \(K\acts \mathbb{C}^N\) is \[
    \Phi_K\colon z\longmapsto \frac{1}{2}\sum_{i=1}^N |z_i|^2\beta^i+\lambda'
.\] Where \(\beta^i=\iota^* e^i\in k^*\), \(\{e^i\}\) is standard basis of
\((\mathbb{R}^n)^*\), \(\lambda'=\lambda_i \beta^i\). By definition, \(Z\) is the
zero level-set of \(\Phi_K\), so \cref{ass:A-orbit-intersect} is implied by the
following two assertions:
\begin{lemma}\label{lem:A-orbit-diffeo}
    Let \(\mathcal{R}\) be an \(A\)-orbit in \(O_\Delta\), then \(\Phi_K\) maps
    \(\mathcal{R}\) diffeomorphically to an open convex set in \(k^*\).
\end{lemma}
\begin{lemma}\label{lem:A-orbit-face}
    If two \(A\)-orbits lie in the same \(O_F\), their images under \(\Phi_K\)
    are identical.
\end{lemma}
We will prove \cref{lem:A-orbit-diffeo,lem:A-orbit-face} by deriving an explicit
formula for \(\Phi_K\) as a Legendre transform.

First, notice that \(\beta^i\)'s are just the weights of the representation of 
\(K_{\mathbb{C}}\) on \(\mathbb{C}^N\). Then for \(a\in A,z\in \mathbb{C}^N\), \[
    a\cdot z=(e^{\beta^1(a)}z_1,\ldots,e^{\beta^N(a)}z_N)
.\] Hence \[
    \Phi_K(a\cdot z)=\frac{1}{2}\sum_{i=1}^N|z_i|^2e^{2\beta^i(a)}\beta^i+\lambda'
.\] Suppose \(z\in O_\Delta\), then \(A\) acts freely at \(z\). We can identify
the \(A\)-orbit through \(z\) with \(A\). Let \(s_i=\frac{1}{2}|z_i|^2\), suppose
\(z\in O_F\), we got \[
    f\colon A\to k^*,\quad
    a\longmapsto \sum_{i\in I_F^c} s_i e^{2\beta^i(a)}\beta^i+\lambda'
.\] Now notice that \(f\) is the Legendre transform of \[
    h(a)=\frac{1}{2}\sum_{i\in I_F^c}s_i e^{2\beta^i(a)}+\lambda'(a)
.\] Easy to see \(h\) is strictly convex. 
\begin{lemma}
    \(\{\beta^i:i\in I_F^c\}\) form a spanning set of \(k^*\).
\end{lemma}
\begin{proof}
    If there is \(v\in k\) \st\ \[
        \iota^*(e^i)(v)=0,\quad\forall\,i\in I_F^c
    .\] Hence \[
        \iota_*(v)\in \Span\{e_i:i\in I_F\}
    .\] Since \(\pi_*\colon e_i\mapsto u_i\) and \(\{u_i:i\in I_F\}\) form a basis,
    we must have \(\iota_*(v)=0\) since \(\pi_*\circ \iota_*=0\). Then by \(\iota_*\)
    injective, \(v=0\).
\end{proof}
Then by \cref{eg:legendre-1}, the image of \(h\) is open convex cone \[
    \Big\{\sum_{i\in I_F^c}t_i\beta^i+\lambda':t_i>0\Big\}
.\] In particular, it only depends on \(I_F\).

\section{The K\"ahler Structure}

\subsection{The Main Result}
From above discussion, we have obtained a K\"ahler manifold \((M_\Delta,\omega)\)
with a Hamiltonian action by \(\mathbb{T}^n\) from a given Delzant polytope \(\Delta\).
The goal of this section is to characterize the K\"ahler form \(\omega\) using
information of \(\Delta\).

Again we write \(\Delta\) as \[
    \bigcap_{i=1}^N\{\alpha\in (\mathbb{R}^n)^*:\left<\alpha,u_i\right> \ge\lambda_i\}
.\] Denote \(l_i\colon \mathbb{R}^n\to \mathbb{R}\) be map \[
    l_i(\alpha)=\left<\alpha,u_i\right> -\lambda_i
.\] And let \(\Delta^\circ \) be interior of \(\Delta\). Then \(\alpha\in\Delta^\circ\)
iff \(l_i(\alpha)>0,\forall\,i\). Let \[
    l_{\infty}(x)=\sum_{i=1}^{N}\left<\alpha,u_i\right> 
.\] Goal of this section is to prove
\begin{theorem}\label{thm:kahler-main}
    The restriction of \(\omega\) to \(\Phi^{-1}(\Delta^\circ )\) is \[
        \omega=\sqrt{-1}\partial\overline{\partial}\Phi^*
        \left(\sum_{i=1}^{N}\lambda_i \log l_i+l_\infty\right)
    .\] 
\end{theorem}

\subsection{Metric on Real Part}
Recall we defined the real part of \(M\), \(\Re(M)\). Also we can similarly define
\(\Re(Z)\), and we have natural projection \[
    \pi\colon \Re(Z)\longrightarrow \Re(M)
.\] With following properties:
\begin{theorem}\label{thm:real-part-covering}
    \(\pi\colon \Re(Z)\to \Re(M)\) is a \(2^{N-n}\)-sheeted covering. The group of
    deck transformations is given by \[
        \{w\in K:\exp(w)^2=1\}=K\cap
        \{w\in \mathbb{T}_{\mathbb{C}}^N:w_i=0\text{ or }\pi\sqrt{-1}\}
    .\] 
\end{theorem}
\begin{prop}
    The covering map is an isometry with respect to K\"ahler metric \(\eval{\omega_0}
    _{Z}\) and \(\omega\).
\end{prop}
\begin{proof}
    This is almost obvious by \[
        \iota^* \omega_0=\pi^* \omega
    \] and that \(\sigma\) interact nicely with \(\pi\).
\end{proof}

Recall that \(Z\) is characterized by quadratic equation \[
    \frac{1}{2}\sum_{i=1}^{N}|z_i|^2\beta^i+\lambda'=0
.\] Then by definition,
\(\Re(Z)\) is characterized by equation \[
    \frac{1}{2}\sum_{i=1}^N x_i^2\beta^i+\lambda'=0
.\] The induced metric by standard metric on \(\mathbb{C}^N\) writes as \[
    \eval{\omega_0}_{\Re(Z)}=\sum_{i=1}^N \dd{x_i}^2
.\] Restrict to one of \(2^N\) open orthants \[
    \eps_i x_i >0,\quad \eps_i=\pm 1
,\] and let \(s_i=x_i^2/2\), the metric becomes \[
    \omega_0=\frac{1}{2}\sum_{i=1}^N \frac{\dd{s_i}^2}{s_i}
.\] Denote \(Z_r^\eps\) be the intersection of \(\Re(Z)\) and the orthant associated to
\(\eps=(\eps_1,\ldots,\eps_N)\). In \(s\) coordinates, it is intersection of \[
    \{s_1>0,s_2>0,\ldots,s_N>0\}
\] and linear subspace \[
    \sum_{i=1}^{N}s_i \beta^i+\lambda'=0
.\] Moreover, the moment map \(\Phi_0\) becomes \[
    s\longmapsto \sum_{i=1}^{N}(s_i+\lambda_i)e^i
.\] Let \(\Psi=(\pi^*)^{-1}\circ \Phi_0\) restricted on \(Z_r^\eps\), then \(\Psi\)
is actually an diffeomorphism onto \(\Delta^\circ\). This makes \(\Delta^\circ\) a
Riemannian manifold with metric 
\begin{equation}\label{eq:metric-on-polytope}
    \omega_\Delta=\frac{1}{2}\sum_{i=1}^{N}\frac{\dd{l_i}^2}{l_i}
.\end{equation}
Note \(l_i\) is just \(s_i\) in \(x\) coordinate.

Recall that we have
\begin{equation}\label{eq:phi-pi-split}
    \Psi=\Phi\circ \pi
.\end{equation}
where \(\Phi\) is induced moment map on \(X=Z/K\). We claim that
\begin{lemma}
    \Cref{eq:phi-pi-split} splits the \(2^N\)-sheeted covering into a
    \(2^{N-n}\)-sheeted one and a \(2^n\)-sheeted one. Both are diffeomorphism on
    each connected component.
\end{lemma}
\begin{proof}
    Just notice how \[
    \xymatrix{
        0 \ar[r] & k \ar[r]^{\iota_*} & \mathbb{R}^N \ar[r]^{\pi_*} & \mathbb{R}^n
        \ar[r] & 0
    }\] induces when restrict to real part. (Detail fill later)
\end{proof}

\subsection{Proof of Main Result}

For the proof we need a few elementary facts abort K\"aehler structures on complex
tori. Denote \(T\) be the complex torus \[
    \mathbb{C}^n/2\pi\sqrt{-1}\mathbb{Z}^n
,\] and \(\mathbb{T}^n\hookrightarrow T\) as imaginary part, acting by ``left
multiplication''. Let \(\omega\) be a \(\mathbb{T}^n\) invariant K\"ahler form on
\(T\).
\begin{theorem}
    \(\omega\) is exact \(\iff \) the action by \(\mathbb{T}^n\) is Hamiltonian.
\end{theorem}

\begin{theorem}\label{thm:trivial-cohomology}
    The invariant Dolbeault cohomology groups \[
        H^{0,q}(T)^{\mathbb{T}^n}
    \] are 0 for \(q>0\).
\end{theorem}
\begin{proof}
    An invariant \(k\)-form can be written as \[
        \eta=\sum_{\card I=k}\eta_I(x)\dd{\overline{z_I}},
        \quad \eta_I\in C^\infty(\mathbb{R}^n)
    .\] Where \(\mathbb{R}^n\hookrightarrow T\) as real part.
    Then \[
        \iota^*\eta=\sum_{\card I=k}\eta_I(x)\dd{x_I}
    .\] So \(\iota\) induces bijection \[
        \iota^*\colon \Omega^{0,q}(T)^{\mathbb{T}^n}
        \longrightarrow\Omega^q(\mathbb{R}^n)
    .\]
\end{proof}

Further, we need the following two results:
\begin{theorem}\label{thm:metric-potential}
    The \(\mathbb{T}^n\) action is Hamiltonian \(\iff \omega\) has a \(\mathbb{T}^n\)
    invariant potential, \ie\ \[
        \omega=2\sqrt{-1}\partial\overline{\partial}F,
        \quad F\in C^\infty(\mathbb{R}^n)
    .\]
\end{theorem}
\begin{proof}
    If the action is Hamiltonian, then there exists a \(\mathbb{T}^n\) invariant
    1-form \(\eta\) \st\ \(\omega=\dd{\eta}\). Write \[
        \eta=\alpha+\overline{\alpha}
    \] where \(\alpha\in \Omega^{0,1}\) is anti-holomorphic part. Then \[
        \overline{\partial}\alpha=\partial\overline{\alpha}=0
    \] since \(\omega\) is (1,1)-form. And \[
        \omega=\dd{\eta}=\partial\alpha+\overline{\partial}\overline{\alpha}
    .\] By \cref{thm:trivial-cohomology}, there exists a \(\mathbb{T}^n\) invariant
    function \(G\in C^\infty(\mathbb{R}^n,\mathbb{C})\) \st\ \(\alpha=
    \overline{\partial}G\).
    Hence \[
        \omega=\partial\overline{\partial}G+\overline{\partial}\partial G
        =2\sqrt{-1}\partial\overline{\partial}\Im(G)
    .\] 
\end{proof}

Now assume the action is Hamiltonian, let \[
    \Phi\colon T\longrightarrow (\mathbb{R}^n)^*
\] be the moment map.
\begin{theorem}
    \(\Phi\) is the Legendre transform associated with \(F\) up to a additive constant.
    \ie\ \[
        \Phi(x+\sqrt{-1}y)=\eval{\dd{F}}_{x}+c
    .\] 
\end{theorem}
\begin{proof}
    By definition \[
        \dd{\Phi_i}=-\iota_{\pd{y^i}}\omega
    .\] On the other hand, by \cref{thm:metric-potential}, \[
        \omega=\sum_{i,j=1}^n \pdv{F}{x^i,x^j} \dd{x^i}\wedge \dd{y^j}
    .\] So \[
        \dd{\Phi_i}=-\iota_{\pd{y^i}}\omega=\dd(\pdv{F}{x^i})
    .\] Thus \[
    \Phi_i=\pdv{F}{x_i}+c_i
    .\] 
\end{proof}
\begin{remark}
    One can eliminate \(c\) by replacing \(F\) with \(F-c_i x^i\), which doesn't
    change the K\"ahler form.
\end{remark}

Now, we turn to the proof of main result.

Let \(T\) be the pre-image \(\Phi^{-1}(\Delta^\circ)\subset X_\Delta\), on which
\(\mathbb{T}_{\mathbb{C}}^n\) acts freely and faithfully. The real part \(X_r\)
intersects \(T\) in \(\mathbb{R}^n\times \{0,\pi\sqrt{-1}\}\). Fix a connected
component indexed by \(\eps \in \{\pm 1\}^n\), we have two metric on \(X_r^{\eps}\):
\begin{enumerate}[(a)]
\item The one induced by Euclidean metric of \(Z\subset \mathbb{C}^N\), 
\begin{equation}\label{eq:oemga-riemannian}
    \omega=\frac{1}{2}\Phi^*\left(\sum_{i=1}^N \frac{\dd{l_i}^2}{l_i}\right)
    \quad\text{(Riemannian metric)}
.\end{equation}
\item The one restricted from the invariant metric on \(T\) in above discussion,
    \begin{equation}\label{eq:omega-kahler}
    \omega=\sum_{i,j=1}^n \pdv{F}{x^i,x^j}\dd{x^i}\wedge\dd{y^j}
    \quad\text{(K\"ahler form)}
    .\end{equation}
\end{enumerate}
Where, to be clear, \(\{z_i\}_{i=1}^N\) be original coordinate on
\(\mathbb{C}^N\supset Z\); \(\{s_i=|z_i|^2/2\}_{i=1}^N\).
\(\{w_i=x_i+\sqrt{-1}y_i\}_{i=1}^n\) be coordinate on
\(T\cong\mathbb{T}_{\mathbb{C}}^n\); \(\{\alpha_i\}_{i=1}^n\) be standard coordinate
on \(\Delta\subset (\mathbb{R}^n)^*\). They are related by \[
    z_i=z^{(0)}\exp(w_i),\quad
    l_i=\alpha_j u^j_i-\lambda_i,\quad
    s_i=l_i\circ \Psi,\quad
    \Psi=\Phi\circ \pi
.\] 

To relate the two metric, first, in Riemannian form, (b) is \[
    \sum_{i,j=1}^n \pdv{F}{x_i,x_j} \dd{x_i}\dd{x_j} \text{ on }X_r^\eps
.\] Then, notice that (a) can be rewritten as
\begin{equation}\label{eq:potential-in-l}
    \sum_{i,j=1}^n \pdv{G}{\alpha_i,\alpha_j} \dd{\alpha_i}\dd{\alpha_j},\quad
    G=\frac{1}{2}\sum_{k=1}^N l_k(\alpha)\log l_k(\alpha)
.\end{equation}
Now let \[
    \sum_{i=1}^n \dd{x_i}\wedge \dd{\alpha_i}
\] be standard symplectic form on \(\mathbb{R}^{2n}\), and let \[
    \Gamma=\left\{(x,\alpha):\alpha=\dd{F}=\pdv{F}{x}\right\}
\] be the graph of Legendre transform of \(F\).

Then \(\Gamma\) is a Lagrangian submanifold of \(\mathbb{R}^{2n}\), and since \(L_F\)
is diffeomorphism, the differentials \[
    \dd{x_1},\ldots,\dd{x_n},\ \dd{\alpha_1},\ldots,\dd{\alpha_n}
\] are independent on \(\Gamma\). Moreover, the restriction to \(\Gamma\) of
quadratic differential \(\sum_i \dd{x_i}\dd{\alpha_i}\) can be written either as \[
    \sum_{i,j}\pdv{F}{x_i,x_j},\quad\text{or}\quad
    \sum_{i,j}\pdv{G}{\alpha_i,\alpha_j}
.\] Set \(x=x(\alpha)\), we can rewrite the differential as \[
    \frac{1}{2}\sum_{i,j}(\pdv{x_i}{\alpha_j}+\pdv{x_j}{\alpha_i})
    \dd{\alpha_i}\dd{\alpha_j}
.\] So \[
    \frac{1}{2}\sum_{i,j}(\pdv{x_i}{\alpha_j}+\pdv{x_j}{\alpha_i})
    =\pdv{G}{\alpha_i,\alpha_j}
.\] Since \(\Gamma\) is Lagrangian, \[
    \pdv{x_i}{\alpha_j}-\pdv{x_j}{\alpha_i}=0
.\] Hence \[
    \pdv{x_i}{\alpha_j}=\pdv{G}{\alpha_i,\alpha_j}
.\] That is, \[
    x_i=\pdv{G}{\alpha_i}+c_i
.\] Let \(H=G+\sum_i c_i \alpha_i\), we conclude that \[
    x=\dd_\alpha H=\pdv{H}{\alpha}
.\] So \(L_H\) is inverse of \(L_F\). So \(F\) must be the Legendre dual of \(H\),
\ie\ \[
    F(x)=\sum_i x_i \alpha_i-H(\alpha),\quad \text{evalulated at }
    \alpha=\pdv{F}{x}
.\] View as function of \(\alpha\), RHS is \[
    \sum_i \alpha_i \pdv{H}{\alpha_i}-H(\alpha)
    = \sum_i \alpha_i \pdv{G}{\alpha_i}-G(\alpha)
.\] Substitute \cref{eq:potential-in-l}, we get 
\begin{equation}\label{eq:F-in-G}
    F=\frac{1}{2}\Phi^*\left(\sum_{i=1}^N \lambda_i \log l_i+l_\infty\right)
.\end{equation}
Which finishes the proof.

\begin{remark}
    If \(\sum_i u_i=0\), the term \(l_\infty\) disappears. There is a nice geometric
    interpretation of this:
    
    Recall \(K\) be kernel of projection \(\pi\colon \mathbb{T}^N\to \mathbb{T}^n\),
    if \(\sum_i u_i=0\), then \(K\) contains the diagonal \[
        L=(\theta,\theta,\ldots,\theta)\subset \mathbb{T}^N
    .\] Then by the principle of ``reduction in stages'', \(X\) is the reduced
    space associated with Hamilton action of \(K/L\acts \mathbb{CP}^{N-1}\).
\end{remark}

\subsection{Curvature of the K\"ahler Structure}
Following notation above, recall that \[
    \omega=\sum_{i,j=1}^{n}\pdv{F}{x^i,x^j}\dd{x^i}\wedge \dd{y^j}
    =\frac{\sqrt{-1}}{2}\sum_{i,j}\pdv{F}{x^i,x^j}\dd{z^i}\wedge \dd{z^j}
.\] Denote by \(F_{ij}=\pdv{F}{x^i,x^j}\). Recall \(G\) is Legendre dual of \(F\),
so \[
    (F_{ij})=\dd(L_F)=\dd{L_G}^{-1}=G^{ij}
.\] We directly calculate the Ricci curvature:
\begin{align*}
    \mathrm{Ric}&=-\sqrt{-1}\partial\overline{\partial} \log\det\omega^n
    =-\sqrt{-1}\partial\overline{\partial}\log\det(F_{pq}) \\
    &=\sqrt{-1}\pd{z^i}\pd{\overline{z}^j}\log\det(G_{pq})\dd{z^i}\wedge\dd{z^j} \\
    &=\frac{\sqrt{-1}}{4}\pdv{}{x^i,x^j}\log\det(G_{pq})\dd{z^i}\wedge \dd{z^j} \\
    &=\frac{\sqrt{-1}}{4}\pd{x^i}\left(G^{pq}\pdv{G_{pq}}{x^j}\right)
    \dd{z^i}\wedge \dd{z^j}
.\end{align*}
Note that \[
    \pdv{x^i}{\alpha_j}=G_{ij},\quad \pdv{\alpha_i,x^j}=F_{ij}=G^{ij}
.\] So we can replace \(x\)-coordinate with \(\alpha\),
\begin{align*}
    \mathrm{Ric}&=\frac{\sqrt{-1}}{4}G^{il}\pd{\alpha_l}\left(
    G^{pq}G^{kj}\pdv{G_{pq}}{\alpha_k}\right)\dd{z^i}\wedge \dd{z^j} \\
    &=\frac{\sqrt{-1}}{4}G^{il}\pd{\alpha_l}(G^{pq}G^{kj}\pdv{G_{pk}}{\alpha_q})
    \dd{z^i}\wedge \dd{z^j} \\
    &=\frac{\sqrt{-1}}{4}G^{il}\pd{\alpha_l}(-G^{pq}\pdv{G^{kj}}{\alpha_{q}}G_{pk})
    \dd{z^i}\wedge \dd{z^j} \\
    &=-\frac{\sqrt{-1}}{4}G^{il}\pd{\alpha_l}\pdv{G^{kj}}{\alpha_k}
    \dd{z^i}\wedge \dd{z^j}
.\end{align*}
For scalar curvature, we have \[
    S=F^{ij}R_{ij}=-\frac{1}{2}\pdv{G^{kl}}{\alpha_k,\alpha_l}
.\] Now we discover the relation to extremal K\"ahler metric, which solves the
Euler-Lagrange equation of Calabi functional \[
    \int_M S^2
\] in a fixed K\"ahler class. The equation is given by \[
    \nabla^{1,0}S=(F^{ij}\pdv{S}{x^j}\pd{z^i})\equiv \text{ holomorphic v.f.}
.\] But coefficients of \(\pd{z^i}\) are real, so only possibility is \[
    F^{ij}\pdv{S}{x^j}=\pdv{S,\alpha_i}\equiv \text{ constant }
.\] \ie\ \(S\) is affine function in \(\alpha\).

\begin{example}
    For \(\mathbb{CP}^2\) as in \cref{eg:cp2}, we have \[
        l_1=\alpha_1,\ l_2=\alpha_2,\ l_3=1-\alpha_1-\alpha_2
    .\] Then
    \begin{align*}
        \omega&=G_{ij}\dd{\alpha_i}\dd{\alpha_j}
        =\frac{1}{2}\sum_{i=1}^3 \frac{\dd{l_i}^2}{l_i} \\
        &=\frac{1}{2}\left(\frac{\dd{\alpha_1}^2}{\alpha_1}
        +\frac{\dd{\alpha_2}^2}{\alpha_2}
        +\frac{\dd{\alpha_1}^2+2\dd{\alpha_1}\dd{\alpha_2}+\dd{\alpha_2}^2}
        {1-\alpha_1-\alpha_2}\right) \\
        &=\frac{1}{2}\frac{1}{1-\alpha_1-\alpha_2}\begin{bmatrix}
            \frac{1}{\alpha_1}(1-\alpha_2) & 1 \\
            1 & \frac{1}{\alpha_2}(1-\alpha_1)
        \end{bmatrix}
    .\end{align*} So \[
        (G^{ij})=2\begin{bmatrix}
            \alpha_1(1-\alpha_1) & -\alpha_1\alpha_2 \\
            -\alpha_1\alpha_2 & \alpha_2(1-\alpha_2)
        \end{bmatrix}
    .\] Hence \[
        S=-\frac{1}{2}\pdv{G^{ij}}{\alpha_i,\alpha_j}
        =2+1+1+2=6
    .\] 
\end{example}

\subsection{Blowing-up}\label{sec:blowing-up}

\begin{theorem}
    The moment map \(\Phi\colon M_\Delta\to \Delta\) maps the fixed points of
    \(\mathbb{T}^n\) bijectively onto vertices of \(\Delta\).
\end{theorem}

Now fix \(q\) be a fixed point and \(p\) be corresponding vertex of \(\Delta\).
\begin{theorem}
    Let \(v_1,\ldots,v_n\) be primitive representative of edges at \(p\), define
    \(\Delta_\eps\) be the polytope by cut-off \[
        \{p+\sum_j t_j v_j:t_j\ge 0,\sum_j t_j <\eps\}
    .\] Result \(\Delta_\eps\) has \(n\) new vertices \({p+\eps v_j}\)
    replacing \(p\).

    Then associated Delzant space \(M_\eps\) is blow-up at \(q\) of \(\eps\) amount
    on \(M=M_\Delta\).
\end{theorem}
\begin{proof}
First, let \(\Delta\) and \(\Delta'\) be two Delzant polytope that only differ from
change of \(\mathbb{Z}\)-basis.\ \ie\ \(\Delta=A\Delta'\) where \(A\) is an invertible
integer matrix. Then \(X_\Delta\) will be the same as \(X_{\Delta'}\), since it is
just a change of coordinate on \(\mathbb{T}^n\).

Now we can wlog assume \(p\) is the origin with edges \(e_1,\ldots,e_n\). We can
choose \(u_1,\ldots,u_n\) to be \(e_1,\ldots,e_n\) and
\(\lambda_1=\cdots =\lambda_n=0\). Then \[
    \Delta=\bigcap_{i=1}^N \left<\alpha,u_i\right> \ge 0
.\] We blow-up the polytope by cutting-off the corner at origin. To be precise, we
add a face with inner normal \(u_0=u_1+\cdots +u_n=(1,\ldots,1)\) and set
\(\lambda_0=\eps\). Denote the new polytope by \(\Delta_\eps\).

To characterize the toric manifold associated to \(\Delta_\eps\), only need to look
at the real part. For original \(X\) associated to \(\Delta\), \(X_r\) is
diffeomorphic to \(\mathbb{R}^n\) around \(q\), with each orthant is diffeomorphic 
to the corner of \(\Delta\) at \(p\). The transition map is \[
    \alpha_i=l_i(\alpha)=s_i=\frac{|z_i|^2}{2}=\frac{x_i^2}{2},\ i=1,\ldots,n
.\] After cutting-off the corner, a new relation is added as \[
    l_0(\alpha)=\frac{1}{2}(x_1^2+\cdots +x_n^2)-\eps\ge 0
.\] So a small disk centered at \(q\) is removed. Then, we also have a new relation,
\ie\ \(u_0=u_1+\cdots +u_n\), so we need to further do quotient of the group action
generated by \[
    (-\theta,\underbrace{\theta,\ldots,\theta}_{n},0,\ldots,0)\in k
.\] Restrict to the real part, \(\theta\) can only take \(0\) or \(\pi\), it is \[
    (-1,-1,\ldots,-1,1\cdots,1)\in K
.\] For points \(\not\in \) boundary of the disk, we have \(z_0\neq 0\), hence the
quotient does not take effect. For points on the boundary, \(z_0=0\), then we must
identify \((x_1,\ldots,x_n)\) and \((-x_1,\ldots,-x_n)\) to be the same point.
Thus, the result space is obtained by identify antipodal points of the boundary
of the removed disk, hence diffeomorphic to \(X_r\) blowing-up at \(q\). In the end,
notice that real part of the blowing-up is blowing-up of the real part. We see
\(X_{\delta_\eps}\) is diffeomorphic to \(X_\Delta\) blow-up at \(q\).

Conversely, if we start from blowing-up the manifold. \(X_\eps\) is obtained from
\(X\) by deleting \(q\) and replacing it with \(\mathbb{CP}^{n-1},
\eps\omega_{\mathrm{FS}}\). Note that if \(\Delta\) is the \(n\)-simplex in 
\((\mathbb{R}^n)^*\) spanned by origin and \(e_1,\ldots,e_n\), then \(X_\Delta=
\mathbb{CP}^n\). Using this fact, it's easy to see that the image of moment map
of action \(\mathbb{T}^n\acts \mathbb{CP}^{n-1}\) is \((n-1)\)-simplex generated
by \(R_{\eps}e_i,i=1,\ldots,n\). If \(\eps\) is small enough, metric around other
\(\mathbb{T}^n\)-fixed points is unchanged. So the moment polytope has the same
vertices as original one. We see that it must be the blowing-up of original polytope
at \(p\).

\end{proof}

\nocite{*}
\printbibliography{}

\end{document}
